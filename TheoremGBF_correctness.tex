\heading{Garbled Bloom filter in random oracle model (ROM). }
Fix $n,k,m \geq 0$ and let $\mathcal{S}=[\univ]^n$.  Then $\Pi_{\mathrm{garb}}= (\calQ,\Rep, \Qry)$ is defined as in Figure~\ref{fig:bf-and-garbled-bf} (right side).  The following result shows,

\begin{theorem}\label{thm:gbf-correctness}
Fix $k,m,n,r>0$, and let $\Pi_{\mathrm{garb}}= (\calQ,\Rep, \Qry)$ be the set-multiplicity data structure just described. For any adversary~$A$ that makes a total of~$q$ queries to the RO, and has time-complexity~$O(t)$,
\[
\AdvCorrect{\Pi_{\mathrm{garb}},\distr{\calS}{},r}{A} \leq  \epsilon +  \max\left(\dfrac{1}{\dbinom{m}{k}-(q-1)}\left(1-\dfrac{n}{2^\lambda}\right), \dbinom{q}{r}\left(\dfrac{1}{2^\lambda}\right)^r \right).
\]
\[
\AdvCorrect{\Pi_{\mathrm{garb}},\distr{\calS}{},r}{A}  \geq \epsilon + \left(\dfrac{1}{\dbinom{m}{k}-(q-1)}\right)
\]
Here, $\epsilon$ is the collision error(described in proof).\acnote{1)Upperbound of $\epsilon$ is loosely equal to the FP upperbound in a classical BF. 2) Recommended values of $\lambda$ in GBF paper = 80 or 256. 3) For time being, number of queries to RO is assumed to include queries to $\Test$.}\end{theorem}

\begin{proof}%[\Cref{thm:gbf-correctness}]
Define an adaptive attacker $A$ that has the following behavior. It lazily samples ~$k$ positions $i_1,i_2,\ldots,i_k$ from $[m]$, xor shares($\lambda$-bit strings) at these positions in ($M$) to get a candidate~$x$.  If $h_{1}(x)=i_1 \wedge h_2(x)=i_2 \wedge \cdots \wedge h_k(x) = i_k$, then either $x \in S$ or~$x$ is a false-positive. If $x$ is not a false-positive, it repeats the attack for a different set of $k$ positions. Let $A$ make $q$ queries to the RO; assume RO returns all $k$ hash values of each query at one go.

Let $\calF_1$ be the event that $x$ is a false-positive(in above attack).
 $$\Prob{\calF_1} = \Prob{h_{1}(x)=i_1 \wedge h_2(x)=i_2 \wedge \cdots \wedge h_k(x) = i_k : x \notin S}.\Prob{x \notin S}.$$
 The probability that $x$ is not one of the $n$ $\lambda$-bit strings is $1-\dfrac{n}{2^\lambda}$, and $ \Prob{h_{1}(x)=i_1 \wedge h_2(x)=i_2 \wedge \cdots \wedge\\ h_k(x) = i_k} = \dfrac{1}{\dbinom{m}{k}-(q-1)}$. It must be noted that the factor $q-1$ is due to lazy sampling of the $k$ positions. Putting together, $\Prob{\calF_1} =  \dfrac{1}{\dbinom{m}{k}-(q-1)} . \Big(1-\dfrac{n}{2^\lambda}\Big)$.  If adversary $A$ makes $q = \dbinom{m}{k}$ queries, it has very high chance of winning(assuming $\lambda$ is large) the correctness game for $r=1$.  Let $q = (r+1) \times q_r$. For $r > 1$, if adversary $A$ makes $q_r = \dbinom{m}{k}$ queries to find a single false-positive, and repeats this for $r$ times, it has high probability of winning the correctness game. The total queries that $A$ makes is $(r+1)\dbinom{m}{k} = O\Big(\dbinom{m}{k}\Big)$.

% Let $A$ be an adaptive adversary(different from $A$)\acnote{this attacker is similar to the adaptive attackers of previous proofs, where there is no known weakness of the filter in general} attacking $\Pi_{\mathrm{garb}}$, and
Suppose $A$ also finds false-positives without exploiting the inherent weakness of $\Pi_{\mathrm{garb}}$. For this case, let $D$ be the corresponding  zero-query adversary equivalent of $A$(shown in \Cref{fig:D}). Dong et. al. \cite{dong2013private} show that the upperbound on false-positive probability in a garbled Bloom filter for a zero-query adversary and $r=1$ is $\dfrac{1}{2^\lambda}$.   Using analyses from previous proofs, the probability that $A$ finds $r$ false-positives is upperbounded by  $\dbinom{q}{r}\left(\dfrac{1}{2^\lambda}\right)^r$. 

Let $\calF$ be the event of finding $r$ false-positives in $\Pi_{\mathrm{garb}}$. From above analysis, we have
\begin{align*}
\Prob{\calF} \leq  \max\left(\dfrac{1}{\dbinom{m}{k}-(q-1)}\left(1-\dfrac{n}{2^\lambda}\right), \dbinom{q}{r}\left(\dfrac{1}{2^\lambda}\right)^r \right)
\end{align*}

A collison in a garbled Bloom filter occurs when two or more distinct elements in a set have the same set of hash values. A collision results in false-negatives as only one of the elements is captured in the representation, and hence $\Qry$ returns 1 for this element, and 0 for others. Let $\epsilon$ be the probability of finding a false-negative in a garbled Bloom filter. As per Dong et. al. \cite{dong2013private}, the upper bound of $\epsilon$  is loosely same as the upper bound of false-positive probability of the classical Bloom filter. Since, the set $S$ is fixed at the start of the game, $\epsilon$ is more or less a contant.

Let $\calE$ be the event of finding $r$ errors in $\Pi_{\mathrm{garb}}$. Then $\Prob{\calE}$ is sum of the false-positive and false-negative probabilities. Also, $\Prob{\calE} = \AdvCorrect{\Pi_{\mathrm{garb}},\distr{\calS}{},r}{A}$ So, 
\begin{align*}
\AdvCorrect{\Pi_{\mathrm{garb}},\distr{\calS}{},r}{A} \leq \epsilon +  \max\left(\dfrac{1}{\dbinom{m}{k}-(q-1)}\left(1-\dfrac{n}{2^\lambda}\right), \dbinom{q}{r}\left(\dfrac{1}{2^\lambda}\right)^r \right)
\end{align*}
and,
\begin{align*}
\AdvCorrect{\Pi_{\mathrm{garb}},\distr{\calS}{},r}{A} & \geq \epsilon + \left(\dfrac{1}{\dbinom{m}{k}-(q-1)}\right)
\end{align*}


\end{proof}
