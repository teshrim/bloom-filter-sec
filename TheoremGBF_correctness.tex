\heading{Garbled Bloom filter in random oracle model (ROM). }
Fix $n,k,m \geq 0$ and let $\mathcal{S}=[\univ]^n$.  Then $\Pi_{\mathrm{garb}}= (\calQ,\Rep, \Qry)$ is defined as in Figure~\ref{fig:bf-and-garbled-bf} (right side).  The following result shows,

\begin{theorem}\label{thm:gbf-correctness}
Fix $r=1$, $k,m,n>0$, and let $\Pi_{\mathrm{garb}}= (\calQ,\Rep, \Qry)$ be the set-multiplicity data structure just described. For any adversary~$A$ that makes a maximum of~$\dbinom{m}{k}$ queries to the RO, and has time-complexity~$O(t)$,
\[
\AdvCorrect{\Pi_{\mathrm{garb}},\distr{\calS}{},r}{A}  \geq   \left(1 - e^{-n/m^k}\right)^{nr} + \dfrac{1}{\dbinom{m}{k}-n-(q-1)} . \Big(1-\dfrac{n}{2^\lambda}\Big)
\]

The RHS of above expression contains 2 components, $\Prob{\epsilon_{\fn}}= \left(1 - e^{-n/m^k}\right)^{nr}$ (false-negative probability), and $\Prob{\epsilon_{\fp}}=\dfrac{1}{\dbinom{m}{k}-n-(q-1)} . \Big(1-\dfrac{n}{2^\lambda}\Big)$ (false-positive probability). For large values of $m$ and $n$, $\Prob{\epsilon_{\fn}} \approx 0$. On the other hand, $A$ finds one false-positive with $\Prob{\epsilon_{\fp}} \approx 1$ by making $\dbinom{m}{k} - n$ queries at maximum.

\end{theorem}

\begin{proof}%[\Cref{thm:gbf-correctness}]
A collison in a garbled Bloom filter occurs when two or more distinct elements in a set have the same set of hash values. A collision results in false-negatives as only one of the elements is captured in the representation, and hence $\Qry$ returns 1 for this element, and 0 for others. Let $\epsilon_{\fn}'$ be the event of finding a false-negative in a garbled Bloom filter.  Let $x_1,x_2,\ldots,x_n$ be the elements in $S$. The probability that the hash values of $x_2,x_3,\ldots,x_n$ are not same as that of $x_1$ is $p = \left( 1-\dfrac{1}{\dbinom{m}{k}} \right)^{n-1}$. For $n >> 0$, $p \approx  e^{-n/m^k}$. The probability that $x_2,x_3,\ldots,x_n$ are false-negatives corresponding to $x_1$( $\Qry$ returns 1 for $x_1$, and 0 for others) is $1-p$. % Similarly, the probability that remaining $n-1$ elements are false negatives of $x_2,x_3,\ldots,x_n$ respectively is also $1-p$
. For $i \neq j$, the event of finding false-negatives of $x_i$ is independent of the event of finding false-negatives of $x_j$. Hence, $\Prob{\epsilon_{\fn}'} = \left( 1-p \right)^n$. Let $\epsilon_{\fn}$ be the event of finding $r$ false-negatives. Since $\epsilon_{\fn}'$ is independent of the index of a query, $\Prob{\epsilon_{\fn}} = \Prob{\epsilon_{\fn}'}^{r} = \left( 1-p \right)^{nr}$.

Define an adaptive attacker $A$ that has the following behavior. It makes $n$ queries to find $r=1$ false-negative. It halts if it finds 1 false-negative. Otherwise it does lazy sampling of ~$k$ positions $i_1,i_2,\ldots,i_k$ from $[m]$, xor shares($\lambda$-bit strings) at these positions in ($M$) to get a candidate~$x$.  If $h_{1}(x)=i_1 \wedge h_2(x)=i_2 \wedge \cdots \wedge h_k(x) = i_k$, then either $x \in S$ or~$x$ is a false-positive. If $x$ is not a false-positive, it repeats the attack for a different set of $k$ positions. Assume $A$ stops after finding 1 false-positive. If $A$ make $q_n$ queries to the RO, then $\max\left(q_n\right) = \dbinom{m}{k}$. Assume RO returns all $k$ hash values of each query at one go.

Let $\epsilon_{\fp}$ be the event that $A$ finds one false-positive. 
 $$\Prob{\epsilon_{\fp}} = \Prob{h_{1}(x)=i_1 \wedge h_2(x)=i_2 \wedge \cdots \wedge h_k(x) = i_k : x \notin S}.\Prob{x \notin S}.$$
 The probability that $x$ is not one of the $n$ $\lambda$-bit strings is $1-\dfrac{n}{2^\lambda}$, and $ \mathrm{Pr}[h_{1}(x)=i_1$ $\wedge h_2(x)=i_2 \wedge \cdots \wedge h_k(x) = i_k] = \dfrac{1}{\dbinom{m}{k}-n-(q-1)}$. It must be noted that the factor $q-1$ is due to lazy sampling of the $k$ positions; n is the set of hash indices corresponding to the elements of set, S. Putting together, $\Prob{\epsilon_{\fp}} =  \dfrac{1}{\dbinom{m}{k}-n-(q-1)} . \Big(1-\dfrac{n}{2^\lambda}\Big)$. 

If $\epsilon$ is the event of finding $r=1$ error, $\Prob{\epsilon} = \AdvCorrect{\Pi_{\mathrm{garb}},\distr{\calS}{},r}{A} =  \Prob{\epsilon_{\fn}} + \Prob{\epsilon_{\fp}}$($\epsilon_{\fn}$ and $\epsilon_{\fp}$ are mutually independent) and hence 
\[
\AdvCorrect{\Pi_{\mathrm{garb}},\distr{\calS}{},r}{A}  \geq \left(1 - e^{-n/m^k}\right)^{nr} + \dfrac{1}{\dbinom{m}{k}-n-(q-1)} . \Big(1-\dfrac{n}{2^\lambda}\Big)
\]

\end{proof}
