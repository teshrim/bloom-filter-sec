\section{Introduction}
\label{sec:intro}

Data structures are a fundamental tool used throughout computer
science. Typically, when the behavior of a data structure (e.g., its
running time or its error probability) is analyzed, the analysis is
performed under the assumption that the input/queries to the data
structure are chosen independently of the data structure itself, and
in particular independent of any internal randomness used by the
data structure. But, as recently highlighted by Naor and
Yogev~\cite{naor2015bloom}, there are many natural scenarios in
which the input/queries may be chosen \emph{adversarially} and
\emph{adaptively} based on partial information and/or prior
observations about the data structure.

In addition, standard applications of data structures assume that
the party querying the data structure is fully trusted, and hence
privacy concerns do not arise. In some scenarios, however, it may be
desirable to allow some party to pose (certain types of) queries to
a data structure, but unacceptable to leak the data in its entirety
to that same party. Existing definitions do not address privacy in
this sense at~all.

The concerns outlined above motivate a careful treatment of data
structures in an adversarial setting, where both \emph{soundness}
and \emph{privacy} are considered. In this work we initiate such a
treatment for abstract data structures, and also explore the
security of several specific data structures---that can all be
viewed as generalizations of Bloom
filters~\cite{bloom1970space}---supporting (approximate)
set-membership queries. Specifically, we show \ldots \todo{highlight
that we are able surface ways in which constructions
  are sensitive to use cases, i.e. schemes secure with secret hash
  functions but not with public hash functions}
\todo{Somewhere in the intro, might what to face the ``nothing hard''
  criticism head on.  Perhaps mention that many of the results follow
  a common pattern (which, for privacy, we tease out in a general
  theorem), and that often the most technically difficult part is
  connecting the adaptive setting to the zero-query setting.}

\heading{Comparison to~\cite{naor2015bloom}.} Naor and
Yogev~\cite{naor2015bloom} recently considered the problem of
adversarial soundness for Bloom filters and, indeed,
their worked served as inspiration for our own. Our work extends
their work significantly in several directions. For starters, we
consider abstract data structures (rather than only the case of
set-membership data structures), and consider privacy in addition to
soundness. Even with respect to the specific case of soundness for
set-membership data structures, our work offers several advantages
to the Naor-Yogev treatment:
\begin{itemize}
\item We explicitly distinguish between public and private
    portions of a data structure, something that was left
    ambiguous in~\cite{naor2015bloom}. This allows us, for
    example, to distinguish between the hash-function keys and
    the bit array when discussing classical Bloom filters.
\item The Naor-Yogev definition of soundness allows the
    adversary to make several set-membership queries, some of
    which may produce incorrect results, and the attacker then
    succeeds if it can output a \emph{fresh} query which leads
    to an error. We propose instead what we find to be a more
    natural definition in which the attacker succeeds if it can
    cause a certain number of incorrect results.
\item Naor and Yogev analyze a new Bloom-filter construction of
    their own design. In contrast, we are interested in
    analyzing practical constructions that have been proposed in
    prior work to understand how secure they are. In particular,
    we analyze Bloom filters in the random-oracle model.
    \jnote{Which are widely used?}
\end{itemize}
\tsnote{I think we can say a bit more here about what they show, and
  why their presentation is lacking, certainly as a work that anchors
  an important area of study.  This feels a bit weak, at the moment.
  For example, at the opening of this section we say they consider
  ``adversarial soundness for Bloom filters'', but I assert that it
  isn't clear that their syntax allows them to study Bloom filters at all.}
  \jnote{I'm a little concerned about walking the line between
  criticizing them and offending them. I also wouldn't say ``their
  syntax doesn't allow them to study Bloom filters''; is that
  because they don't deal with random oracles?}
We give a more detailed comparison to, and criticism
of~\cite{naor2015bloom} in Appendix~\ref{sec:xxx}. \jnote{Any
comparison/criticism should be in the Intro. The Appendix is just a
technical comparison of different styles of definition.}

\todo{Describe the BF variants we consider, and briefly why they are
  natural/interesting.}
\todo{Surface contribution of setting the stage for a bunch of future
  work: dynamic data-structures (including counting BFs, stable BFs,
  hierarchical BFs, count-min-sketches, etc.); security of common
  usages of BF and related data structures (e.g. CDNs and other
  distributed storage, find more); more generally bringing to the
  attention of the community the need to give proper cryptographic
  treatment to data structures (which may not have been conceived with
  security-critical applications in mind but have been co-opted for
  them (like BFs).}
%false-positive error, is a parameter of the Bloom filter.  (The
%values of~$\epsilon,m,k$, and $|\mathcal{S}|$ are related in a
%well-known way.) NY recast this soundness condition as a game in
%which an adversary~$A$ is given~$\mathcal{S}$ and an oracle for
%set-membership queries, and it attempts to find one or more false
%positives. The fact that~$A$ can \emph{adapt} its queries and its
%output based on what it learns from the set-membership oracle (but,
%as we will see, not upon the hash functions) is the key definitional
%contribution of~NY.

%Bloom filters have a broad range of practical applications, for
%example, in content-distribution networks, large database query
%processing, routing algorithms for peer-to-peer networks, protocols
%for establishing linkages between medical-record databases, fair
%routing of TCP packets, Bitcoin wallet synchonization~\cite{schnell2011novel,niedermeyer2014cryptanalysis,gervais2014privacy,nojima2009cryptographically,feng2001stochastic,reynolds2003efficient,byers2002informed,broder2004network}.

\ignore{We provide various security notions for these primitives...
FP with private, public representation coins; set privacy with
private, public representation coins; FP/set privacy in the
multi-representation setting (outsider attack, think CDNs); FP/set
privacy in the multi-representation setting with corruptions
(insider attack, think P2P networks) \tsnote{Those are just the ones
that come to mind now.  Could also consider authenticity of
representations, think CDN or P2P settings.}}

\heading{Related work.} \jnote{See the discussion of related work in
Naor-Yogev. We should cite [LN93]. I can't tell how relevant [MNS11]
is. [HW13] seems relevant, though I don't think it is captured by
our current definitions.}

\jnote{Discuss use of BFs and BF variants in a security-oriented
setting. Also some prior work formalizing BFs in an adversarial
environment. See commented out portion of the .tex}

\ignore{\heading{Related work: attacks}
\begin{itemize}
\item Niedermayer et al., ``Cryptanalysis of Basic Bloom Filters Used for Privacy-Preserving Record Linkage'', breaking privacy of secret-hash-function Bloom filters. \tsnote{Journal of Privacy and Confidentiality, 2014}
\item Gerbet, Kumar and Lauradoux, ``The power of evil choices in bloom filters''.
\tsnote{DSN'15: Looks like a real goldmine of related work!}
\item Crosby and Wallach, ``Denial of Service via Algorithmic Complexity Attacks'' \tsnote{Gives attacks on Squid}
\item Gao et al., ``Internet Cache Pollution Attacks and Countermeasures''
\end{itemize}}

\ignore{\heading{Related work: definitions(?)}
\begin{itemize}
\item Nojima and Kadobayashi, ``Cryptographically Secure Bloom Filters''.
\tsnote{Gives some security definitions for privacy. Quick scan, not super
clear what they achieve. The definition of client-privacy
 (Definition 1) for example, makes no sense to me.  Actually,
 likewise for server-privacy (Definition 2).  Both seem vague
 and thoroughly underspecified.}
\item Naor and Yogev
\item Eujin Goh, ``Secure Indexes'' \tsnote{A secure index can
    be used for set membership.  Builds a secret-key data
    structure (an Index) that allows searching for keyword~$w$
    if one holds the trapdoor $T_w$ for~$w$, where the trapdoor
    depends on the secret key.  Main construction uses
    traditional Bloom filters and a PRF.  Construction appears
    quite inefficient, needing a very long secret key, turning a
    keyword~$w$ into a bunch of PRF outputs, and then storing
    each of these PRF outputs in the BF.  Haven't read the full
    analysis; don't know if this was ever published. }
    \jnote{Never published. I think this work uses Bloom filters
    for encrypted search; I don't remember the paper having much
    to say about Bloom filters themselves.}
\end{itemize}

\heading{Related work: constructions}
\begin{itemize}
\item Bellovin and Cheswick, ``Privacy-Enhanced Searches Using
    Encrypted Bloom Filters''.
\item Kerschbaum , ``Public-Key Encrypted Bloom Filters with Applications to Supply Chain Integrity''.
\item S\"{a}rell\"{a} et al., ``BloomCasting: Security in Bloom Filter Based Multicast''.
\item Dong, Chen, Wen, ``When Private Set Intersection Meets Big Data: An Efficient and Scaleable Protocol'' \tsnote{``garbled bloom filters'', which actually store the set element by storing~$k$ xor-shares, one at each of the~$k$ hash indices (with care for reusing shares if hash collisions occur); also and``oblivious bloom intersection''}\tsnote{If the filter and the hash functions are public, there is a naive attack that works for some interesting parameters.  Pick~$k$ positions $i_1,i_2,\ldots,i_k$, xor strings at these positions to get a candidate~$x$.  If $h_{1}(x)=i_1 \wedge h_2(x)=i_2 \wedge \cdots \wedge h_k(x) = i_k$ then either $x \in S$ or~$x$ is a false-positive. (You don't have to do all~$k$ hashes, you can stop as soon as one fails to match the index.) Say you have a 1000-position filter ($m=1000$) and a 100-element set ($n=100$), and you are willing to tolerate a 1\% FP rate.  Then you'd set $k=4$ (a la Squid).  There are roughly $2^{35}$ subsets of 4 indices to consider, which is reasonable on a laptop these days.  (Also, since $n=100$ the expected number of trials is roughly $2^{28}$.) With the FP rate of 1\%, if you find a winner then almost certainly it is in the set.  Granted, this attack is naive, but notice that you \emph{cannot} mount it against a standard Bloom filter. Also notice that this attack might be significantly more efficient than brute force guessing over the universe of inputs (imagine that $U=\bits^{64}$), even if you know that the distribution deviates significantly from uniform. }
\item Tarkoma, Rothenberg, Lagerspetz ``Theory and Practice of Bloom Filters in Distributed Systems'' \tsnote{Great high-level coverage.  Only found preprint version though.}
\item Durham, Kantarcioglu, Xue, Kuzu, Malin ``Composite Bloom Filters for Secure Record Linkage'' \tsnote{Per-field BFs, sampled and composed into single BF that is then permuted by a secret random permtation.  No clear statement of the problem that is being solved.  Should pull full version and get details.}
\end{itemize}

\heading{Related work: tangential}
\begin{itemize}
\item Chang and Mitzenmacher ``Privacy Preserving Keyword Searches on Remote Encrypted Data''.
\item Mitzenmacher and Vadhan. ``Why Simple Hash Functions Work: Exploiting the Entropy in a Data Stream''.
\item Dodis et al. ``Fuzzy Extractors: How to Generate Strong Keys from Biometrics and Other Noisy Data'' \tsnote{Introduces ``secure sketches'', which is a representation of a single-element set that is information theoretically private (up to some function of the min-entropy of the element); only tangentially related to ``sketches'' as defined in the Bloom filter literature.}
\end{itemize}
}
