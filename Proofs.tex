\section{Proofs}
%F_{K}(\langle 1,x \rangle) + j\cdot F_{K}(\langle 2,x \rangle) \bmod m
\begin{proof}[\Cref{thm1}]
The proof of the theorem will use a game playing argument as shown in \Cref{fig:Game}. $\game{0}(A)$ exactly simulates $\ExpCorrect{\setprim_\mathrm{lin},r}{A}$ with hash functions $h_j(x) =  F_{K}(\langle 1,x \rangle) + j\cdot F_{K}(\langle 2,x \rangle) \bmod m$. So,

\begin{equation}
\AdvCorrect{\setprim_{\mathrm{lin}},r}{A} = \Prob{\game{0}(A)=1}\label{eq:0}
\end{equation}

 In $\game{1}(A)$, $F_{K}(\langle 1,\cdot \rangle)$ is replaced by a random function, whereas in $\game{2}(A)$, both $F_{K}(\langle 1,\cdot \rangle)$ and $F_{K}(\langle 2, \cdot \rangle)$ are replaced by random functions.

%\caption{Adversary $B_1, B_2$ simulating $\game{1}(A)$ and $\game{2}(A)$ respectively}\label{fig:BGame1}
\begin{figure}
\fpage{0.9}{
\hpagessl{0.5}{0.5}
{
$\adversaryv{B_1^{\calO}(S, \pubaux)}$\\
%$S \getsr A$\\
$\calC \gets \emptyset$; $\err \gets 0$\\
%$(\pubaux,\privaux) \getsr \Rep^{\HashOracle}(S)$\\
Run~$A(\pubaux)$\\
When $A$ asks $\TestOracle(\qry_x)$:\\
\nudge if $x \in \mathcal{C}$ then Return $\bot$\\
\nudge $\calC \gets \calC \cup x$\\
\nudge $a \gets \Qry^{\HashOracle}(\pubaux,\privaux, x)$\\
\nudge if $a \neq \qry_x(S)$ then \\
\nudge \nudge $\err \gets \err + 1$\\
\nudge Return~$a$\\
When $A$ halts: \\
\nudge if $\err  < r$ then Return 0\\
\nudge Return 1\\\\
%
\oracle{$\HashOracle(x)$}\\
for $i \in \{1,2,\ldots,k\}$\\
\nudge $v_i \gets F_{K}(\langle 1,x \rangle) + i \cdot \calO(\langle 2,x \rangle) \bmod m$\\
Return ${v_1,v_2,\ldots,v_k}$
}
{
$\adversaryv{B_2^{\calO}(S, \pubaux)}$\\
%$S \getsr A$\\
$\calC \gets \emptyset$; $\err \gets 0$\\
%$(\pubaux,\privaux) \getsr \Rep^{\HashOracle}(S)$\\
Run~$A(\pubaux)$\\
When $A$ asks $\TestOracle(\qry_x)$:\\
\nudge if $x \in \mathcal{C}$ then Return $\bot$\\
\nudge $\calC \gets \calC \cup x$\\
\nudge $a \gets \Qry^{\HashOracle}(\pubaux,\privaux, x)$\\
\nudge if $a \neq \qry_x(S)$ then \\
\nudge \nudge $\err \gets \err +1$\\
\nudge Return~$a$\\
When $A$ halts: \\
\nudge if $\err  < r$ then Return 0\\
\nudge Return 1\\\\
%
\oracle{$\HashOracle(x)$}\\
$a \getsr [m]$\\
if $T[x] \neq \undefined$\\
\nudge $a \gets T[x]$\\
$T[x] \gets a$\\
for $i \in \{1,2,\ldots,k\}$\\
\nudge $v_i \gets \calO(\langle 1,x \rangle) + i\cdot a$\\
Return ${v_1,v_2,\ldots,v_k}$
}
}
\caption{\Cref{thm1}:Adversary $B_1, B_2$ simulating $\game{1}(A)$ and $\game{2}(A)$ respectively}\label{fig:BGame1}
\end{figure}

The PRF security notion is shown in \Cref{fig:prf}. Let
prf-adversaries $B_1, B_2$ be as shown in \Cref{fig:BGame1}.
Adversary~$B_1$ is constructed so that, if its challenge bit~$b=1$,
then it exactly simulates $\game{0}$ for~$A$; if $b=0$, it
simulates~$\game{1}$.  Similarly, adversary~$B_2$ simulates~$\game{1}$
if its challenge bit~$b=1$, and $\game{2}$ otherwise.  We can conclude
that 
\begin{align*}
\Prob{\ExpPRF{F}{B_1} = 1\,|\,b=1} &= \Prob{\game{0}(A)=1} \\%\label{eq:m1}\\
\Prob{\ExpPRF{F}{B_1} = 1\,|\,b=0} &= 1-\Prob{\game{1}(A)=1} \\%\label{eq:m3}\\
\Prob{\ExpPRF{F}{B_2} = 1\,|\,b=1} &= \Prob{\game{1}(A)=1} \\%\label{eq:m2}\\
\Prob{\ExpPRF{F}{B_2} = 1\,|\,b=0} &= 1-\Prob{\game{2}(A)=1} \\%\label{eq:m4}
\end{align*}
%
\noindent
By a standard conditioning of the PRF advantage, we have 
\begin{equation*}
\Prob{\game{0}(A)=1} = \AdvPRF{F}{B_1} +\AdvPRF{F}{B_2} + \Prob{\game{2}(A)=1}%\label{eq:13}
\end{equation*}


In $\game{3}(A)$, we replace the random functions used in
$\HashOracle$ (in $\game{2}(A)$) with lazy sampling of elements from
$[m]$.  This change does not alter the behavior of the oracle, as the
random functions are (by the no-pointless-queries assumption on~$A$) always called on a new
value of~$x$.
%
Also, the $\Test$ oracle now sets $\bad$ to $\true$ if the
queries indexed by the set $\mathcal{I}$ fail to be false-positives,
or if any of the remaining queries are false-positives. It can be
observed that the $\bad$ flag has no impact on the output of the
$\Test$, nor does it affect  the value returned
by~$\game{3}$. Thus we have $\Prob{\game{2}(A)=1} =
\Prob{\game{3}(A)=1}$ and
\[
\Prob{\game{3}(A)=1 \wedge \game{3}(A): \neg\bad} = \Prob{\game{3}(A)=1} \Prob{\game{3}(A) : \neg\bad}%\label{eq:1} \\
\]

We note that $\game{3}$ and $\game{4}$ are
``identical-untill-$\bad$''~\cite{BeRo-gameplaying}.  In particular,
in $\game{4}$, $A$ returns 0 when $\game{3}$ would have returned 1 (i.e.,
the adversary outputs~$r$ false-positives, but $\bad=\true$).  
Moreover,  $\game{4}(A)$ returning 1 implies that $\bad$ is never set to $\true$. Therefore,
$\Prob{\game{4}(A)=1} = \Prob{\game{3}(A)=1 \wedge \neg\bad} =
\Prob{\game{3}(A)=1} \Prob{\game{3}(A) : \neg\bad}$, by our previous observation.

Before continuing the proof, a bit of intuition for the introduction
of the $\bad$-flag in games~$\game{3},\game{4}$.   Intuitively, if the index
set~$\mathcal{I}$ exactly predicts which of the $\Test$ queries will be
false-positives (and, conversely, which will not be), then a
zero-query FP-finding adversary can ``guess'' which of~$A$'s queries require it
to return 1 when simulating the $\Test$, and which require it to
return 0.  In this way, no queries ever actually need to be made.
Moving to a zero-query setting will allow us to connect to classical
results on the FP-rate of Bloom filters; more in a moment. 

Now, consider $\Prob{\game{3}(A) : \neg\bad}$.  The indicated event
happens iff the $r$-set~$\calI$ contains exactly the indices of
$\Test$-queries that are false-positives.  Since the value
of~$\bad$ does not alter the responses of~$\Test$, $\err$'s update
and the set~$\calI$ are independent.
Thus $\bad$ remains $\false$ after~$A$ halts with probability
$1/\dbinom{q}{r}$, and we have 
\begin{equation*}
\Prob{\game{4}(A)=1} = \frac{1}{\dbinom{q}{r}}\Prob{\game{3}(A)=1}
\end{equation*}

Recall that adversary~$A$ wins $\ExpCorrect{\setprim_{\mathrm{lin}},r}{}$ by finding~$r$ false positives. We can assume, without loss of generality,  that~$A$ halts immediately upon finding its $r$-th false positive. % We can also assume that every point in $\calZ$ has already been queried to $\Test$.  This latter assumption is not without loss, but can be  easily enforced (in the original security definitions) by counting as part of the number of queries~$q$, the points in $\mathcal{Z}$. 
%
%Under these assumptions,
So, if $\game{4}(A)=1$, then there is a zero-query adversary~$D$ that finds~$r$ false positives against $\tilde{\setprim}_{\mathrm{lin}}$, which is $\setprim_{\mathrm{lin}}$ but with $F_{K_1}$ and $F_{K_2}$ replaced by random functions $\rho_1$ and $\rho_2$, respectively.  We give this~$D$ in \Cref{fig:D}. Intuitively, adversary~$D$ guesses which~$r$ of~$A$'s queries will be false positives; it simulates the~$\Test$ oracle by returning 1 on the guessed queries, and 0 otherwise; and it halts after finding $r$ FP's. It must be noted that, by the no-pointless queries assumption and the fact that the hash functions in $D$ are instantiated with random functions, $\pubaux$ is information theoretically independent of elements not in $S$. Thus we have
$\Prob{\game{4}(A)=1} =\Prob{\ExpCorrect{\tilde{\setprim}_{\mathrm{lin}},r}{D} = 1}$.

Note that $\tilde{\setprim}_{\mathrm{lin}}$ is the scheme that is analyzed in the classical Bloom filter literature ---the hash functions are modeled as independent random functions, and the parameters $k,m,n$  are set to achieve a target maximum false-positive probability (or rate).  Kirsch and Mitzenmacher~\cite{KM} show that the upperbound on false-positive probability for a zero-query adversary and $r=1$ is $\left( (1-e^{-kn/m})^k + O(1/n) \right)$.  For $r>1$, we observe that~$A$'s queries $x_1,x_2,\ldots,x_q$ are distinct and not elements of the original set~$S$, since we assume no pointless queries. Let $\calF_i$ be the event that~$x_i(i \in \mathcal{I})$ is a false positive.  Since $\game{4}$ (equivalently $\tilde{\setprim}_{\mathrm{lin}}$) use independent random functions, the events $\calF_1,\calF_2,\ldots,\calF_r$ are independent. 
\[
\Prob{\ExpCorrect{\tilde{\setprim}_{\mathrm{lin}},r}{D} = 1}\leq \left( (1-e^{-kn/m})^k + O(1/n) \right)^r 
\]

To summarize, we have 
\[
\AdvCorrect{\setprim_{\mathrm{lin}},r}{A}  \leq  \AdvPRF{F}{B_1} +\AdvPRF{F}{B_2} +\dbinom{q}{r}\left( (1-e^{-kn/m})^k + O(1/n) \right)^r
\]
which is the bound claimed in the theorem statement. %\hfill \qed

%\caption{Game playing argument}\label{fig:Game}
\begin{figure}
\fpage{.99}{
\hpagessl{.52}{.5}
{
\underline{\game{0}(A)}\\
$K \getsr \calK $\\
$S \getsr A$; $\calC \gets \emptyset$; $\err \gets 0$\\
$(\pubaux,\privaux) \getsr \Rep^{\HashOracle}(S)$\\
$\bot \getsr A^{\TestOracle}(\pubaux)$\\
if $\err  < r$ then Return 0\\
Return 1\\
%
\oracle{$\TestOracle(\qry_x)$}\\
if $x \in \mathcal{C}$ then Return $\bot$\\
$\calC \gets \calC \cup x$\\
$a \gets \Qry^{\HashOracle}(\pubaux,\privaux, x)$\\
if $a \neq \qry_x(S)$ then $\err \gets \err + 1$\\
Return~$a$\\
%
\oracle{$\HashOracle(x)$}\\
for $i = 1$ to $k$\\
\nudge $h_i(x) = F_{K}(\langle 1,x \rangle) + i\cdot F_{K}(\langle 2,x \rangle) \bmod m$\\
Ret $\left(h_1(x),\ldots,h_k(x)\right)$
}
{
\underline{\game{1}(A)}\\
$K \getsr \calK $, $\rho \getsr \Func{\univ,[m]}$\\
$S \getsr A$; $\calC \gets \emptyset$; $\err \gets 0$\\
$(\pubaux,\privaux) \getsr \Rep^{\HashOracle}(S)$\\
$\bot \getsr A^{\TestOracle}(\pubaux)$\\
if $\err  < r$ then Return 0\\
Return 1\\
%
\oracle{$\TestOracle(\qry_x)$}\\
if $x \in \mathcal{C}$ then Return $\bot$\\
$\calC \gets \calC \cup x$\\
$a \gets \Qry^{\HashOracle}(\pubaux,\privaux, x)$\\
if $a \neq \qry_x(S)$ then $\err \gets \err + 1$\\
Return~$a$\\
%
\oracle{$\HashOracle(x)$}\\
for $i=1$ to $k$\\
\nudge $h_i(x) = F_{K}(\langle 1,x \rangle) + i \cdot \rho(x) \bmod m$\\
Ret $\left(h_1(x),\ldots,h_k(x)\right)$
}
}
\fpage{.99}{
\hpagessl{.52}{.5}
{
\underline{\game{2}(A)}\\
$\rho_1,\rho_2 \getsr \Func{\univ,[m]}$\\
$S \getsr A$; $\calC \gets \emptyset$; $\err \gets 0$\\
$(\pubaux,\privaux) \getsr \Rep^{\HashOracle}(S)$\\
$\bot \getsr A^{\TestOracle}(\pubaux)$\\
if $\err  < r$ then Return 0\\
Return 1\\
%
\oracle{$\TestOracle(\qry_x)$}\\
if $x \in \mathcal{C}$ then Return $\bot$\\
$\calC \gets \calC \cup x$\\
$a \gets \Qry^{\HashOracle}(\pubaux,\privaux, x)$\\
if $a \neq \qry_x(S)$ then $\err \gets \err + 1$\\
Return~$a$\\\\
%
\oracle{$\HashOracle(x)$}\\
for $i=1$ to $k$\\
\nudge $h_i(x) = \rho_1(x) + i \cdot \rho_2(x) \bmod m$\\
Ret $\left(h_1(x),\ldots,h_k(x)\right)$
}
{
\underline{{$\game{3}(A)$},\fbox{$\game{4}(A)$}}\\
$c\gets 0$, $\bad \gets \false$, $\err \gets 0$;  $\calC \gets \emptyset$\\
$S \getsr A$; $\mathcal{I}\getsr [\{1,2,\ldots,q\}]^r$\\
$(\pubaux,\privaux) \getsr \Rep^{\HashOracle}(S)$\\
$\bot \getsr A^{\TestOracle}(\pubaux)$\\
if $\bad = \true$ then \fbox{Return 0}\\
if $\err  < r$ then Return 0\\
Return 1\\
%
\oracle{$\TestOracle(\qry_x)$}\\
if $x \in \mathcal{C}$ then Return $\bot$\\
$\calC \gets \calC \cup x$\\
$c \gets c+1$\\
$v \gets \Qry^{\HashOracle}(\pubaux,\privaux, x)$\\
if $c \in \mathcal{I}$ and $v = \qry_x(S)$ then \\
\nudge $\bad \gets \true$ \\
if $c \not\in \mathcal{I}$ and $v \neq \qry_x(S)$ then \\
\nudge $\bad \gets \true$\\
if $v \neq \qry_x(S)$ then $\err \gets \err +1$\\
Return~$v$\\
%
\oracle{$\HashOracle(x)$}\\
$a,b \getsr [m]$\\
for $i = 1$ to~$k$\\
\nudge $v_i = a+i \cdot b \bmod m$\\
Return $\left(v_1,\ldots,v_k\right)$
}
}
\caption{\Cref{thm1}: Game playing argument}\label{fig:Game}
\end{figure}


%\caption{PRF game}\label{fig:prf}
\begin{figure}
\centering
\fpage{0.55}{
\hpagess{0.5}{0.35}
{
$\experimentv{\ExpPRF{F}{B}}$\\
$K \getsr \calK $\\
$\rho \getsr \Func{\univ,[m]}$\\
$b \getsr \bits$\\
$b' \getsr B^\calO$\\
Return $[b = b']$\\
}
{
$\oracle{\calO \smallskip(x)}$\\
if $ b = 1$ then\\
\nudge Return $F_K(x)$\\
Return $\rho(x)$\\
}
}
\caption{PRF security notion}\label{fig:prf}
\end{figure}

%\caption{Non-adaptive adversary $D$ simulating $\game{4}(A)$} \label{fig:D}
\begin{figure}
\centering
\fpage{0.35}
{
$\adversaryv{D(S, \pubaux)}$\\[4pt]
$c \gets 0$\\%, $\calX \gets \emptyset$\\
$\calC \gets \emptyset$; $\err \gets 0$\\
$\mathcal{I}\getsr [\{1,2,\ldots,q\}]^r$\\
Run $A(\pubaux)$\\
When $A$ asks $\TestOracle(\qry_x)$:\\
\nudge if $x \in \mathcal{C}$ then Return $\bot$\\
\nudge $\calC \gets \calC \cup x$\\
\nudge $c \gets c+1$\\
\nudge $r \gets 0$\\
\nudge if $c \in \mathcal{I}$\\
\nudge \nudge $r \gets 1$\\
\nudge \nudge $\err \gets \err + 1$\\
\nudge Return $r$\\
When $A$ halts:\\
\nudge Return $\bot$
}
\caption{Zero-query adversary $D$} \label{fig:D}
\end{figure}	

\end{proof}



\tsnote{Jon: ignore these for now.  This is Animesh's workspace.}
\newcommand{\FK}{F_K(\langle j,x \rangle)}
\newcommand{\rhoK}{\rho(\langle j,x \rangle)}
\newcommand{\OK}{\calO(\langle j,x \rangle)}

\begin{proof}[\Cref{thm2}]
We use a game playing argument (shown in \Cref{fig:2TGame}) similar to \Cref{thm1} to prove this theorem. In \game{0}, adversary attacks the correctness game with  hash functions $h_j(x) = \FK $, and hence exactly simulates $\ExpCorrect{\Pi_{\mathrm{ds}},\distr{\calS}{},r}{A}$.
\begin{equation}
\AdvCorrect{\Pi_{\mathrm{ds}},\distr{\calS}{},r}{A} = \Prob{\game{0}(A)=1}\label{eq:2T0}
\end{equation}

Let $B$ be a prf-adversary as shown in \Cref{fig:2TD}. Based on its challenge bit $b$, it exactly simulates $\game{0}$ or $\game{1}$. So,
\begin{align*}
\Prob{\ExpPRF{F}{B} = 1\,|\,b=1} &= \Prob{\game{0}(A)=1} \\%\label{eq:m1}\\
\Prob{\ExpPRF{F}{B} = 1\,|\,b=0} &= 1-\Prob{\game{1}(A)=1}
\end{align*}

By a standard conditioning of PRF advantage, we have

\begin{equation*}
\Prob{\game{0}(A)=1} = \AdvPRF{F}{B} + \Prob{\game{1}(A)=1}%\label{eq:13}
\end{equation*}

In $\game{1}$, $A$ attacks a different filter $\tilde{\Pi}_\mathrm{ds}$, in which $F_K$ is replaced by a random function. $\game{2}$ and $\game{3}$ on the other hand use lazy sampling in $\Hash$ to select random values from $[m]$. It can be observed that, except  $\Hash$, $\game{2}$ and $\game{3}$ are identical to $\game{3}$ and $\game{4}$ of previous theorem. Since, $\Hash$ in all of these games returns random values, the analyses and intermediate results of \Cref{thm1} apply to this theorem as well. Therefore,
\begin{align*}
\Prob{\game{1}(A)=1} = \Prob{\game{2}(A)=1}\\
\Prob{\game{3}(A)=1} = \frac{1}{\dbinom{q}{r}}\Prob{\game{2}(A)=1}
\end{align*}

Let $D$ be a zero-query adversary (shown in \Cref{fig:D}) that finds ~$r$ false positives against $\tilde{\Pi}_{\mathrm{ds}}$. Using explanations from previous theorem, if $A$ wins $\game{3}$, then $D$ wins $\ExpCorrect{\tilde{\Pi}_{\mathrm{ds}},\distr{\calS}{},r}{\cdot}$ game. So, 
\begin{equation*}
\AdvCorrect{\tilde{\Pi}_{\mathrm{ds}},\distr{\calS}{},r}{D} = \Prob{\game{3}(A) = 1}\label{eq:2T8}
\end{equation*}

Note that like $\tilde{\Pi}_{\mathrm{lin}}$, $\tilde{\Pi}_{\mathrm{ds}}$ is the scheme that is analyzed in the classical Bloom filter literature. So, for a zero-query adversary $D$ and $r=1$, the upper bound on false-positive probability of a classical Bloom filter applies to $\tilde{\Pi}_{\mathrm{ds}}$ as well. For $r > 1$, based on the assumption that $A$ makes no pointless queries, the queries made by $A$ are distinct and not elements of the original set $S$. Also, since $\game{3}$ uses independent random functions, the event of getting a false positive is independent of the index of the query. So,
\begin{equation*}
\AdvCorrect{\tilde{\Pi}_{\mathrm{ds}},\distr{\calS}{},r}{D} =   {\dbinom{q}{r}} \left( (1-e^{-kn/m})^k + O(1/n) \right)^r
\end{equation*}
\noindent
To summarize, we have
\[
\AdvCorrect{\Pi_{\mathrm{ds}},\distr{\calS}{},r}{A} \leq  \AdvPRF{F}{B}  + {\dbinom{q}{r}} \left( (1-e^{-kn/m})^k + O(1/n) \right)^r
\]
which is the bound claimed in the theorem statement. %\hfill \qed

%\caption{Game playing argument}\label{fig:Game}
\begin{figure}
\fpage{.9}{
\hpagessl{.45}{.5}
{
\underline{\game{0}(A)}\\
$K \getsr \calK $\\
$S \getsr A$; $\calC \gets \emptyset$; $\err \gets 0$\\
$(\pubaux,\privaux) \getsr \Rep^{\HashOracle}(S)$\\
$\bot \getsr A^{\TestOracle}(\pubaux)$\\
if $\err  < r$ then Return 0\\
Return 1\\\\
%
\oracle{$\TestOracle(\qry_x)$}\\
if $x \in \mathcal{C}$ then Return $\bot$\\
$\calC \gets \calC \cup x$\\
$a \gets \Qry^{\HashOracle}(\pubaux,\privaux, x)$\\
if $a \neq \qry_x(S)$ then $\err \gets \err + 1$\\
Return~$a$\\\\
%
\oracle{$\HashOracle(x)$}\\
for $j = 1$ to $k$\\
\nudge $h_j(x) =\FK $\\
Return $\left(h_1(x),\ldots,h_k(x)\right)$
}
{
\underline{\game{1}(A)}\\
$\rho \getsr \Func{\univ,[m]}$\\
$S \getsr \distr{\calS}{}$; $\calC \gets \emptyset$; $\err \gets 0$\\
$(\pubaux,\privaux) \getsr \Rep^{\HashOracle}(S)$\\
$\bot \getsr A^{\TestOracle}(\pubaux)$\\
if $\err  < r$ then Return 0\\
Return 1\\\\
%
\oracle{$\TestOracle(\qry_x)$}\\
if $x \in \mathcal{C}$ then Return $\bot$\\
$\calC \gets \calC \cup x$\\
$a \gets \Qry^{\HashOracle}(\pubaux,\privaux, x)$\\
if $a \neq \qry_x(S)$ then $\err \gets \err + 1$\\
Return~$a$\\\\
%
\oracle{$\HashOracle(x)$}\\
for $j=1$ to $k$\\
\nudge $h_j(x) = \rhoK $\\
Return $\left(h_1(x),\ldots,h_k(x)\right)$
}
}
\fpage{.9}{
\hpagessl{.45}{.5}
{
\underline{\game{2}(A)\fbox{\game{3}(A)}}\\
$c\gets 0$, $\bad \gets \false$, $\err \gets 0$;  $\calC \gets \emptyset$\\
$S \getsr \distr{\calS}{}$; $\mathcal{I}\getsr [\{1,2,\ldots,q\}]^r$\\
$(\pubaux,\privaux) \getsr \Rep^{\HashOracle}(S)$\\
$\bot \getsr A^{\TestOracle}(\pubaux)$\\
if $\bad = \true$ then \fbox{Return 0}\\
if $\err  < r$ then Return 0\\
Return 1
}
%
{
\oracle{$\TestOracle(\qry_x)$}\\
if $x \in \mathcal{C}$ then Return $\bot$\\
$\calC \gets \calC \cup x$\\
$c \gets c+1$\\
$v \gets \Qry^{\HashOracle}(\pubaux,\privaux, x)$\\
if $c \in \mathcal{I}$ and $v = \qry_x(S)$ then \\
\nudge $\bad \gets \true$ \\
if $c \not\in \mathcal{I}$ and $v \neq \qry_x(S)$ then \\
\nudge $\bad \gets \true$\\
if $v \neq \qry_x(S)$ then $\err \gets \err +1$\\
Return~$v$\\\\
%
\oracle{$\HashOracle(x)$}\\
for $j = 1$ to~$k$\\
\nudge $v_j \getsr [m]$\\
Return $\left(v_1,\ldots,v_k\right)$
}
}
\caption{\Cref{thm2}: Game playing argument}\label{fig:2TGame}
\end{figure}

%\caption{PRF adversary $B$ simulating $\game{1}(A)$ and $\game{2}(A)$
\begin{figure}
\centering
\fpage{0.5}{
$\adversaryv{B^{\calO}(S, \pubaux)}$\\
%$S \getsr A$\\
$\calC \gets \emptyset$; $\err \gets 0$\\
%$(\pubaux,\privaux) \getsr \Rep^{\HashOracle}(S)$\\
Run~$A(\pubaux)$\\
When $A$ asks $\TestOracle(\qry_x)$:\\
\nudge if $x \in \mathcal{C}$ then Return $\bot$\\
\nudge $\calC \gets \calC \cup x$\\
\nudge $a \gets \Qry^{\HashOracle}(\pubaux,\privaux, x)$\\
\nudge if $a \neq \qry_x(S)$ then \\
\nudge \nudge $\err \gets \err +1$\\
\nudge Return~$a$\\
When $A$ halts: \\
\nudge if $\err  < r$ then Return 0\\
\nudge Return 1\\\\
%
%
\oracle{$\HashOracle(x)$}\\
\nudge Return $\OK$, for $j\in[1,k]$\\
}
\caption{\Cref{thm2}: PRF adversary $B$ simulating $\game{1}(A)$ and $\game{2}(A)$} \label{fig:2TD}
\end{figure}	


\end{proof}

\acnote{To be verified before moving the theorem to section 4}
\heading{Classical Bloom filter in random oracle model (ROM). }
As discussed earlier, the classical Bloom filter (BF) is an instantiation of a set multiplicity data structure and admits errors in form of false  positives (FP) only. In the ROM, $\Rep$ queries the random oracle to get the hash indices of elements of set $S$. It then computes the $m$-bit array, $M$ and outputs $\privaux=\emptystring$, $\pubaux = \emptystring$ and $M$ as the representation of
the Bloom filter. The $\Qry$ oracle uses the random oracle in a similar way. For any query $q$, it gets the hash indices $h_1, \ldots, h_k$ from the random oracle and returns 1 if $M[h_i(x)]=1$ for all $i\in [k]$, and 0 otherwise.
\begin{theorem}\label{thm3}
Fix $k,m,n,r>0$. Let $A$ be an adaptive adversary (for attacking BF) asking $q$ queries to the random oracle. Let $t$ be the running time of $A$ given these queries. Then, 
\[
\AdvCorrect{{BF},\distr{\calS}{},r}{A} \leq  {\dbinom{q}{r}} \left( (1-e^{-kn/m})^k + O(1/n) \right)^r\,.
\]
\end{theorem}

\begin{proof}[\Cref{thm3}]
The proof of this theorem will follow along the lines of \Cref{thm2}.
\end{proof}

%\begin{theorem}\label{thm3}
A classic Bloom filter is private in random oracle model(ROM), i.e. an adaptive adversary $A$ has low probability of winning the $\ExpPrivPubHashBB{B}{\distr{\univ}{n},A}$ game in ROM. For $q$ queries and min-entropy ($\mu$) of $\distr{\univ}{n}$,

\begin{equation}
\AdvPrivPubHashBB{B}{A} \leq  \frac{q}{2^\mu} + \frac{n}{\abs{\univ}}
\end{equation}

\end{theorem}

\begin{proof}[\Cref{thm3}]
As shown in \Cref{fig:GameT3}, $\game{0}(A)$ and $\game{1}(A)$ exactly simulate $\ExpPrivPubHashBB{B}{\distr{\univ}{n}, A}$ in ROM, except in $\game{1}(A)$, the random oracle always returns $k$ uniformly random values for any $x \in \univ$.
\begin{equation}
\AdvPrivPubHashBB{B}{A} = \Prob{\game{0}(A)=1}\label{eq:0T3}
\end{equation}
%\caption{Game playing argument}\label{fig:Game}
\begin{figure}
\centering
\fpage{.6}{
\hpagess{.4}{.45}
{
$\game{1}(A)$\\
\fbox{$\game{0}(A)$\\
}
$\\ \calS \getsr \distr{\univ}{n}$\\
$(M,\tau) \getsr \Rep^{\HashOracle}(\calS)$\\
$x \getsr A^{\MemOracle,\QryOracle,\HashOracle}(M)$\\
if $x\in \calS$ then Ret 1\\
Ret 0
}
%
{
\oracle{$\MemOracle(x)$}\\
Ret $[x \in \calS]$\\

\medskip
\oracle{$\QryOracle(x)$}\\
Ret $\Qry^{\HashOracle}(M,\tau,x)$\\

\medskip
\oracle{$\HashOracle(x)$}\\
$\rho_i \getsr Func(\univ,[m])$, for $i\in[1,k]$\\
$z_{i} \gets \rho_i(x)$, for $i\in[1,k]$\\
if $T[x] \neq $ undefined\\
\nudge Bad $\gets$ true\\
\fbox{
$z_i \gets T[x]$, for $i\in[1,k]$
}
$\\T[x] \gets z_i$, for $i\in[1,k]$\\
Ret $z_i$, for $i\in[1,k]$\\
}
}
\caption{\Cref{thm3}: Game playing argument}\label{fig:GameT3}
\end{figure}

\noindent
It is evident from $\game{0}(A)$ and $\game{1}(A)$ that 
\begin{equation}
\Prob{\game{0}(A) : \mathrm{Bad}=\mathrm{true}} = \Prob{\game{1}(A) : \mathrm{Bad}=\mathrm{true}}\label{eq:1T3}
\end{equation}

\noindent
Using fundamental lemma of game playing argument and \Cref{eq:1T3}
\begin{eqnarray}
\nonumber \Prob{\game{0}(A)=1} & \leq & \Prob{\game{0}(A)=1 | \mathrm{Bad} \neq \mathrm{true}} + \Prob{\game{0}(A) : \mathrm{Bad}=\mathrm{true}}\\
&=& \Prob{\game{1}(A)=1} + \Prob{\game{1}(A) : \mathrm{Bad}=\mathrm{true}}\label{eq:2T3}
\end{eqnarray}

\noindent
Let the random oracle be modified to output $k$ uniformly random values without setting any Bad flag. Let's call it $\game{2}(A)$ which is shown in \Cref{fig:Game2T3}. It turns out that the  $\QryOracle$ is no more deterministic due to coins of the random oracle. Hence, the results of the $q$ queries by $A$ are mutually independent and random. So, neglecting the output of $A$, the game samples one of the elements queried by $A$, and returns 1/0. 

\begin{equation}
\Prob{\game{1}(A)=1} = \Prob{\game{2}(A)=1} = \frac{q}{2^\mu}\label{eq:3T3}
\end{equation}

\begin{figure}
\centering
\fpage{.6}{
\hpagess{.5}{.5}
{
\underline{$\game{2}(A)$}\\
$\calZ \gets \emptyset$\\
$\calS \getsr \distr{\univ}{n}$\\
$(M,\tau) \getsr \Rep^{\HashOracle}(\calS)$\\
$x' \gets A^{\MemOracle,\QryOracle,\HashOracle}(M)$\\
$x \getsr \calZ$\\
if $x\in \calS $ then Ret 1\\
Ret 0
}
%
{
\oracle{$\MemOracle(x)$}\\
Ret $[x \in \calS]$\\

\medskip
\oracle{$\QryOracle(x)$}\\
Ret $\Qry^{\HashOracle}(M,\tau,x)$\\

\medskip
\oracle{$\HashOracle(x)$}\\
$\calZ \gets \calZ \cup x$\\ 
$\rho_i \getsr Func(\univ,[m])$, for $i\in[1,k]$\\
$z_i \gets \rho_i(x)$, for $i\in[1,k]$\\
Ret $z_i$, for $i\in[1,k]$\\
}
}
\caption{\Cref{thm3}: Game playing argument}\label{fig:Game2T3}
\end{figure}

In $\game{1}(A)$, $T[x]\neq$ undefined, if earlier $\Rep$ asked hash-indices of $x$. So,  Bad $\gets$ true if $A$ queries an element present in $\calS$. 
\begin{equation}
\Prob{\game{1}(A):\mathrm{Bad} = \mathrm{true}} = \frac{n}{\abs{\univ}}\label{eq:4T3}
\end{equation}

From \Cref{eq:0T3,eq:1T3,eq:2T3}, \Cref{eq:3T3} and \Cref{eq:4T3}

\begin{equation}
\AdvPrivPubHashBB{B}{A} \leq  \frac{q}{2^\mu} + \frac{n}{\abs{\univ}}
\end{equation}

\end{proof}


\iffalse
\section{Summary of attacks(based on Bloom filter exploit) on Squid}
\acnote{For reference}\\
\heading{Crossby and Wallach: Squid Attack on hash tables} 
\begin{itemize}
\item A Squid server was ran on a stand-alone machine.
\item A file containing a list of URL's was prepared.
\item The URL requests were parsed and served by the Squid server. This completed the set-up phase.
\item Crossby and Wallach targeted hash collisions in Squid to increase the average URL load time. The authors quote ``This attack does not represent a ''smoking gun'' for algorithmic complexity attacks, but it does illustrate how common network services may be sensitive to these attacks". 2 experiments were carried out to illustrate this.
\item In experiment 1, around 143K random URL's were requested to the proxy server, whereas in experiment 2, same number of carefully chosen URL's that collide with the already present URL's in the hash-table were requested. 
\item Observation: The load time of random URL's was less than the chosen URL's. It must be noted that, the authors did not use a brute-force approach to find collisions. Rather,  they proposed an efficient method to find hash-collisions. The method involves finding a set of elements that hash to the same index and using a combination of those elements to find more elements. 
\end{itemize}

\heading{Gerber et. al. : Squid attack on cache digest:}
\begin{itemize}
\item Squid servers follow a hierarchical model, and in this model the peers can exchange summary of their cache using Bloom filters(cache-digest). 
\item In the set-up phase, the authors configure 2 peer Squid servers for caching and an HTTP server which responds to all queries of the Squid servers. 
\item Peer 1's cache-digest is polluted using fake URL's. Both the peers exchange their cache-digests 
\item Client(s) of Peer 2, start making URL requests. The URL's which are not cached by the latter are searched in its neighbors cache-digest and false positives are counted.
\item Observation: There is a substantial increase in FP's due to pollution as compared to FP's without it.
\item Reason: Fake URL's increase the Hamming weight of the cache digest of peer 1 and hence increase false positives.
\end{itemize}

\fi
