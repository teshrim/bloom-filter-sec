\section{Theorems and proofs}
\begin{theorem}
Fix $n,k, m > 0$ and define a hash-based filter as in section \ref{sec:hbf}, except the hash functions are defined by $h_j(x) = \left( f_{K0}(x) + j\cdot g_{K1}(x)\right) \bmod m$. Let $Func(m,m)$ be the set of all functions $\rho \colon \calK \times \bits^m \sends \bits^m$. Let $B_1, B_2$ be 2 adversaries with keys $K_1, K_2 \getsr \calK$ attacking the PRF advantage of $F \colon \calK \times \bits^m \sends \bits^m$ and $A$ be another adversary attacking the false positive game of hash-based filters, then 
\begin{equation}
\AdvFPSecHash{B}{\distr{U}{n},A}  = 0.5\Big[\AdvPRF{F}{B_1} + \AdvPRF{F}{B_2}\Big] + \varepsilon, \nudge where \nudge \varepsilon = (1-(1-e^{-kn/m})^k)
\end{equation}

\end{theorem}

\begin{proof}
Let adversary $B$ simulate the fp-priv game of $A$ as shown in Fig. \ref{fig:AdvB}, where the hash functions are defined as $h_j(x) = \left( f_{K0}(x) + j\cdot g_{K1}(x)\right) \bmod m$. Based on the variable $c(c\getsr \bits)$,  $B$ either acts as adversary $B_1$ or $B_2$. Fig. \ref{fig:PRFexp} shows the PRF experiment of $B$. When $B$ makes an oracle query on behalf of $A$ and  $b=0$, it exactly simulates $A's$ experiment as the hash functions are uniformly random. When $b=1$, it exactly simulates the double hashing scheme proposed by Kirsh and Mitzenmacher. The fp-error($\varepsilon$)  for the double hashing scheme for large $n$ is $\varepsilon = (1-e^{-kn/m})^k$ \cite{less hashing paper}.

\begin{figure}[t]
\centering
\fpage{.5}
{
\experimentv{Adversary $B^{\calO}$:}\\
$c \getsr \bits$\\
$K \getsr \calK$\\
$S \getsr \distr{U}{n}$\\
if $c=0$\\
$ \nudge h_j = \calO(\cdot) + jF_K(\cdot), \nudge \mathrm{where} \nudge j \in [1,k]$\\
else\\
$\nudge  h_j = F_K(\cdot) + j\calO(\cdot), \nudge \mathrm{where} \nudge j \in [1,k]$\\
$M \getsr \Rep^{h_1, \cdots, h_k}(S)$\\
Run $A$\\
When $A$ asks $\QryOracle(x)$: \\
$\nudge$ Ret $\QryOracle^{h_1, \cdots, h_k}(M, x)$ to $A$\\
When $A$ halts with output $x^{'}$: \\
if $x^{'} \not\in S \wedge \Qry^{h_1, \cdots, h_k}(M,x^{'})=1$ then\\
$\nudge$ Ret 1\\
Ret 0
}
\caption{Adversary $B$ simulating $A^{'}s$ fp-prv game}\label{fig:AdvB}
\end{figure}

\begin{figure}[h]
\center
\fpage{.45}{
\hpagess{.4}{.5}
{
\experimentv{$\ExpPRF{F}{B}$:}\\
$b \getsr \bits$\\
$\rho \getsr Func(m,m)$\\
$K_1 \getsr \calK$\\
$b^{'} \getsr B^{\calO(\cdot,\cdot)}$\\
Ret $[b=b']$
}
{
\oracle{$\calO(x)$}\\
$Y \gets \calF_{K_1}(M)$\\
if $b=0$\\
\nudge$Y \getsr \rho(x)$\\
Ret $Y$
}
}
\caption{PRF experiment}\label{fig:PRFexp}
\end{figure}
\acnote{Equations to be aligned}
\begin{multline*}
\Prob{\ExpPRF{F}{B} = 1} =   .5\Prob{\ExpPRF{F}{B} = 1 | b =0} + .5\Prob{\ExpPRF{F}{B} = 1 | b = 1}\\
.5\Prob{\ExpPRF{F}{B} = 1|c=0} + .5\Prob{\ExpPRF{F}{B} = 1|c=1}  = .5\Prob{\ExpFPSecHash{B}{\distr{U}{n},A}=1} + .5(1-e^{-kn/m})^k  \\
.5\Prob{\ExpPRF{F}{B} = 1|c=0} + .5\Prob{\ExpPRF{F}{B} = 1|c=1}  = .5\Prob{\ExpFPSecHash{B}{\distr{U}{n},A}=1} + .5(1-e^{-kn/m})^k \\
\Prob{\ExpPRF{F}{B} = 1|c=0} + \Prob{\ExpPRF{F}{B} = 1|c=1} = \Prob{\ExpFPSecHash{B}{\distr{U}{n},A}=1} + (1-e^{-kn/m})^k \\
 2\Prob{\ExpPRF{F}{B_1} = 1} - 1 + 2\Prob{\ExpPRF{F}{B_2} = 1} - 1 = 2\AdvFPSecHash{B}{\distr{U}{n},A} + 2(1-e^{-kn/m})^k -2\\
\AdvPRF{F}{B_1} + \AdvPRF{F}{B_2} = 2\AdvFPSecHash{B}{\distr{U}{n},A} + 2(1-e^{-kn/m})^k -2\\
 \AdvFPSecHash{B}{\distr{U}{n},A} = 0.5(\AdvPRF{F}{B_1} + \AdvPRF{F}{B_2})  + \varepsilon, \nudge \mathrm{where} \nudge \varepsilon = 1-(1-e^{-kn/m})^k
\end{multline*}

\end{proof}


