\section{Proofs}
\tsnote{Jon: ignore these for now.  This is Animesh's workspace.}
\begin{theorem}\label{thm}
Fix $n,k, m > 0$ and define a hash-based filter as in section \ref{sec:hbf}, except the hash functions are defined by $h_j(x) = \left( f_{K0}(x) + j\cdot g_{K1}(x)\right) \bmod m$. Let $Func(m,m)$ be the set of all functions $\rho \colon \calK \times \bits^m \sends \bits^m$. Let $B_1, B_2$ be 2 adversaries with keys $K_1, K_2 \getsr \calK$ attacking the PRF advantage of $F \colon \calK \times \bits^m \sends \bits^m$ and $A$ be another adversary attacking the false positive game of hash-based filters, then 

\begin{equation}
\AdvFPSecHash{B}{\distr{U}{n},A}  = 0.5\Big[\AdvPRF{F}{B_1} + \AdvPRF{F}{B_2}\Big] + 1-\epsilon, \nudge where \nudge \epsilon = (1-e^{-kn/m})^k
\end{equation}
Note: $\epsilon$ is the false positive error in the double hashing scheme given by Kirsch and Mitzenmacher \cite{xxx}
\end{theorem}

\begin{proof}
The proof of the theorem will use a game playing argument as shown in Fig. \ref{fig:Game}. In $\game{0}$, the hash functions are uniformly random functions sampled from $Func(m,m)$, whereas in $\game{1}$ and $\game{2}$, the hash functions are linear combinations of a PRF and a random function. It can be seen that, $\game{0}$ exactly simulates $\ExpFPSecHash{B}{A}$. So, $\AdvFPSecHash{B}{A}=\Prob{\game{0}(A)=1}$. Also, the probability of winning $\game{1}$ and $\game{2}$ are equal, i.e.,$\Prob{\game{1}(A)=1}=\Prob{\game{2}(A)=1}$. The hash functions in $\game{3}$ are linear combinations of 2 random functions or 2 PRF's. 

\begin{figure}[h]
\fpage{.9}{
\hpagessd{.5}{.5}
{
\underline{\game{0}(A)}\\
$h_1,h_2, \cdots, h_k \getsr Func(m,m)$\\
%$K \getsr \calK$\\
$S \getsr \distr{U}{n}$\\
$M \getsr \Rep^{h_1, \cdots, h_k}(S)$\\
$x \getsr A^{\QryOracle(\cdot)}(S)$\\
if $x \not\in S \wedge \Qry^{h_1, \cdots, h_k}(M,x)=1$ then\\
\nudge Ret 1\\
Ret 0\\\\\\
%
\oracle{$\QryOracle(x)$}\\
Ret $\Qry^{h_1, \cdots, h_k}(M,x)$
}
{
\underline{\game{1}(A)}\\
$K \getsr \calK$\\
$h_j(\cdot) \getsr F_K(\cdot)+j\calO_2(.), \nudge \mathrm{where} \nudge j \in [1,k]$\\
$S \getsr \distr{U}{n}$\\
$M \getsr \Rep^{h_1, \cdots, h_k}(S)$\\
$x \getsr A^{\QryOracle(\cdot)}(S)$\\
if $x \not\in S \wedge \Qry^{h_1, \cdots, h_k}(M,x)=1$ then\\
\nudge Ret 1\\
Ret 0\\\\
%
\oracle{$\QryOracle(x)$}\\
Ret $\Qry^{h_1, \cdots, h_k}(M,x)$
}
{
\underline{\game{2}(A)}\\
$K \getsr \calK$\\
$h_j = \calO_1(\cdot) + jF_K(\cdot), \nudge \mathrm{where} \nudge j \in [1,k]$\\
$S \getsr \distr{U}{n}$\\
$M \getsr \Rep^{h_1, \cdots, h_k}(S)$\\
$x \getsr A^{\QryOracle(\cdot)}(S)$\\
if $x \not\in S \wedge \Qry^{h_1, \cdots, h_k}(M,x)=1$ then\\
\nudge Ret 1\\
Ret 0\\\\\\
%
\oracle{$\QryOracle(x)$}\\
Ret $\Qry^{h_1, \cdots, h_k}(M,x)$
}
{
\underline{\game{3}(A)}\\
$h_j = \calO_1(\cdot) + j\calO_2(\cdot), \nudge \mathrm{where} \nudge j \in [1,k]$\\
$S \getsr \distr{U}{n}$\\
$M \getsr \Rep^{h_1, \cdots, h_k}(S)$\\
$x \getsr A^{\QryOracle(\cdot)}(S)$\\
if $x \not\in S \wedge \Qry^{h_1, \cdots, h_k}(M,x)=1$ then\\
\nudge Ret 1\\
Ret 0\\\\\\\\
%
\oracle{$\QryOracle(x)$}\\
Ret $\Qry^{h_1, \cdots, h_k}(M,x)$
}
}
\caption{Game playing argument}\label{fig:Game}
\end{figure}

\begin{eqnarray}
\nonumber \Prob{\game{3}(A)=1} &=&\frac{1}{4}\Prob{\game{3}(A)=1|b_1=0, b_2=0} + \frac{1}{4}\Prob{\game{3}(A)=1|b_1=0, b_2=1}\\
\nonumber & + & \frac{1}{4}\Prob{\game{3}(A)=1|b_1=1, b_2=0}+ \frac{1}{4}\Prob{\game{3}(A)=1|b_1=1, b_2=1}\\
\nonumber & = & \frac{1}{4}\Prob{\game{0}(A)=1} + \frac{1}{4}\Prob{\game{2}(A)=1}\\
\nonumber &\nudge + & \frac{1}{4}\Prob{\game{3}(A)=1} + \frac{1}{4}\Prob{\game{2}(A)=1|b=1}\\
\nonumber & = & \frac{1}{4}\Prob{\game{0}(A)=1} + \frac{1}{4}\Prob{\game{2}(A)=1}\\
\nonumber&\nudge + & \frac{1}{4}\Prob{\game{2}(A)=1} + \frac{1}{4}\Prob{\game{2}(A)=1|b=1}\\
\nonumber& = & \frac{1}{4}\Prob{\game{0}(A)=1} + \frac{1}{2}\Prob{\game{2}(A)=1}\\
\nonumber &\nudge + & \frac{1}{4}\Prob{\game{2}(A)=1|b=1}\\
\nonumber & = & \frac{1}{4}\Prob{\game{0}(A)=1} + \frac{1}{2}\Big[\frac{1}{2}\Prob{\game{2}(A)=1|b=0}\\
\nonumber	& \nudge + & \frac{1}{2}\Prob{\game{2}(A)=1|b=1}\Big] + \frac{1}{4}\Prob{\game{2}(A)=1|b=1}\\
\nonumber & = & \frac{1}{4}\Prob{\game{0}(A)=1} + \frac{1}{4}\Prob{\game{0}(A)=1}\\
\nonumber & \nudge + & \frac{1}{4}\Prob{\game{2}(A)=1|b=1} + \frac{1}{4}\Prob{\game{2}(A)=1|b=1}\\
& = & \frac{1}{2}\Prob{\game{0}(A)=1} + \frac{1}{2}\Prob{\game{2}(A)=1|b=1}
\end{eqnarray}
 
The PRF-adversary $B$ simulates $\game{3}(A)$ as shown in Fig. \ref{fig:AdvB}. 
\begin{figure}[h]
\centering
\fpage{0.6}{
\hpagess{0.5}{0.2}
{
$\experimentv{Adversary \nudge B^{\calO}:}$\\
$d \getsr \bits$\\
if $d=0$\\
$\nudge \calO_1 \gets \calO$\\
$\nudge K \getsr \calK$\\
$\nudge \rho \getsr Func(m,m)$\\
$\nudge c \getsr \bits$\\
$\nudge$if $c=0$\\
$\nudge \nudge \calO_2 \gets \rho$\\
$\nudge$else\\
$\nudge \nudge \calO_2 \gets F_K$\\
Run $\game{3}$\\
When $A$ asks $\QryOracle(x)$:\\
$\nudge$ Ret $\Qry^{h_1, \cdots, h_k}(M,x^{'})$\\
When $A$ halts with output $x^{'}$: \\
if $x^{'} \not\in S \wedge \Qry^{h_1, \cdots, h_k}(M,x^{'})=1$ then\\
$\nudge$ Ret 1\\
Ret 0
}
{
\vspace{.6cm}
if $d=1$\\
$\nudge \calO_2 \gets \calO$\\
$\nudge K \getsr \calK$\\
$\nudge \rho \getsr Func(m,m)$\\
$\nudge c \getsr \bits$\\
$\nudge$if $c=0$\\
$\nudge \nudge \calO_1 \gets \rho$\\
$\nudge$else\\
$\nudge \nudge \calO_1 \gets F_K$\\
}
}
\caption{Adversary $B$ simulating $\game{3}$}\label{fig:AdvB}
\end{figure}

$B$ exactly simulates \game{3}(A).
%\setlength{\mathindent}{1pt}
\begin{eqnarray}
\nonumber\Prob{\ExpPRF{F}{B} = 1} &=&  \Prob{\game{3}(A)=1}\\
\nonumber &=& \frac{1}{2}\Prob{\game{0}(A)=1} + \frac{1}{2}\Prob{\game{2}(A)=1|b=1}\\
\nonumber &=&  \frac{1}{2}\Prob{\ExpFPSecHash{B}{A}=1} + \frac{1}{2}\epsilon\\
 & \mathrm{where}& \epsilon = (1-e^{-kn/m})^k = \text{FP-error in double hashing scheme}
\end{eqnarray}

%\setlength{\mathindent}{1pt}
\begin{eqnarray}
\Prob{\ExpPRF{F}{B} = 1|d=1} &= \Prob{\ExpPRF{F}{B_1} = 1}\\
\Prob{\ExpPRF{F}{B} = 1|d=0} &= \Prob{\ExpPRF{F}{B_2} = 1}
\end{eqnarray}

%\setlength{\mathindent}{1pt}
\begin{eqnarray}
\nonumber\Prob{\ExpPRF{F}{B} = 1} &=& \frac{1}{2}\Prob{\ExpPRF{F}{B} = 1|d=1}  + \frac{1}{2}\Prob{\ExpPRF{F}{B} = 1|d=0}\\
 &=&  \frac{1}{2}\Big[\Prob{\ExpPRF{F}{B_1} = 1} + \Prob{\ExpPRF{F}{B_2} = 1}\Big]
\end{eqnarray}
%\setlength{\mathindent}{1pt}
\begin{flalign}
\nonumber\frac{1}{2}\Big[\Prob{\ExpPRF{F}{B_1} = 1} &+ \Prob{\ExpPRF{F}{B_2} = 1}\Big] = \frac{1}{2}\Prob{\ExpFPSecHash{B}{A}=1} + \frac{1}{2}\epsilon\\
\nonumber2\Prob{\ExpPRF{F}{B_1} = 1}-1 &+ 2\Prob{\ExpPRF{F}{B_2} = 1}-1  = 2\Big[\Prob{\ExpFPSecHash{B}{A}=1} + \epsilon -1\Big]\\
\nonumber \frac{1}{2}\Big[\AdvPRF{B_1}{A} &+ \AdvPRF{B_2}{A}\Big]  =  \AdvFPSecHash{B}{A} + \epsilon -1\\
\AdvFPSecHash{B}{A}  &=  \frac{1}{2}\Big[\AdvPRF{B_1}{A} + \AdvPRF{B_2}{A}\Big] + \Big[1-\epsilon\Big]
\end{flalign}

\end{proof}

%***Theorem 2
\begin{theorem}
In Theorem \ref{thm}, let $h_j(x) = f_{K}(\langle j \rangle \concat x) \bmod m$, where $f$ is a PRF. This setting compared to the double hashing scheme ensures independence of hash functions. Let $B$ be an adversary attacking the PRF advantage of $F \colon \calK \times \bits^m \sends \bits^m$ and $A$ be another adversary attacking the false positive game of the modified hash-based filter, then 

\begin{equation}
\AdvFPSecHash{B}{\distr{U}{n},A}  = 0.5\Big[\AdvPRF{F}{B_1} + \AdvPRF{F}{B_2}\Big] + 1-\epsilon, \nudge \mathrm{where} \nudge \epsilon = (1-e^{-kn/m})^k
\end{equation}
\end{theorem}

\begin{proof}

\begin{figure}[h]
\centering
\fpage{0.6}{
$\experimentv{Adversary \nudge B^{\calO}:}$\\
$h_j \getsr \calO(j\times \cdot), \nudge \mathrm{where} \nudge j\in [k]$\\
$S \getsr \distr{U}{n}$\\
$M \getsr \Rep^{h_1, \cdots, h_k}(S)$\\
$x \getsr A^{\QryOracle(\cdot)}(S)$\\
if $x \not\in S \wedge \Qry^{h_1, \cdots, h_k}(M,x)=1$ then\\
\nudge Ret 1\\
Ret 0\\
%
\oracle{$\QryOracle(x)$}\\
Ret $\Qry^{h_1, \cdots, h_k}(M,x)$
}
\caption{Adversary $B$ simulating modified hash-based filter}\label{fig:AdvB1}
\end{figure}

Adversary $B$ simulates $A^{'}s$ game as shown in Fig. \ref{fig:AdvB1}. When, $b=0$, $B$ exactly simulates $\ExpFPSecHash{B}{A}$. 

\begin{eqnarray*}
\Prob{\ExpPRF{F}{B} = 1|b=0} &=&  \Prob{\ExpFPSecHash{B}{A}=1}
\end{eqnarray*}

When $b=1$, adversary $B$ attacks fp-priv game of the modified scheme. The false positive error of this scheme is calculated in section \ref{sec:fpe_calculation} using Lemma 4.1 of Kirch et. al. paper\cite{xxx}.

\subsection{False positive error of hash-based filter with $h_j(x) = f_{K}(\langle j \rangle \concat x) \bmod m$ }\label{sec:fpe_calculation}

In the proposed hash-based filter, since, $f_K$ is a PRF, each hash function $h_j$ is independent and uniformly distributed over $[m]$, and hence, satisifies condition 1 of Lemma 4.1\cite{xxx}. Conditions $2$ and $5$ are trivially satisfied. Fix $u \in U$, $r, r_1, r_2 \in [m]$ and $j, j_1, j_2 \in [k]$.

\begin{eqnarray}
\nonumber\Prob{\exists j | h_j(u)=r} & \leq & \sum_{j}\Prob{f_K(\langle j \rangle \concat x)=r}\\
& = & \frac{k}{m}\\\nonumber & &\\
\nonumber\Prob{h_{j1}(u) = r_1, h_{j2}(u) = r_2} & = & \Prob{h_{j1}(u) = r_1}\Prob{h_{j2}(u) = r_2 | h_{j1}(u) = r_1}\\
\nonumber& = & \frac{1}{m} \times \frac{k}{m}\\
&=&	 O\Big(\frac{1}{n^2}\Big)\\\nonumber & &\\
\nonumber \Prob{\exists j | h_j(u)=r } & \geq & \sum_{j} \Prob{h_j(u)=r} - \sum_{j_1 < j_2}\Prob{h_{j1}(u) = r, h_{j2}(u) = r}\\
& = & \frac{k}{m} - k^2O\Big(\frac{1}{n^2}\Big)\label{eq:1}\
\end{eqnarray}

For $\lambda = k^2/c, \nudge \mathrm{where} \nudge c=n/m$,  and $\gamma(n)=\frac{1}{n^2}$, $\Prob{r\in H(u)} = \Prob{\exists j \in {k}: h_j(u)=r }$ which satisfies condition 3 of Lemma 4.1. 
\begin{eqnarray}
\nonumber\Prob{r_1, r_2 \in H(u)} & \leq & \sum_{j_1 < j_2}\Prob{h_{j1}(u) = r_1, h_{j2}(u) = r_2} \\
\nonumber& = & k^2O\Big(\frac{1}{n^2}\Big)\\
& = & O\Big(\frac{1}{n^2}\Big)\label{eq:2}
\end{eqnarray}
From eqn. \ref{eq:2}, condition 4 of Lemma 4.1 is satisfied. Since, all conditions of the Lemma are satisfied, Theorem 4.1 can be applied to the proposed hash-based filter. Let $\calF$ is the false positive probability of the proposed scheme.
\begin{eqnarray}
\nonumber\lim_{n \rightarrow \infty}\Prob{\calF} & = & (1-e^{-\lambda/k})^k \\
\nonumber& = & (1-e^{-\frac{k^2/c}{k}})^k\\
& = & (1-e^{-\frac{kn}{m}})^k
\end{eqnarray}
Let $\epsilon = (1-e^{-\frac{kn}{m}})^k$.
\begin{eqnarray}
\nonumber\Prob{\ExpPRF{F}{B}=} & = & .5\Prob{\ExpPRF{F}{B}=1|b=0} + .5\Prob{\ExpPRF{F}{B}=1|b=1}\\
\nonumber\Prob{\ExpPRF{F}{B}=} & = & .5\Prob{\ExpFPSecHash{B}{A}=1} + .5\Prob{\calF}\\
\nonumber 2\Prob{\ExpPRF{F}{B}=}-1 & = & \Prob{\ExpFPSecHash{B}{A}=1} + \epsilon - 1\\
\AdvFPSecHash{B}{A} & = & \AdvPRF{F}{B} + [1-\epsilon]
\end{eqnarray}
\end{proof}

