\section{Tasks, and Ideas for Tasks}

\begin{itemize}
\item Prove that a Bloom filter using hash functions defined by $h_j(x) = \left( f_{K0}(x) + j\cdot g_{K1}(x)\right) \bmod m$, for secret keys $K0,K1$, is FP-secure when either~$f$ or~$g$ is a secure PRF.  (This is a real proposal, see the intro.)  Interesting that these $h_j$ are not independent, even if $f_{K0},g_{K1}$ are replaced by independent random functions, whereas the traditional BF false positive analysis treats the $h_j$ as independent.  I wonder if the analysis for determining the number of hash functions still holds when the $h_j$ are dependent?  Note: definitely not in general, just consider $h_j$ a fixed function for all $j$; but it's an interesting question to ask for this particular kind of dependence.  (Should check the Kirsh, Mitzenmacher paper first; they probably address this.)

\item Show the same thing for the simpler $h_j(x) = f_{K}(\langle j \rangle \concat x) \bmod m$.  This, by the way, does give independent $h_j$ when you replace~$f_k$ by a random function.

\item Break a Squid server!  I'm thinking about a privacy break here.  One could start by simulating the (very simple) Bloom filter that Squid implements, and eventually stand up an actual Squid proxy.  Perhaps populate the filter with a random sampling of URLs from the Alexa top 1000.  (How large a sample?)  Given access to the filter, we can probably use the same attacks leveraged by Niedermayer et al.\ to infer information about which URLs are cached.  Of course, there's a timing based attack, just pinging the Squid proxy with URLs.   But this is probably harder than it sounds, and I don't know if/how Squid updates its filter on false-positive/cache misses.

\item Fix a Squid server!  Implement a fix that is *fast* and doesn't suffer from deployment headaches at interesting scales.  If the attack is bad enough, and the fix is efficient enough, we might even get this pushed into the Squid codebase.

\item Find measurement info on false positive rates for Squid filters/HTTP proxies in practice.  This must have been done already.

\item Considering the P2P setting, how should we formalize insider attacks?  I'm thinking something like FP and privacy with corruptions.  What about authenticity of published Bloom filters?

\item What about content-based routing (or whatever they are calling that area now)?  A Bloom filter seems like an obvious tool for that setting, and the content-seeker might want to keep its preferences private.

\item Formalize a security goal for Stochastic Fair Blue (or similar TCP stream routing primitives) and prove that a secure, mutable hash-based filter implies secure Stochastic Fair Blue.

\item Do the same for other applications of Bloom-filter-like primitives.  (For example, attenuated Bloom filters, used in the OceanStore project (at least).)

\item Would a secure Bloom filter (of some kind) give us a secure strike list (whatever secure means), like what was originally proposed in QUIC?

\end{itemize}