We initiate a provable-security treatment of abstract data structures, by formalizing them as syntactic primitives, and considering their correctness and privacy first-class security properties.  Our syntax is quite broad, capturing a wide variety of in-use and academic data structures.  Concretely, we explore the security of several data structures that support (approximate) set-membership tests.  Our notion of correctness captures an adversary's ability to cause the data structure to err via its allowed set of queries.  Our main privacy notion neatly captures what is leaked by the public portion of the data structure.  As certain widely used set-membership data structures, like the Bloom filter, do not meet our notion, we consider a secondary notion of onewayness; essentially, capturing the inability to reproduce any element of the set given the public portion of the data structure.  We also compare our notions to recent work by Naor and Yogev~\cite{naor2015bloom}, which also considers Bloom-filter-like data structures, albeit in a much more limited way.  Our syntax and notions let us consider security under a variety of usage scenarios, by clearly surfacing the split between public and private portions of the data structure.

We see this work as laying the foundation for a number of follow-on efforts, for example to study dynamic data structures (e.g. Bloom filter variants whose representation changes over time), and the security of primitives built on top of abstract data structures.
