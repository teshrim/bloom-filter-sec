\newcommand{\FK}{F_K(\langle j,x \rangle)}
\newcommand{\rhoK}{\rho(\langle j,x \rangle)}
\fixme{Clean up to match style, language and typesetting (macros,
  especially) of previous theorem.}
\begin{theorem}\label{thm2}
A hash-based filter with private hash functions $h_j(x) = \FK$ is secure in the false positive game of an adaptive adversary (A) who makes $q$ set-membership queries in time $O(t)$ , if F is PRF secure and the filter has low FP-error($\epsilon$) in the non-adaptive setting. Here, adversary $B$ attacks the PRF game in time $O(t)$, makes $q$ oracle queries, and 
\begin{equation}
\AdvFPSecHash{B,t}{A} \leq  \AdvPRF{F}{B} + q \cdot \epsilon
\end{equation}
Note: $\FK$ provides domain separation and hence mutual independence, unlike hash functions in theorem \ref{thm1}
\end{theorem}

\begin{proof}
We use a game playing argument similar to theorem \ref{thm1} to prove this theorem. In \game{0}(A), adversary attacks the FP game with private hash functions $h_j(x) = \FK $, and hence exactly simulates $\ExpFPSecHash{B,t}{A}$.
\begin{equation}
\AdvFPSecHash{B,t}{A} = \Prob{\game{0}(A)=1}\label{eq:0T2}
\end{equation}
\game{1}(A) simulates $\ExpFPSecHash{B,t}{A}$ with a random function $\rho \getsr Func(U,[m])$ used to generate uniformly random  hash-indices with domain separation, of elements $x \in U$. Let $A^{'}$ be the non-adaptive friend of $A$. \game{2}(A) simulates behavior of $A^{'}$ as closely as possible. The three games are shown in Fig. \ref{fig:GameT2}.

%\caption{Game playing argument}\label{fig:Game}
\begin{figure}
\fpage{.99}{
\hpagesssl{.34}{.34}{.31}
{
\underline{\game{0}(A)}\\
$K \getsr \calK $\\
$S \getsr \distr{U}{n}$\\
$(M,\tau) \getsr \Rep^{\HashOracle}(\calS)$\\
$\calX \getsr A^{\QryOracle}(\calS)$\\
if $|\calX| < t$ or $\calX \cap \calS \neq \emptyset$ then \\
\nudge Ret 0\\
if $\exists x \in \calX$ s.t. $\Qry^{\HashOracle}(M,\tau,x)=0$ then\\
\nudge Ret 0\\
Ret 1\\\\
%
\oracle{$\QryOracle(x)$}\\
Ret $\Qry^{\HashOracle}(M,\tau,x)$\\\\
%
\oracle{$\HashOracle(x)$}\\
$h_j(x) = \FK$, for $j\in[1,k]$\\
Ret $\left(h_1(x),\ldots,h_k(x)\right)$
}
{
\underline{\game{1}(A)}\\
$\rho \getsr Func(U,[m])$\\
$S \getsr \distr{U}{n}$\\
$(M,\tau) \getsr \Rep^{\HashOracle}(\calS)$\\
$\calX \getsr A^{\QryOracle}(\calS)$\\
if $|\calX| < t$ or $\calX \cap \calS \neq \emptyset$ then \\
\nudge Ret 0\\
if $\exists x \in \calX$ s.t. $\Qry^{\HashOracle}(M,\tau,x)=0$ then\\
\nudge Ret 0\\
Ret 1\\\\
%
\oracle{$\QryOracle(x)$}\\
Ret $\Qry^{\HashOracle}(M,\tau,x)$\\\\
%
\oracle{$\HashOracle(x)$}\\
$h_j(x) = \rhoK $, for $j\in[1,k]$\\
Ret $\left(h_1(x),\ldots,h_k(x)\right)$
}
{
\underline{$\game{2}(A)$}\\
$\ell \getsr [q]$, where $[q] = {0,1, \cdots, q-1}$\\
$count = 0$ \\%count will be used to keep track of queries made to $\QryOracle$\\
$win \gets false$\\
$\rho \getsr Func(U,[m])$\\
$S \getsr \distr{U}{n}$\\
$(M,\tau) \getsr \Rep^{\HashOracle}(\calS)$\\
$\calX \getsr A^{\QryOracle}(\calS)$\\
if $|\calX| < t$ or $\calX \cap \calS \neq \emptyset$ then \\
\nudge Ret 0\\
if $\exists x \in \calX$ s.t. $\Qry^{\HashOracle}(M,\tau,x)=0$\\
\nudge Ret 0\\
Ret 1\\\\
%
\oracle{$\QryOracle(x)$}\\
$count=count +1$\\
if $(count == \ell)$\\
$\nudge win = true$\\
$\nudge \mathrm{Ret} \nudge \Qry^{\HashOracle}(M,\tau,x)$\\
else \\
$\nudge \mathrm{Ret}\nudge 0$\\\\
%
\oracle{$\HashOracle(x)$}\\
$h_j(x) = \rhoK$, for $j\in[1,k]$\\
Ret $\left(h_1(x),\ldots,h_k(x)\right)$
}
}
\caption{Theorem \ref{thm2}:Game playing argument}\label{fig:GameT2}
\end{figure}

In Fig. \ref{fig:BGame1T2}, adversary $B$ simulates $\game{1}(A)$ and $\game{2}(A)$ for $b=1$ and $b=0$ respectively .

%\caption{Adversary $B$ and $A^{'}$ simulating $\game{1}(A)$ and $\game{2}(A)$ respectively}
\begin{figure}
\centering
\fpage{0.9}{
\hpagessl{0.5}{0.5}
{
$\experimentv{Adversary \nudge B^{\calO}:}$\\
$S \getsr \distr{U}{n}$\\
$(M,\tau) \getsr \Rep^{\HashOracle}(\calS)$\\
When $A$ asks $\QryOracle(x)$:\\
$\nudge$ Ret $\Qry^{\HashOracle}(M,\tau,x)$\\
When $A$ halts with output $\calX^{'}$: \\
if $|\calX^{'}| < t$ or $\calX^{'} \cap \calS \neq \emptyset$ then \\
\nudge Ret 0\\
if $\exists x \in \calX^{'}$ s.t. $\Qry^{\HashOracle}(M,\tau,x)=0$ then\\
\nudge Ret 0\\
Ret 1\\\\
\oracle{$\HashOracle(x)$}\\
$ \mathrm{Ret} \nudge \calO(\langle j,x \rangle)$, for $j\in[1,k]$\\
}
{
$\experimentv{Adversary \nudge A^{'\HashOracle}:}$\\
$\ell \getsr [q]$, where $[q] = {0,1, \cdots, q-1}$\\
$count = 0$\\
$\rho \getsr Func(U,[m])$\\
$S \getsr \distr{U}{n}$\\
$(M,\tau) \getsr \Rep^{\HashOracle}(\calS)$\\
When $A$ asks $\QryOracle(x)$:\\
$\nudge cnt = cnt + 1$\\
\nudge if $count == \ell$\\
$\nudge$ $\nudge r \gets \Qry^{\HashOracle}(M,\tau,x)$\\
\nudge else\\
$\nudge$ $\nudge r = 0$\\
\nudge Ret $r$\\
When $A$ halts with output $\calX^{'}$\\
\nudge if $|\calX^{'}| < t$ or $\calX^{'} \cap \calS \neq \emptyset$ then \\
\nudge\nudge Ret 0\\
\nudge if $\exists x \in \calX^{'}$ s.t. $\Qry^{\HashOracle}(M,\tau,x)=0$ then\\
\nudge \nudge Ret 0\\
\nudge Ret 1\\\\
%
\oracle{$\HashOracle(x)$}\\
$\mathrm{Ret} \nudge \rhoK$, for $j\in[1,k]$
}
}
\caption{Theorem \ref{thm2}:Adversary $B$ and $A^{'}$ simulating $\game{1}(A)$ and $\game{2}(A)$ respectively }\label{fig:BGame1T2}
\end{figure}

\begin{eqnarray}
\Prob{\ExpPRF{F}{B} = 1|b=1} &=& \Prob{\game{0}(A)=1}\label{eq:m1T2}\\
\Prob{\ExpPRF{F}{B} = 1|b=0} &=& 1-\Prob{\game{1}(A)=1}\label{eq:m2T2}
\end{eqnarray}

From eqns. \ref{eq:m1T2}, \ref{eq:m2T2},
\begin{eqnarray}
\nonumber \Prob{\ExpPRF{F}{B} = 1} &=& .5\Prob{\ExpPRF{F}{B_1}=1|b=0} + .5\Prob{\ExpPRF{F}{B_1} = 1|b=1}\\
\nonumber 2\Prob{\ExpPRF{F}{B} = 1} &=& 1-\Prob{\game{1}(A)=1} + \Prob{\game{0}(A)=1}\\
\nonumber (2\Prob{\ExpPRF{F}{B} = 1} - 1)  &=& \Prob{\game{0}(A)=1} - \Prob{\game{1}(A)=1}\\
 \AdvPRF{F}{B} &=& \Prob{\game{0}(A)=1} - \Prob{\game{1}(A)=1} \label{eq:1T2}
\end{eqnarray}

Following a similar argument from theorem \ref{thm1},
\begin{equation}
\Prob{\game{1}(A)=1} \leq q \cdot \Prob{\game{2}(A)=1} = q \cdot \Prob{\game{2}(A)=1|win=true}\label{eq:3T2}
\end{equation}
\begin{eqnarray}
\nonumber \Prob{\game{2}(A)|win=true} &=& \Prob{A^{'} \mathrm{outputs \nudge FP}}\\
& \leq & \epsilon, \nudge \mathrm{where} \nudge \epsilon = (1-e^{-kn/m})^k\label{eq:4T2}
\end{eqnarray}

From eqn. \ref{eq:0T2}, \ref{eq:1T2}, \ref{eq:3T2},  and \ref{eq:4T2}: \\
\begin{eqnarray}
\nonumber \AdvFPSecHash{B,t}{A} &-& \Prob{\game{1}(A)=1} =  \AdvPRF{F}{B}   \\
\nonumber \AdvFPSecHash{B,t}{A}  &=&  \AdvPRF{F}{B} + \Prob{\game{1}(A)=1}  \\
\AdvFPSecHash{B,t}{A} &\leq&  \AdvPRF{F}{B}  + q \cdot \epsilon
\end{eqnarray}


\end{proof}
