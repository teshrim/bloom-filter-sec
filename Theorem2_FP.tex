\newcommand{\FK}{F_K(\langle j,x \rangle)}
\newcommand{\rhoK}{\rho(\langle j,x \rangle)}
\newcommand{\OK}{\calO(\langle j,x \rangle)}

\heading{Construction of PRF with domain separation.}
Let $B_{\mathrm{ds}}=(\Hash,\Rep,\Qry)$ be a hash-based filter defined as follows. Fix non-empty sets ~$\univ$, $\calK$ and integers $k,m,n>0$.  Let $F\colon\mathcal{K} \times \univ \to [m]$ be a function family.  The $\Hash$ algorithm provides randomness with domain separation, as it samples key $K \getsr \mathcal{K}$ and defines $h_j(x) = \FK $ for $j \in [1,k]$.  The $\Rep$ and $\Qry$ algorithms are the standard (determinisitic) BF ones. Informally, the following result shows, if~$F$ is PRF secure and the filter has low FP-error in the non-adaptive setting, then $B_{\mathrm{ds}}$ is secure against adaptive FP-finding adversaries.

\begin{theorem}\label{thm2}
Fix $k,m,n,r >0$. Let $B_{\mathrm{ds}}$ be the hash based filter just described. If there exists an adaptive adversary, $A$ compatible with $\ExpFPSecHash{B_{\mathrm{ds},r}}{\cdot}$ asking $q$ queries, and has time complexity O(t), then there exists an adversary $B$ (explicitly constructed in the proof of this theorem), such that 
\[
\AdvFPSecHash{B_\mathrm{ds},r}{A} \leq  \AdvPRF{F}{B}  + {\dbinom{q}{r}} \left( (1-e^{-kn/m})^k + O(1/n) \right)^r\,.
\]
Here, $B$ asks $q$ queries, and has time complexity $O(t+qm)$.
\end{theorem}

\begin{proof}[\Cref{thm2}]
We use a game playing argument similar to \Cref{thm1} to prove this theorem. In \game{0}(A), adversary attacks the FP game with  hash functions $h_j(x) = \FK $, and hence exactly simulates $\ExpFPSecHash{B_\mathrm{ds},r}{A}$.
\begin{equation}
\AdvFPSecHash{B_\mathrm{ds},r}{A} = \Prob{\game{0}(A)=1}\label{eq:2T0}
\end{equation}

Let $B$ be a prf-adversary as shown in \Cref{fig:2TD}. Based on its challenge bit $b$, it exactly simulates $\game{0}$ or $\game{1}$. So,
\begin{align*}
\Prob{\ExpPRF{F}{B} = 1\,|\,b=1} &= \Prob{\game{0}(A)=1} \\%\label{eq:m1}\\
\Prob{\ExpPRF{F}{B} = 1\,|\,b=0} &= 1-\Prob{\game{1}(A)=1}
\end{align*}

By a standard conditioning of PRF advantage, we have

\begin{equation*}
\Prob{\game{0}(A)=1} = \AdvPRF{F}{B} + \Prob{\game{1}(A)=1}%\label{eq:13}
\end{equation*}

In $\game{1}$, $A$ attacks a different filter $\tilde{B}_\mathrm{ds}$, in which $F_K$ is replaced by a random function. $\game{2}$ and $\game{3}$ on the other hand use lazy sampling in $\Hash$ to select random values from $[m]$. This makes $\game{2}$ and $\game{3}$ identical to $\game{3}$ and $\game{4}$ of previous theorem, and hence the analysis and results of \Cref{thm1} apply to this theorem as well. Therefore,
\begin{align*}
\Prob{\game{1}(A)=1} = \Prob{\game{2}(A)=1}\\
\Prob{\game{3}(A)=1} = \frac{1}{\dbinom{q}{r}}\Prob{\game{2}(A)=1}
\end{align*}

Let $D$ be a non-adaptive adversary (shown in \Cref{fig:2TD}) that finds ~$r$ false positives against $\tilde{B}_{\mathrm{ds}}$. As discussed in the previous theorem, if $A$ wins $\game{3}$, then $D$ wins $\ExpFPSecHash{\tilde{B}_{\mathrm{ds}},r}{\cdot}$ game. So, 
\begin{equation*}
\AdvFPSecHash{\tilde{B}_{\mathrm{ds}},r}{D} = \Prob{\game{3}(A) = 1}\label{eq:2T8}
\end{equation*}
Note that like $\tilde{B}_{\mathrm{lin}}$, $\tilde{B}_{\mathrm{ds}}$ is the scheme that is analyzed in the classical Bloom filter literature. So, for a non-adaptive adversary $D$ and $r=1$, the upper bound on false-positive probability of a classical Bloom filter applies to $\tilde{B}_{\mathrm{ds}}$ as well. For $r > 1$, based on the assumption that $A$ makes no pointless queries, the queries made by $A$ as well as the elements returned by $D$ are distinct and not elements of the original set $S$. Also, since $\game{3}$ uses independent random functions, the event of getting a false positive is independent of the index of the query. So,
\begin{equation*}
\AdvFPSecHash{\tilde{B}_{\mathrm{ds}},r}{D} =   {\dbinom{q}{r}} \left( (1-e^{-kn/m})^k + O(1/n) \right)^r
\end{equation*}
\noindent
To summarize, we have
\[
\AdvFPSecHash{B_\mathrm{ds},r}{A} \leq  \AdvPRF{F}{B}  + {\dbinom{q}{r}} \left( (1-e^{-kn/m})^k + O(1/n) \right)^r
\]
which is the bound claimed in the theorem statement. \hfill \qed

%\caption{Game playing argument}\label{fig:Game}
\begin{figure}
\fpage{.9}{
\hpagessl{.45}{.5}
{
\underline{\game{0}(A)}\\
$K \getsr \calK$\\
$S \getsr \distr{\univ}{n}$\\
$M \getsr \Rep^{\HashOracle}(\calS)$\\
$\calZ \getsr A^{\QryOracle}(\calS)$\\
if $|\calZ| < r$ or $\calZ \cap \calS \neq \emptyset$ then \\
\nudge Ret 0\\
if $\exists z \in \calZ$ s.t. $\Qry^{\HashOracle}(M,z)=0$ then\\
\nudge Ret 0\\
Ret 1\\\\
%
\oracle{$\QryOracle(x)$}\\
Ret $\Qry^{\HashOracle}(M,x)$\\\\
%
\oracle{$\HashOracle(x)$}\\
for $j = 1$ to $k$\\
\nudge $h_j(x) =\FK $\\
Ret $\left(h_1(x),\ldots,h_k(x)\right)$
}
{
\underline{\game{1}(A)}\\
$\rho \getsr \Func{\univ,[m]}$\\
$S \getsr \distr{\univ}{n}$\\
$M \getsr \Rep^{\HashOracle}(\calS)$\\
$\calZ \getsr A^{\QryOracle}(\calS)$\\
if $|\calZ| < r$ or $\calZ \cap \calS \neq \emptyset$ then \\
\nudge Ret 0\\
if $\exists z \in \calZ$ s.t. $\Qry^{\HashOracle}(M,z)=0$ then\\
\nudge Ret 0\\
Ret 1\\\\
%
\oracle{$\QryOracle(x)$}\\
Ret $\Qry^{\HashOracle}(M,x)$\\\\
%
\oracle{$\HashOracle(x)$}\\
for $j=1$ to $k$\\
\nudge $h_j(x) = \rhoK $\\
Ret $\left(h_1(x),\ldots,h_k(x)\right)$
}
}
\fpage{.9}{
\hpagessl{.55}{.3}
{
\underline{{$\game{2}(A)$},\fbox{$\game{3}(A)$}}\\
$c \gets 0$\\
$\mathcal{I}\getsr [\{1,2,\ldots,q\}]^r$\\
$\bad \gets \false$\\
$S \getsr \distr{\univ}{n}$\\
$M \getsr \Rep^{\HashOracle}(\calS)$\\
$\calZ \getsr A^{\QryOracle}(\calS)$\\
if $|\calZ| < r$ or $\calZ \cap \calS \neq \emptyset$ then\\
\nudge Ret 0\\
if $\forall z \in \calZ,\,\Qry^{\HashOracle}(M,z)=1$ then \\
\nudge if $\bad=\true$ then \fbox{Ret 0}\\
\nudge Ret 1\\
Ret 0
}
{
\oracle{$\QryOracle(x)$}\\
%$\calY \gets \calY \cup \{x\}$\\
$c \gets c+1$\\
$v \gets \Qry^{\HashOracle}(M,x)$\\
if $c \in \mathcal{I}$ and $v\neq 1$ then\\
\nudge $\bad \gets \true$ \\
%\nudge \fbox{Ret 1} \\
if $c \not\in \mathcal{I}$ and $v\neq 0$ then\\
\nudge $\bad \gets \true$\\
%\nudge \fbox{Ret 0}\\
Ret~$v$\\\\
%
\oracle{$\HashOracle(x)$}\\
$a \getsr [m]$\\
for $j = 1$ to~$k$\\
\nudge $v_j = a$\\
Ret $\left(v_1,\ldots,v_k\right)$
}
}
\caption{\Cref{thm2}: Game playing argument}\label{fig:Game}
\end{figure}

%\caption{PRF adversary $B$ simulating $\game{1}(A)$ and $\game{2}(A)$, and non-adaptive adversary $D$ simulating $\game{4}(A)$} 
\begin{figure}
\centering
\fpage{0.9}{
\hpagessl{0.5}{0.5}
{
$\adversaryv{B^{\calO}}$\\
$\calS \getsr \distr{\univ}{n}$\\
$M \getsr \Rep^{\HashOracle}(\calS)$\\
When $A$ asks $\QryOracle(x)$:\\
$\nudge$ Ret $\Qry^{\HashOracle}(M,x)$\\
When $A$ halts with output $\calZ'$: \\
\nudge if $|\calZ'| < r$ or $\calZ' \cap \calS \neq \emptyset$ then \\
\nudge \nudge Ret 0\\
\nudge if $\exists z \in \calZ'$ s.t. $\Qry^{\HashOracle}(M,z)=0$ then\\
\nudge \nudge Ret 0\\
\nudge Ret 1\\\\
\oracle{$\HashOracle(x)$}\\
\nudge Ret $\OK$, for $j\in[1,k]$\\
}
{
$\adversaryv{D^{\QryOracle, \calS}}$\\
$c \gets 0$\\
$\calX \gets \emptyset$\\
$\mathcal{I}\getsr [\{1,2,\ldots,q\}]^r$\\
Run $A(S)$\\
When $A$ asks $\QryOracle(x)$:\\
\nudge $c \gets c+1$\\
\nudge $r \gets 0$\\
\nudge if $c \in \mathcal{I}$\\
\nudge \nudge $\calX \gets \calX \cup \{x\}$\\
\nudge \nudge $r \gets 1$\\
\nudge Ret $r$\\
When $A$ halts:\\
\nudge Ret $\calX$\\\\
%
%\oracle{$\QryOracle(x)$}\\
% Ret $\Qry^{\HashOracle}(M,x)$\\\\
%
%\oracle{$\HashOracle(x)$}\\
%for $j = 1$ to~$k$\\
%\nudge $v_j = \rhoK $\\
%Ret $\left(v_1,\ldots,v_k\right)$
}
}
\caption{\Cref{thm2}: \emph{Left}:PRF adversary $B$ simulating $\game{1}(A)$ and $\game{2}(A)$.\emph{Right}: Non-adaptive adversary $D$ simulating $\game{4}(A)$} \label{fig:2TD}
\end{figure}	


\end{proof}
