\newcommand{\FK}{F_K(\langle j,x \rangle)}
\newcommand{\rhoK}{\rho(\langle j,x \rangle)}
\newcommand{\OK}{\calO(\langle j,x \rangle)}

\heading{Construction of PRF with domain separation.}
Let $B_{\mathrm{ds}}=(\Hash,\Rep,\Qry)$ be a hash-based filter defined as follows. Fix non-empty sets ~$\univ$, $\calK$ and integers $k,m,n>0$.  Let $F\colon\mathcal{K} \times \univ \to [m]$ be a function family.  The $\Hash$ algorithm provides randomness with domain separation, as it samples key $K \getsr \mathcal{K}$ and defines $h_j(x) = \FK $ for $j \in [1,k]$.  The $\Rep$ and $\Qry$ algorithms are the standard (determinisitic) BF ones. Informally, the following result shows, if~$F$ is PRF secure and the filter has low FP-error in the non-adaptive setting, then $B_{\mathrm{ds}}$ is secure against adaptive FP-finding adversaries.

\begin{theorem}\label{thm2}
Fix $k,m,n,r >0$. Let $B_{\mathrm{ds}}$ be the hash based filter just described. If there exists an adaptive adversary, $A$ compatible with $\ExpFPSecHash{B_{\mathrm{ds},r}}{\cdot}$ asking $q$ queries, and has time complexity O(t), then there exists an adversary $B$ (explicitly constructed in the proof of this theorem), such that 
\[
\AdvFPSecHash{B_\mathrm{ds},t}{A} =  \AdvPRF{F}{B}  + {\dbinom{q}{r}} \left( (1-e^{-kn/m})^k  \right)^r
\]
Here, $B$ asks $q$ queries, and has time complexity $O(t+qm)$.
\end{theorem}

\begin{proof}[\Cref{thm2}]
We use a game playing argument similar to \Cref{thm1} to prove this theorem. In \game{0}(A), adversary attacks the FP game with  hash functions $h_j(x) = \FK $, and hence exactly simulates $\ExpFPSecHash{B_\mathrm{ds},r}{A}$.
\begin{equation}
\AdvFPSecHash{B_\mathrm{ds},r}{A} = \Prob{\game{0}(A)=1}\label{eq:2T0}
\end{equation}
\noindent
In $\game{1}(A)$, $A$ attacks a different filter $\tilde{B}_\mathrm{ds}$, in which $F_K$ is replaced by a random function $\rho$ sampled from $\Func{\univ,[m]}$. $\game{2}(A)$ and $\game{3}(A)$ on the other hand use lazy sampling in $\Hash$ to select random values from $[m]$. This makes $\game{2}(A)$ and $\game{3}(A)$ identical to $\game{3}(A)$ and $\game{4}(A)$ of previous theorem, and hence the analysis and results of \Cref{thm1} apply to this theorem as well. Therefore,
\begin{align}
\Prob{\game{2}(A)=1 \wedge \neg\bad} &= \Prob{\game{2}(A)=1}\Prob{\game{2}(A) : \neg\bad}\label{eq:2T1} \\
\Prob{\game{1}(A)=1} &= \Prob{\game{2}(A)=1}\label{eq:2T2}
\end{align}
From \Cref{eq:2T1,eq:2T2}
\begin{equation}
\Prob{\game{2}(A)=1 \wedge \neg\bad} = \Prob{\game{1}(A)=1}\Prob{\game{2}(A) : \neg\bad}\label{eq:2T3}
\end{equation}
\noindent
Also,
\begin{equation}
\Prob{\game{3}(A)=1} = \Prob{\game{2}(A)=1 \wedge \neg\bad}\label{eq:2T4}
\end{equation}
\noindent
Using \Cref{eq:2T3} in \Cref{eq:2T4}
\begin{equation}
\Prob{\game{3}(A)=1} = \Prob{\game{1}(A)=1}\Prob{\game{2}(A) : \neg\bad}\label{eq:2T5}
\end{equation}
\begin{equation}
\Prob{\game{2}(A) : \neg\bad} = \frac{1}{\dbinom{q}{r}}\label{eq:2T6}
\end{equation}
\noindent
From \Cref{eq:2T5,eq:2T6}
\begin{align}
\nonumber \Prob{\game{3}(A)=1} &= \Prob{\game{1}(A)=1}\frac{1}{\dbinom{q}{r}}\\
\Prob{\game{1}(A)=1} &= \dbinom{q}{r} \times \Prob{\game{3}(A)=1} \label{eq:2T7}
\end{align}

Let $D$ be a non-adaptive adversary (shown in \Cref{fig:2TD}) that finds ~$r$ false positives against $\tilde{B}_{\mathrm{ds}}$. As discussed in the previous theorem, if $A$ wins $\game{3}(\cdot)$, then $D$ wins $\ExpFPSecHash{\tilde{B}_{\mathrm{ds}},r}{\cdot}$ game. So, 
\begin{equation}
\AdvFPSecHash{\tilde{B}_{\mathrm{ds}},r}{D} = \Prob{\game{3}(A) = 1}\label{eq:2T8}
\end{equation}
Since, $D's$ $\HashOracle$ returns $k$ random values which are mutually independent of each other, for $r=1$, $\AdvFPSecHash{\tilde{B}_{\mathrm{ds}},1}{D}$ is same as the false positive probability ($(1-e^{-kn/m})^k$) of a classic Bloom filter. Also, as discussed in the previous theorem, for $r \neq 1$, the event of getting a false positive is independent of the index of the query for a non-adaptive adversary. So,
\begin{equation}
\AdvFPSecHash{\tilde{B}_{\mathrm{ds}},r}{D} =   \left( (1-e^{-kn/m})^k  \right)^r \label{eq:2T8a}
\end{equation}
\noindent
From \Cref{eq:2T7,eq:2T8,eq:2T8a}
\begin{equation}
\Prob{\game{1}(A)=1} = \dbinom{q}{r} \times \left( (1-e^{-kn/m})^k  \right)^r \label{eq:2T8b}
\end{equation}

%\caption{Game playing argument}\label{fig:Game}
\begin{figure}
\fpage{.9}{
\hpagessl{.45}{.5}
{
\underline{\game{0}(A)}\\
$K \getsr \calK$\\
$S \getsr \distr{\univ}{n}$\\
$M \getsr \Rep^{\HashOracle}(\calS)$\\
$\calZ \getsr A^{\QryOracle}(\calS)$\\
if $|\calZ| < r$ or $\calZ \cap \calS \neq \emptyset$ then \\
\nudge Ret 0\\
if $\exists z \in \calZ$ s.t. $\Qry^{\HashOracle}(M,z)=0$ then\\
\nudge Ret 0\\
Ret 1\\\\
%
\oracle{$\QryOracle(x)$}\\
Ret $\Qry^{\HashOracle}(M,x)$\\\\
%
\oracle{$\HashOracle(x)$}\\
for $j = 1$ to $k$\\
\nudge $h_j(x) =\FK $\\
Ret $\left(h_1(x),\ldots,h_k(x)\right)$
}
{
\underline{\game{1}(A)}\\
$\rho \getsr \Func{\univ,[m]}$\\
$S \getsr \distr{\univ}{n}$\\
$M \getsr \Rep^{\HashOracle}(\calS)$\\
$\calZ \getsr A^{\QryOracle}(\calS)$\\
if $|\calZ| < r$ or $\calZ \cap \calS \neq \emptyset$ then \\
\nudge Ret 0\\
if $\exists z \in \calZ$ s.t. $\Qry^{\HashOracle}(M,z)=0$ then\\
\nudge Ret 0\\
Ret 1\\\\
%
\oracle{$\QryOracle(x)$}\\
Ret $\Qry^{\HashOracle}(M,x)$\\\\
%
\oracle{$\HashOracle(x)$}\\
for $j=1$ to $k$\\
\nudge $h_j(x) = \rhoK $\\
Ret $\left(h_1(x),\ldots,h_k(x)\right)$
}
}
\fpage{.9}{
\hpagessl{.55}{.3}
{
\underline{{$\game{2}(A)$},\fbox{$\game{3}(A)$}}\\
$c \gets 0$\\
$\mathcal{I}\getsr [\{1,2,\ldots,q\}]^r$\\
$\bad \gets \false$\\
$S \getsr \distr{\univ}{n}$\\
$M \getsr \Rep^{\HashOracle}(\calS)$\\
$\calZ \getsr A^{\QryOracle}(\calS)$\\
if $|\calZ| < r$ or $\calZ \cap \calS \neq \emptyset$ then\\
\nudge Ret 0\\
if $\forall z \in \calZ,\,\Qry^{\HashOracle}(M,z)=1$ then \\
\nudge if $\bad=\true$ then \fbox{Ret 0}\\
\nudge Ret 1\\
Ret 0
}
{
\oracle{$\QryOracle(x)$}\\
%$\calY \gets \calY \cup \{x\}$\\
$c \gets c+1$\\
$v \gets \Qry^{\HashOracle}(M,x)$\\
if $c \in \mathcal{I}$ and $v\neq 1$ then\\
\nudge $\bad \gets \true$ \\
%\nudge \fbox{Ret 1} \\
if $c \not\in \mathcal{I}$ and $v\neq 0$ then\\
\nudge $\bad \gets \true$\\
%\nudge \fbox{Ret 0}\\
Ret~$v$\\\\
%
\oracle{$\HashOracle(x)$}\\
$a \getsr [m]$\\
for $j = 1$ to~$k$\\
\nudge $v_j = a$\\
Ret $\left(v_1,\ldots,v_k\right)$
}
}
\caption{\Cref{thm2}:Game playing argument}\label{fig:Game}
\end{figure}

%\caption{PRF adversary $B$ simulating $\game{1}(A)$ and $\game{2}(A)$, and non-adaptive adversary $D$ simulating $\game{4}(A)$} 
\begin{figure}
\centering
\fpage{0.9}{
\hpagessl{0.5}{0.5}
{
$\adversaryv{B^{\calO}}$\\
$\calS \getsr \distr{\univ}{n}$\\
$M \getsr \Rep^{\HashOracle}(\calS)$\\
When $A$ asks $\QryOracle(x)$:\\
$\nudge$ Ret $\Qry^{\HashOracle}(M,x)$\\
When $A$ halts with output $\calZ'$: \\
\nudge if $|\calZ'| < r$ or $\calZ' \cap \calS \neq \emptyset$ then \\
\nudge \nudge Ret 0\\
\nudge if $\exists z \in \calZ'$ s.t. $\Qry^{\HashOracle}(M,z)=0$ then\\
\nudge \nudge Ret 0\\
\nudge Ret 1\\\\
\oracle{$\HashOracle(x)$}\\
\nudge Ret $\OK$, for $j\in[1,k]$\\
}
{
$\adversaryv{D^{\QryOracle, \calS}}$\\
$c \gets 0$\\
$\mathcal{I}\getsr [\{1,2,\ldots,q\}]^r$\\
When $A$ asks $\QryOracle(x)$:\\
\nudge $c \gets c+1$\\
\nudge $r \gets 0$\\
\nudge if $c \in \mathcal{I}$\\
\nudge \nudge $r \gets \QryOracle(x)$\\
\nudge Ret $r$\\
When $A$ halts with output $\calZ$\\
\nudge Ret $\calZ$\\\\
%
\oracle{$\QryOracle(x)$}\\
 Ret $\Qry^{\HashOracle}(M,x)$\\\\
%
\oracle{$\HashOracle(x)$}\\
for $j = 1$ to~$k$\\
\nudge $v_j = \rhoK $\\
Ret $\left(v_1,\ldots,v_k\right)$
}
}
\caption{PRF adversary $B$ simulating $\game{1}(A)$ and $\game{2}(A)$, and non-adaptive adversary $D$ simulating $\game{4}(A)$} \label{fig:2TD}
\end{figure}	

Let adversary $B$ be as  shown in \Cref{fig:2TD}. It can be observed that, by construction $B$ simulates $\game{0}(A)$ and $\game{1}(A)$ respectively. For $b=1$, $B$ wins, if $A$ wins $\game{0}(\cdot)$. While for $b = 0$, if $A$ wins $\game{1}(\cdot)$, $B$ wins by returning 0. Therefore,
\begin{align}
\Prob{\ExpPRF{F}{B} = 1\,|\,b=1} &= \Prob{\game{0}(A)=1}\label{eq:2T9}\\
\Prob{\ExpPRF{F}{B} = 1\,|\,b=0} &= 1-\Prob{\game{1}(A)=1}\label{eq:2T10}
\end{align}
\noindent
From \Cref{eq:2T9,eq:2T10}
\begin{align}
\nonumber \Prob{\ExpPRF{F}{B} = 1} &= .5\Prob{\ExpPRF{F}{B}=1\,|\,b=0} + .5\Prob{\ExpPRF{F}{B} = 1\,|\,b=1}\\
\nonumber 2\Prob{\ExpPRF{F}{B} = 1} &= 1-\Prob{\game{1}(A)=1} + \Prob{\game{0}(A)=1}\\
\nonumber (2\Prob{\ExpPRF{F}{B} = 1} - 1)  &= \Prob{\game{0}(A)=1} - \Prob{\game{1}(A)=1}\\
 \AdvPRF{F}{B} &= \Prob{\game{0}(A)=1} - \Prob{\game{1}(A)=1} \label{eq:2T11}
\end{align}
\noindent
Using \Cref{eq:2T0,eq:2T8b} in \Cref{eq:2T11}
\begin{align}
 \nonumber \AdvPRF{F}{B} &= \AdvFPSecHash{B_\mathrm{ds},r}{A} - \dbinom{q}{r} \times \left( (1-e^{-kn/m})^k  \right)^r \\
\AdvFPSecHash{B_\mathrm{ds},r}{A} &= \AdvPRF{F}{B} + \dbinom{q}{r} \times \left( (1-e^{-kn/m})^k  \right)^r\label{eq:2T12}
\end{align}
\end{proof}

\iffalse
\acnote{I think we should give a proof of the non-adaptive FP-error bound $(1-e^{-kn/m})^k + O(1/n) $. This bound which we picked from KM's theorem 6.2, is a particular case (double hashing scheme).  Theorem 6.1, on the other hand holds good for any hashing scheme that satisfies Lemma 4.1. \\
 I tried to give the FP-bound as a Lemma. 
\begin{lemma}\label{lemma1}
Fix $k,m,n >0$. The FP-error of hash-based filter $B_{\mathrm{ds}}$ with hash functions $h_j(x) = \FK, \, j \in [k]$ is upperbounded by $(1-e^{-kn/m})^k + O(1/n)$.
\end{lemma}
}
\acnote{As I had told you earlier, I was not able to understand the proofs of KM paper and now I am stuck in this proof. 
\begin{proof}($Lemma$ \ref{lemma1})
In \ref{KM}, $Theorem \,6.1$ gives the upperbound of FP error of any hashing scheme satisfying the 5 conditions enlisted in Lemma 4.1, and all these conditions are satisfied by $B_{\mathrm{ds}}$ as shown below.\\
Since the hash functions of $B_{\mathrm{ds}}$ provide randomness with domain separation and the number of hash functions are controlled by parameter $j$, condition 1 and 2 are trivially satisfied. The last condition is also trivial, since size of $B_{\mathrm{ds}}$ will be some constant multiple of size of $S$, as in the classical bloom filter.\\
The satisfiability of remaining 2 conditions is non-trivial ...
\end{proof}}
%
\tsnote{Clean up so it looks like the previous theorem.}
\tsnote{I don't see why this is necessary.  Doesn't it follow from the
standard bounds on Bloom filters (go look them up and adjust the
theorem accordingly) which assume that the hash functions are random
functions?  Whatever argument you use in Theorem 1 to justify
independence, i.e. turning one non-adaptive FP into~$r$ non-adaptive
FPs, should apply here, too.  You don't need to understand the KM
results deeply for this argument, by the way.  You just need to
understand why you get independence, i.e. why it is that, after the
set~$S$ has been hashed (hence the bits in~$M$ are set), then the
probabilities that distinct $x_1,\ldots,x_r$ are each FPs are
independent.  This really is pretty immediate from the fact that
$\rho_1,\rho_2$ are random functions and the $x_i$ are distinct.}
\tsnote{I guess if you want, you could give a 1-to-$r$ lemma for the hash
  functions (with random functions $\rho1,\rho_2$ used in Theorem 1, prove it, and use that lemma in the
  proof of Theorem 1.  Then do the parallel thing here, for Theorem 2.}
\fi
