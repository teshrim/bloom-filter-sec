\newcommand{\FK}{F_K(\langle j,x \rangle)}
\newcommand{\rhoK}{\rho(\langle j,x \rangle)}

\fixme{Clean up to match style, language and typesetting (macros,
  especially) of previous theorem.}

Let $B_{\mathrm{ds}}=(\Hash,\Rep,\Qry)$ be the hash-based filter defined as earlier, except the hash functions are defined as $h_j(x) = \FK, \, j \in [1,k]$. Informally, the following result shows, if F is PRF secure and the filter has low FP-error in the non-adaptive setting, then $B_{\mathrm{ds}}$ is secure against adaptive FP-finding adversaries.

\begin{theorem}\label{thm2}
Fix $k,m,n >0$. Let $B_{\mathrm{ds}}$ be the hash based filter described earlier. If there exists an adaptive adversary $A$ compatible with $\ExpFPPubHash{B_{\mathrm{ds},t}}{\cdot}$ asking $q \times r$ queries in $t$ time, then there exists an adversary $B$ attacking the PRF game in time $O(t)$ and asking same number of queries as $A$ such that 
\begin{equation}
\AdvFPSecHash{B_\mathrm{ds},t}{A}\leq  \AdvPRF{F}{B}  + {\dbinom{q}{r}} \left( (1-e^{-kn/m})^k + O(1/n) \right)^r
\end{equation}
\end{theorem}

\acnote{I think we should give a proof of the non-adaptive FP-error bound $(1-e^{-kn/m})^k + O(1/n) $. This bound which we picked from KM's theorem 6.2, is a particular case (double hashing scheme).  Theorem 6.1, on the other hand holds good for any hashing scheme that satisfies Lemma 4.1. \\
 I tried to give the FP-bound as a Lemma. 
\begin{lemma}\label{lemma1}
Fix $k,m,n >0$. The FP-error of hash-based filter $B_{\mathrm{ds}}$ with hash functions $h_j(x) = \FK, \, j \in [k]$ is upperbounded by $(1-e^{-kn/m})^k + O(1/n)$.
\end{lemma}
}
\acnote{As I had told you earlier, I was not able to understand the proofs of KM paper and now I am stuck in this proof. 
\begin{proof}($Lemma$ \ref{lemma1})
In \ref{KM}, $Theorem \,6.1$ gives the upperbound of FP error of any hashing scheme satisfying the 5 conditions enlisted in Lemma 4.1, and all these conditions are satisfied by $B_{\mathrm{ds}}$ as shown below.\\
Since the hash functions of $B_{\mathrm{ds}}$ provide randomness with domain separation and the number of hash functions are controlled by parameter $j$, condition 1 and 2 are trivially satisfied. The last condition is also trivial, since size of $B_{\mathrm{ds}}$ will be some constant multiple of size of $S$, as in the classical bloom filter.\\
The satisfiability of remaining 2 conditions is non-trivial ...
\end{proof}}
%
\begin{proof}[\Cref{thm2}]
We use a game playing argument similar to \Cref{thm1} to prove this theorem. In \game{0}(A), adversary attacks the FP game with  hash functions $h_j(x) = \FK $, and hence exactly simulates $\ExpFPSecHash{B_\mathrm{ds},r}{A}$.
\begin{equation}
\AdvFPSecHash{B_\mathrm{ds},r}{A} = \Prob{\game{0}(A)=1}\label{eq:0T2}
\end{equation}
\noindent
\game{1}(A), \game{2}(A), and \game{3}(A) are same as in the previous theorem, except the hash-functions are distinct random functions. The non-adaptive counterparts of $B_\mathrm{ds}$ as well as the prf adversary are  identical to those of $B_\mathrm{lin}$. As shown in \game{1}(A) and \game{2}(A), the $\HashOracle$ oracle returns distinct random values in these simulations. $B$, on the other hand uses its oracle to generate random values (for $b=0$) in the following way, $h_j(x) = \calO(\langle j,x \rangle)$. 

%\caption{Game playing argument}\label{fig:Game}
\begin{figure}
\fpage{.9}{
\hpagessl{.45}{.5}
{
\underline{\game{0}(A)}\\
$K_1, K_2 \getsr \calK$\\
$S \getsr \distr{\univ}{n}$\\
$M \getsr \Rep^{\HashOracle}(\calS)$\\
$\calZ \getsr A^{\QryOracle}(\calS)$\\
if $|\calZ| < r$ or $\calZ \cap \calS \neq \emptyset$ then \\
\nudge Ret 0\\
if $\exists z \in \calZ$ s.t. $\Qry^{\HashOracle}(M,z)=0$ then\\
\nudge Ret 0\\
Ret 1\\\\
%
\oracle{$\QryOracle(x)$}\\
Ret $\Qry^{\HashOracle}(M,x)$\\\\
%
\oracle{$\HashOracle(x)$}\\
$h_j(x) = \FK$\\
Ret $\left(h_1(x),\ldots,h_k(x)\right)$
}
{
\underline{\game{1}(A)}\\
$\rho \getsr \Func{\univ,[m]}$\\
$S \getsr \distr{\univ}{n}$\\
$M \getsr \Rep^{\HashOracle}(\calS)$\\
$\calZ \getsr A^{\QryOracle}(\calS)$\\
if $|\calZ| < r$ or $\calZ \cap \calS \neq \emptyset$ then \\
\nudge Ret 0\\
if $\exists z \in \calZ$ s.t. $\Qry^{\HashOracle}(M,z)=0$ then\\
\nudge Ret 0\\
Ret 1\\\\
%
\oracle{$\QryOracle(x)$}\\
Ret $\Qry^{\HashOracle}(M,x)$\\\\
%
\oracle{$\HashOracle(x)$}\\
$h_j(x) = \rhoK $\\
Ret $\left(h_1(x),\ldots,h_k(x)\right)$
}
}
\fpage{.9}{
\hpagessl{.45}{.5}
{
\underline{{$\game{2}(A)$},\fbox{$\game{3}(A)$}}\\
$c \gets 0$\\
$\mathcal{I}\getsr [\{1,2,\ldots,q\}]^r$\\
$\bad \gets \false$\\
$S \getsr \distr{\univ}{n}$\\
$M \getsr \Rep^{\HashOracle}(\calS)$\\
$\calZ \getsr A^{\QryOracle}(\calS)$\\
if $|\calZ| < r$ or $\calZ \cap \calS \neq \emptyset$ then Ret 0\\
if $\forall z \in \calZ,\,\Qry^{\HashOracle}(M,z)=1$ then \\
\nudge if $\bad=\true$ then \fbox{Ret 0}\\
\nudge Ret 1\\
Ret 0\\\\
%
\oracle{$\QryOracle(x)$}\\
%$\calY \gets \calY \cup \{x\}$\\
$c \gets c+1$\\
$v \gets \Qry^{\HashOracle}(M,x)$\\
if $c \in \mathcal{I}$ and $v\neq 1$ then\\
\nudge $\bad \gets \true$ \\
%\nudge \fbox{Ret 1} \\
if $c \not\in \mathcal{I}$ and $v\neq 0$ then\\
\nudge $\bad \gets \true$\\
%\nudge \fbox{Ret 0}\\
Ret~$v$\\\\
%
\oracle{$\HashOracle(x)$}\\
for $i = 1$ to~$k$\\
\nudge $v_i  \getsr [m]$\\
Ret $\left(v_1,\ldots,v_k\right)$
}
{
\underline{$\game{4}(A)$}\\
$c \gets 0$\\
$\mathcal{I}\getsr [\{1,2,\ldots,q\}]^r$\\
$\calY \gets \emptyset$\\
$S \getsr \distr{\univ}{n}$\\
$M \getsr \Rep^{\HashOracle}(\calS)$\\
$\calZ \getsr A^{\QryOracle}(\calS)$\\
if $|\calY| < r$ or $\calY \cap \calS \neq \emptyset$ then\\
\nudge Ret 0\\
if $\forall y \in \calY,\,\Qry^{\HashOracle}(M,y)=1$ then \\
\nudge Ret 1\\
Ret 0\\\\
%
\oracle{$\QryOracle(x)$}\\
$c \gets c+1$\\
if $c \in \mathcal{I}$ then\\
\nudge $\calY \gets \calY \cup {x}$\\
\nudge Ret 1\\
else\\
\nudge Ret 0 \\\\
%
\oracle{$\HashOracle(x)$}\\
for $i = 1$ to~$k$\\
\nudge $v_i  \getsr [m]$\\
Ret $\left(v_1,\ldots,v_k\right)$
}
}
\caption{\Cref{thm2}: Game playing argument}\label{fig:GameT2}
\end{figure}
\noindent
Using analysis of previous theorem and \Cref{lemma1}\acnote{change if lemma1 not required}, we have
\begin{align}\label{eq:1T2}
\AdvPRF{F}{B} &= \Prob{\game{0}(A)=1} - \Prob{\game{1}(A)=1} \\
\Prob{\game{1}(A)=1} &= \Prob{\game{2}(A)=1}\\
\Prob{\game{3}(A)=1} &= \frac{1}{\dbinom{q}{r}}\Prob{\game{2}(A)=1}\\
\Prob{\game{4}(A)=1} &= \Prob{\game{3}(A)=1}\\
\Prob{\game{4}(A)=1} & \leq \Big[(1-e^{-kn/m})^k +O(1/n)\Big]^r
\end{align}

\noindent
From \Cref{eq:0T2,eq:1T2}
\begin{align}
\nonumber \AdvPRF{F}{B} &= \AdvFPSecHash{B_\mathrm{ds},r}{A} - \frac{1}{\dbinom{q}{r}}\Prob{\game{2}(A)=1}\\
\nonumber \AdvFPSecHash{B_\mathrm{ds},r}{A} &= \AdvPRF{F}{B} + \frac{1}{\dbinom{q}{r}}\Prob{\game{2}(A)=1}\\
\nonumber \AdvFPSecHash{B_\mathrm{ds},r}{A} &= \AdvPRF{F}{B} +  \dbinom{q}{r}\Prob{\game{4}(A)=1}\\
\AdvFPSecHash{B_\mathrm{ds},r}{A} &= \AdvPRF{F}{B} +  \dbinom{q}{r}\Big[(1-e^{-kn/m})^k +O(1/n)\Big]^r
\end{align}

\end{proof}
