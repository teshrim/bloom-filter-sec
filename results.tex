\section{Achieving Security Notions}
\todo{Below are two lists of things that we should show.}

\heading{Correctness. }  Here we give security proofs and attacks with respect to the correctness notions.
\begin{itemize}
\item Show Bloom Filter is correct in the ROM 
\item correctness security/attacks on bigram/PRF-based double-hashing construction ($h_j(x) = F_{K1}(x) + j\cdot F_{K2}(x) \bmod m$ where the~$x$ are bigrams of names, etc.) 
\item correctness-security/attacks on bigram/domain-separated PRF construction ($h_j(x)=F_K(\langle j,x \rangle) \bmod m$, or just assume range of~$F$ is $[m]$.)
\item correctness-security/attacks Dong, Chen, Wen ``garbled Bloom filter'' construction

\end{itemize}

\heading{Privacy. } Here we give security proofs and attacks with respect to the privacy notions we have defined.
\tsnote{Under our current notions, you can never hope to prove privacy bounds that beat the element wise min-entropy of $\distr{}{}$.}
\begin{itemize}
\item Prove that the basic BF, in the ROM, is private.  \tsnote{Relative the strongest notion for which this is true.  If it isn't true for all, then give the attack(s).}

\item As a simple corrolary, show that the basic BF with $h_j(x)=F_K(\langle j,x \rangle)$ is private.  \tsnote{I believe this also shows that the bigram/PRF approach examined by Neidermayer et al.\ is secure when the hash functions are the $h_j(x)$ just described.  This was, in fact, one of the countermeasures they proposed. }

\item Prove privacy bounds for the constructions from NY

\item Cast Neidermayer et al.\ attacks on the bigram/PRF-based double-hashing construction (see their paper) into our formalism.  \tsnote{Note that their attack would allow one to recover \emph{every} name, not just one.} 

\item Prove a privacy bound for the bigram/PRF-based double-hashing construction that Neidermayer et al.\ attacks. I expect that the provable security bound is considerably worse than the min-entropy because of the way the representation is made.  (Essentially, the Hamming weight of the representation is a good estimate of the length of the longest surname in filter.)

\item Attack priv-security of Dong, Chen, Wen ``garbled Bloom Filter'' construction when the hash functions are public.\tsnote{I give the attack in the related work section.  Copy it here and flesh it out.}  Prove priv-security privacy of construction when hash functions are secret.  \tsnote{Makes the point that the application setting really matters.}

\item Prove privacy bounds of other suggestions, such as the record-level BF from Durham et al.?  This would be a nice pairing with the Neidermayer et al.\ results

\item ...
\end{itemize}
