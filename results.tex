\section{Achieving Security Notions}


%%%%%%%%%%%%%%%%% CLASSICAL BLOOM FILTER IN ROM %%%%%%%%%%%%%%%%%%%%

\begin{figure}[tp]
\centering
\hfpagess{.4}{.4}
{
\algorithmv{$\Rep^H(S)$}\\
$M \gets 0^m$\\
for $x \in S$\\
\nudge for $j \in \{1,2,\ldots,k\}$\\
\nudge\nudge $h_j \gets H(j,x)$\\
\nudge\nudge $M[h_j] \gets 1$\\
$\pubaux \gets M$, $\privaux \gets \emptystring$\\
Return $(\pubaux,\privaux)$\\

\medskip
\algorithmv{$\Qry^H(\pubaux,\privaux,q_x)$}\\
$M\gets\pubaux$\\
for $j \in \{1,2,\ldots,k\}$\\
\nudge $h_j \gets H(j,x)$\\
\nudge if $M[h_j] \neq 1$ then Return 0\\
Return 1
}
{
\algorithmv{$\Rep(S)$}\\
$M \gets 0^m$; $K \getsr \calK$\\
for $x \in S$\\
\nudge for $j \in \{1,2,\ldots,k\}$\\
\nudge\nudge $h_j \gets F_{K}(\langle j,x \rangle) $\\
\nudge\nudge $M[h_j] \gets 1$\\
$\pubaux \gets M$, $\privaux \gets \langle K \rangle$\\
Return $(\pubaux,\privaux)$\\

\medskip
\algorithmv{$\Qry(\pubaux,\privaux,q_x)$}\\
$M\gets\pubaux$, $K \gets \privaux$\\
for $j \in \{1,2,\ldots,k\}$\\
\nudge $h_j \gets F_{K}(\langle j,x \rangle) $\\
\nudge if $M[h_j] \neq 1$ then Return 0\\
Return 1
}
\caption{{\bf Left:} $\Rep$ and $\Qry$ algorithms for classic Bloom
  filter $\setprim_{\mathrm{Bloom}}$ in the ROM for a hash function $H\colon \mathbb{N} \times
  \elts \to [m]$. {\bf Right:} $\Rep$ and $\Qry$
  algorithms for the domain-separated PRF data structure
  $\setprim_{\mathrm{ds}}$, using function family $F \colon
  \calK \times \elts \to [m]$.  Both constructions have query set is $\calQ=\{q_x\colon
  \univ \to \bits\}$ where $q_x(S)=1 \Leftrightarrow x \in S$.  }
\label{fig:bf-and-lin}
\end{figure}

\subsection{Classical Bloom filter in random oracle model (ROM). }

\heading{Correctness. }
Fix $n,k,m \geq 0$, and fix a set $\elts$.  Then the classical Bloom filter set-membership data structure $\setprim_{\mathrm{Bloom}}= (\Rep, \Qry)$ over~$\elts$ is defined as in Figure~\ref{fig:bf-and-lin} (left side).  We define~$\setprim_{\mathrm{Bloom}}$ in the random oracle model, as the traditional analysis of Bloom filter correctness (as a function of $n,k,m$) is done in this model.\footnote{The standard model version would have $\Rep$ choose $H_1,H_2,\ldots,H_k$ from a hash function family, compute $h_j$ using these, and placing their description in $\pubaux$.}  The following theorem proves that the Bloom filter meets our correctness notion in the ROM.  Again, we stress that this correctness is with respect to adaptive adversaries, aiming to find $r \geq 1$ errors (in this case, false-positives).
The proof appears in Appendix~\ref{sec:thm3_fp}.

\begin{theorem}\label{thm3}\label{thm:bf-correctness}
Fix $k,m,n,r>0$, a set~$\elts$, and let $\setprim_{\mathrm{Bloom}}= (\Rep, \Qry)$ be classical Bloom filter (set-membership) data structure over~$\elts$. For any adversary~$A$ that makes a total of~$q_T$ queries to the $\Test$ oracle, and $q_R$ queries to the RO, and has time-complexity~$O(t)$,
\[
\AdvCorrect{\setprim_{\mathrm{Bloom}},r}{A} \leq  {\dbinom{q_T + q_H}{r}} \left( (1-e^{-kn/m})^k + O(1/n) \right)^r\,.
\]
\end{theorem}
The proof appears in Appendix~\ref{app:xxx}.  We note that Kirsch and Mitzenmacher~\cite{xxx} show
that the ($r=1$, zero-query) FP-probability for~$\setprim_\mathrm{Bloom}$ is
always within $O(1/n)$ of $(1-e^{-kn/m})^k$, although the asymptotic
rate of convergence is faster than~$1/n$.  The multiplicative (``hybrid'') term arises as a result
of moving from the adaptive to the zero-query setting.  
Finally, $\dbinom{q_T+q_H}{r} \leq (q_T+q_H)^r$ (with reasonable tightness when~$r$ is small) in 
which case the final term in the bound behaves as
$((q_T+q_H) (1-e^{-kn/m})^k + O(q/n) )^r$; loosely, this is the bound one expects for
trying to find a single FP in each of~$r$ independent``rounds'', each round
consisting of~$(q_T+q_H)$ attempts.
%\acnote{To be verified before moving the theorem to section 4}
\heading{Classical Bloom filter in random oracle model (ROM). }
As discussed earlier, the classical Bloom filter (BF) is an instantiation of a set multiplicity data structure and admits errors in form of false  positives (FP) only. In the ROM, $\Rep$ queries the random oracle to get the hash indices of elements of set $S$. It then computes the $m$-bit array, $M$ and outputs $\privaux=\emptystring$, $\pubaux = \emptystring$ and $M$ as the representation of
the Bloom filter. The $\Qry$ oracle uses the random oracle in a similar way. For any query $q$, it gets the hash indices $h_1, \ldots, h_k$ from the random oracle and returns 1 if $M[h_i(x)]=1$ for all $i\in [k]$, and 0 otherwise.
\begin{theorem}\label{thm3}
Fix $k,m,n,r>0$. Let $A$ be an adaptive adversary (for attacking BF) asking $q$ queries to the random oracle. Let $t$ be the running time of $A$ given these queries. Then, 
\[
\AdvCorrect{{BF},\distr{\calS}{},r}{A} \leq  {\dbinom{q}{r}} \left( (1-e^{-kn/m})^k + O(1/n) \right)^r\,.
\]
\end{theorem}

\begin{proof}[\Cref{thm3}]
The proof of this theorem will follow along the lines of \Cref{thm2}.
\end{proof}


\heading{Privacy. }
Turning to privacy, we observe immediately that the classical BF, in the
ROM, does not meet our SS-privacy notion when the leakage function
fails to reveal the entire set~$S$.  This is because the set~$S$ is
adversarially chosen, and the adversary has access to the random
oracle, i.e. the hash functions used to make~$\pubaux$.  Thus the
adversary can simply check that the $\pubaux$ it receives is correct.
Thus, unless $m,k,n$ are such that the false-positive rate is large,
the simulator needs to call the random oracle on the points in~$S$ in order to ``fool'' the adversary.
%\tsnote{I'm still not completely comfortable with this.  It isn't explicit in our SS notion, but I assume that~$A$ is stateful and so knows~$S$ in its second stage?  If so, I wonder if the classic BF in the ROM would be secure relative to a weaker notion (but still stronger than OW) in which the second stage does not know~$S$?  This seems likely to me.}

We note that this will be true for \emph{any} data structure that returns an empty $\privaux$, because this implies that $\pubaux$ alone suffices to evaluate $\Qry(\pubaux,\privaux,\cdot)$.  

\begin{theorem}\label{thm3}\label{thm:bf-ow} 
Fix $k,m,n,r>0$, a set~$\elts$, and let $\setprim_{\mathrm{Bloom}}= (\Rep, \Qry)$ be the classical Bloom filter data structure over~$\elts$. Let $\distr{}{}$ be a distribution over subsets of universe~$\univ=2^\elts$, where $\mu = H_\infty(\distr{}{})$.  For any adversary~$A$ that makes $q$~queries to its oracle
\[
\AdvPrivOW{\setprim_{\mathrm{Bloom}},\distr{}{},r}{A} \leq  \frac{(kqn+n)}{2^{\mu}}
\]
\end{theorem}
\todo{Typeset simple proof.  Result is on chalkboard.}

\subsection{Privacy of Secretly-Keyed ``Canonical'' Data Structures}
Here we prove a result that can be used to show that a large class of canonical
data structures are SS-secure, with respect to leakage of the size of the set.
\jnote{It would not be too hard to generalize this to arbitrary data structures over
arbitrary objects, so long as the number of distinct queries to the PRF made by $\Rep$
depends only on ${\sf leak}$.}
We first define what we mean by ``canonical.''

%\def\bits{\{0,1\}}

\begin{definition}
A data structure $(\Rep, \Qry)$ is {\sf canonical} if
there exist algorithms $\Rep_1, \Rep_2$, and a function
$F: \calK \times \bits^{\ell_{in}} \rightarrow \bits^{\ell_{out}}$
such that the following hold:
\begin{itemize}
\item $\Rep_1$ takes as input a set $S=\{x_1,\ldots,x_n\}$ and outputs distinct values $s_1, \ldots, s_t \in \bits^{\ell_{in}}$,
where $t$ depends only on~$n$.
\item $\Rep_2$ takes as input $n$ and $y_1, \ldots, y_t \in \bits^{\ell_{out}}$, and outputs
$\pubaux$.
\item For any $S$, the distribution on $\pubaux$ output by $\Rep(S)$ is identical to the distribution
on $\pubaux$ computing as follows: 
(1) $s_1, \ldots, s_t \getsr \Rep_1(S)$;
(2) $K \getsr \calK$;
(3) $\pubaux \getsr \Rep_2(n, F_{K}(s_1), \ldots,F_{K}(s_t))$. 
%\item $K_1,K_2,\ldots,K_p \getsr \calK$ for some fixed number of keys~$p$;
%\item $\pubaux \getsr \Rep_2(n, \{F_{K_j}(s_1),
%\ldots,F_{K_j}(s_t)\}_{j\in[p]})$. 
\end{itemize}
We say that $(\Rep_1, \Rep_2, F)$ is the {\sf canonical representation}
of~$\Rep$. \hfill\dqed \tsnote{Seems odd to call it this without explaining how
  this canonical representation handles $\privaux$, too.  }
\end{definition}
%\jnote{Not sure how general to make the above. Need to verify that it is general enough
%to prove security of all the constructions we care about.}
\tsnote{Point out an example or two of the schemes this
  covers.  Domain-separated PRF BF, for example.  I think it covers
  classical BF, too, if $F_K(s)=s$ for all $K \in \calK$, $\Rep_1$
  simply returns the elements of~$S$, and $\Rep_2$ actually picks hash
  functions and does the hashing. (Although the following theorem does
  not work in this case, of course.) It does not cover Neidermayer because there's no guarantee of distinct $s_1,\ldots,s_t$. }
We next show that any canonical data structure in which~$F$ is a pseudorandom function is
semantically secure with respect to leakage consisting of the size of the set.

\begin{theorem}\todo{Rewrite to match old DS-PRF proof}
Let $\setprim=(\Rep, \Qry)$ be a canonical data structure for which $\Rep$ has canonical representation
$(\Rep_1, \Rep_2, F)$. If $F$ is a $(t, \epsilon)$-pseudorandom
function \tsnote{overloading~$t$}, then $\setprim$ is semantically secure
with respect to leakage function ${\sf leak}(S) = |S|$.
\end{theorem}
\begin{proof}
We define a simulator $\Sim$ as follows. On input $n$, $\Sim$ fixes a
set $S^*=\{e_1, \ldots, e_n\}$. In then computes:
(1) $s_1, \ldots, s_t \getsr \Rep_1(S^*)$;
(2) $K \getsr \calK$;
(3) $\pubaux \getsr \Rep_2(|S|, F_K(s_1), \ldots, F_K(s_t))$, and
finally returns $\pubaux$.

\def\hyb{\mbox{{\sc Hyb}}}

For a given set $S$, define distribution $\mbox{{\sc Real}}_S$ as
\[ \{ (\pubaux, \privaux) \getsr \Rep(S) : \pubaux\},\]
and define distribution $\mbox{{\sf Ideal}}_S$ as
\[ \{ \pubaux \getsr \Sim(|S|) : \pubaux \}.\]
Our goal is to show that $\mbox{{\sc Real}}_S$ and $\mbox{{\sf Ideal}}_S$ are computationally
indistinguishable for any set~$S$. \jnote{I'm not quite sure how to elegantly do a proof
like this in the concrete setting.}\tsnote{I think you just follow the
sketch that I typeset for the Domain-Separated PRF scheme. (I'll embed
notes below, too.) The bound will be exactly what is there, shoudl be able to just cut-and-paste,
basically.  All that's needed, I think, is to do resource accounting
for $\Sim$ and the PRF-adversary~$B$.}
To do so, first note that we can express $\mbox{{\sc Real}}_S$ as the outcome of
the following experiment: \tsnote{I'd call this ``Game $G0$''...}
\begin{enumerate}
\item $s_1, \ldots, s_t \getsr \Rep_1(S)$.
\item $K \getsr \bits^n$.
\item $\pubaux \getsr \Rep_2(|S|, F_k(s_1), \ldots, F_k(s_t))$.
\item Return $\pubaux$.
\end{enumerate}

Now, consider a hybrid distribution $\hyb_S$ computed as follows:
\tsnote{...and this ``Game $G1$''...}
\begin{enumerate}
\item $s_1, \ldots, s_t \getsr \Rep_1(S)$.
\item $y_1, \ldots, y_t \leftarrow \bits^{\ell_{out}}$
\item $\pubaux \getsr \Rep_2(|S|, y_1, \ldots, y_t)$.
\item Return $\pubaux$.
\end{enumerate}
It follows immediately from the fact that $F$ is a pseudorandom function (and efficiency
of $\Rep_2$) that $\hyb_S$ is computationally indistinguishable from $\mbox{{\sc Real}}_S$.
\tsnote{...and then it's clear that the difference between the
  probability of $G0 \outputs 1$ and $G1 \outputs 1$ is exactly the
  PRF advantage of~$B$, who simply uses it's oracle to simulate calls
  to $F_K$ prior to running $\Rep_2$...}
Next, consider the following distribution $\hyb'_S$:
\begin{enumerate}
\item $S^* := \{e_1, \ldots, e_n\}$, where $n = |S|$.
\item $s_1, \ldots, s_{t^*} \getsr \Rep_1(S^*)$.
\item $y_1, \ldots, y_{t^*} \leftarrow \bits^{\ell_{out}}$
\item $\pubaux \getsr \Rep_2(|S|, y_1, \ldots, y_{t^*})$.
\item Return $\pubaux$.
\end{enumerate}
We claim that $\hyb'_S$ is identically distributed to $\hyb_S$. The follows immediately from the
observation that the distribution on $t$ (in $\hyb_S$) is identical to the distribution on~$t^*$
(in $\hyb'_S$) by definition
of what it means to be canonical.
\tsnote{...and it is clear that the distriubtion of everything is the
  same as in Game $G1$, so we are done. }

To complete the proof, we simply note that $\hyb'_S$ and $\mbox{{\sf Ideal}}_S$
are computationally indistinguishable, again relying on pseudorandomness of~$F$.
\end{proof}
\fixme{ ----The following is left only for cut-and-paste value---- }
\begin{theorem}\label{thm:ds-ss}
Fix $k,m,n>0$, and let $\setprim_{\mathrm{ds}}= (\Rep, \Qry)$ be the set-multiplicity data structure described in Figure~\ref{fig:lin-and-ds}.  For any set~$S \in 2^\elts$, let $\leak(S)=|S|$.  There exists a simulator~$\Sim$ such that, for any adversary~$A$
\[
\AdvPrivSS{\setprim_{\mathrm{ds}},\leak}{A,\Sim} \leq  \AdvPRF{F}{B}
\]
where~$B$ is constructed from~$A$.  (Both~$B$ and~$\Sim$ are explicitly constructed in the proof of this theorem.)
\end{theorem}
\begin{proof}[Proof (sketch). ] \todo{Move this sketch to be a proof
    of Jon's meta-result. Downgrade theorem to be a corollary of that
    result, the proof simply being a description of how $\Rep$
    decomposes into $\Rep_1$ and $\Rep_2$.}
Adversary~$B^{f}$ simulates the steps of $\ExpPrivSSreal{\setprim_{\mathrm{ds}}}{A}$, using its oracle to simuate~$\Rep(S)$.  When~$f=F_K$ then the simulation is perfect.  When~$f=\rho$ then~$B$ simulates $\ExpPrivSSreal{\overline{\setprim}_{\mathrm{ds}}}{A}$ where $\overline{\Rep}$ effectively uses~$k$ independent random functions as the hash functions.  But the $\pub$ produced by $\overline{\Rep}$ has a distribution that depends only~$|S|=n$, not the specific elements of~$S$, and the other public parameters~$m,k$.  Thus, a simulator~$\Sim$ given only $\leak(S)=|S|$ can produce a correct~$\pub$ as follows: starting with an all-zeros array~$M$ of length~$m$, uniformly sample (with replacement)~$nk$ integers in~$[m]$ and set the resulting positions of~$M$ to 1.
\end{proof}
\fixme{ ----The above is left only for cut-and-paste value---- }

%Here is a proof sketch:
%The simulator, given n=|S|, runs the honest scheme on inputs x_1, ...,
%x_n and outputs the public portion of the result.
%
%To see that this is indistinguishable from the real public portion on
%the actual set S, consider the following hybrids:
%H0: The real pub produced from the real set S={s_1, ..., s_n}
%H1: Run the representation algorithm on S, but using random functions
%in place of PRFs, and output pub.
%H2: Run the representation algorithm on {x_1, ..., x_n}, still using
%random functions in place of PRFs, and output pub. The claim is that
%this is identically distributed to H1. The main observation is that
%since we are using random functions, the inputs to the random
%functions don't matter, subject to the technical condition that the
%number of distinct inputs to the random functions does not change.
%(Here is where we rely on the fact that |S|=n is known, as well as an
%appropriate definition of "natural.")
%H3: Run the representation algorithm on {x_1, ..., x_n}, using the
%real Rep (with PRFs).

%\jnote{Organize by scheme, not by security notions.}


%%%%%%%%%%%%%%%%% DOMAIN-SEPARATED PRF %%%%%%%%%%%%%%%%%%%%

\subsection{Domain-separated PRF-Hash}
%\heading{Construction of PRF with domain separation.}
\heading{Correctness. }
Let $\setprim_{\mathrm{ds}}= (\Rep, \Qry)$ be the set-membership data
structure defined on the right-side of Figure~\ref{fig:lin-and-ds}.
As with the preceding theorem, the following says (informally) that
if~$F$ is a secure PRF, then this domain-separated PRF construction is
correct against adaptive adversaries.  
% \tsnote{Not really appropriate to say this next thing; I'll add a new section that addresses bigram DS-PRF hash.}
% We note that Neidermayer et
% al. in~\cite{xxx} suggest replacing the linear-PRF hash functions in
% the previous construction with independently keyed PRFs as a way to
% thwart their attacks.  The construction we consider now achieves the
% same end, albeit via explict domain separation rather than separate
% keys.

\begin{theorem}\label{thm2}\label{thm:ds-correctness}
Fix $k,m,n,r>0$, a set~$\elts$, and let $\setprim_{\mathrm{ds}}= (\Rep, \Qry)$ be the set-membership data structure (over~$\elts$) in Figure~\ref{fig:lin-and-ds}.  For any adversary~$A$ that makes a total of~$q$ queries to its oracle and has time-complexity~$O(t)$, there exists a prf-adversary ~$B$ (explicitly constructed in the proof of this theorem) such that
\[
\AdvCorrect{\setprim_{\mathrm{ds}},r}{A} \leq  \AdvPRF{F}{B}  + {\dbinom{q}{r}} \left( (1-e^{-kn/m})^k + O(1/n) \right)^r\,.
\]
Here, $B$ asks $q$ queries, and has time complexity $O(t+qm)$.
\end{theorem}

\heading{Privacy. }
The privacy of $\setprim_{\mathrm{ds}}$ follows directly from our general result, Theorem~\ref{thm:xxx}.
\begin{theorem}\label{thm:ds-ss}
Fix $k,m,n>0$, a set~$\elts$, and let $\setprim_{\mathrm{ds}}= (\Rep, \Qry)$ be the set-membership data structure described in Figure~\ref{fig:lin-and-ds}.  For any set~$S \in 2^\elts$, let $\leak(S)=|S|$.  There exists a simulator~$\Sim$ and adversary~$B$ such that, for any adversary~$A$
\[
\AdvPrivSS{\setprim_{\mathrm{ds}},\leak}{A,\Sim} \leq  \AdvPRF{F}{B}
\]
%(Both~$B$ and~$\Sim$ are explicitly constructed in the proof of this theorem.)
\end{theorem}
\begin{proof}
It suffices to give the decomposition of $\Rep$ in to $\Rep_1$ and $\Rep_2$ as required by Theorem~\ref{thm:xxx}.
On input a set of distinct elements $S=\{x_1,x_2,\ldots,x_n\}$, algorithm~$\Rep_1$ returns $s_1=\langle 1,x_1 \rangle, s_2=\langle 2,x_1 \rangle, \ldots, s_k\langle k,x_1\rangle, s_{k+1}=\langle 1,x_2\rangle, \ldots, s_{2k}=\langle k,x_2\rangle, \ldots ,s_{kn+1}=\langle 1,x_n\rangle, \ldots, s_{kn}=\langle k,x_n \rangle$.  On input $n,F_K(s_1),\ldots,F_K(s_{kn})$, algorithm $\Rep_2$ simply initializes an all-zero array~$M$ of~$m$-bits, and sets the positions $M[F_K(s_i)]$ to 1.  It returns~$M$ as $\pubaux$.
\end{proof}


%%%%%%%%%%%%%%%%% LINEAR-PRF %%%%%%%%%%%%%%%%%%%%
\begin{figure}[tp]
\centering
\hfpagess{.5}{.5}
{
\algorithmv{$\Rep(S)$}\\
$M \gets 0^m$; $K \getsr \calK$\\
for $x \in S$\\
\jnote{Fix}\tsnote{Fix what?}\\
\nudge for $j \in \{1,2,\ldots,k\}$\\
\nudge\nudge $h_j \gets F_{K}(\langle 1,x \rangle) + j\cdot F_{K}(\langle 2,x \rangle) \bmod m$\\
\nudge\nudge $M[h_j] \gets 1$\\
$\pubaux \gets M$, $\privaux \gets \langle K \rangle$\\
Return $(\pubaux,\privaux)$\\

\medskip
\algorithmv{$\Qry(\pubaux,\privaux,q_x)$}\\
$M\gets\pubaux$, $K \gets \privaux$\\
\jnote{Fix}\tsnote{Fix what?} \\
for $j \in \{1,2,\ldots,k\}$\\
\nudge $h_j \gets F_{K}(\langle 1,x \rangle) + j\cdot F_{K}(\langle 2,x\rangle) \bmod m$\\
\nudge if $M[h_j] \neq 1$ then Return 0\\
Return 1
}
{
\algorithmv{$\Rep(S)$}\\
$\pubaux \gets \emptystring$\\
$K_1,K_2, \getsr \calK$\\
for each $x \in S$\\
\nudge $M_x \gets 0^m$\\
\nudge $B_x \gets \mathsf{bigram}(x)$\\
\nudge for each $b \in B_x$\\
\nudge\nudge for $j \in [k]$\\
\nudge\nudge\nudge $i \gets F_{K}(\langle 1, b\rangle)+j\cdot
F_{K}(\langle 2, b  \rangle) \bmod m$\\
\nudge\nudge\nudge $M_x[i] \gets 1$\\
\nudge $\pubaux \gets \langle\pubaux, M_x \rangle$\\
$\privaux \gets \langle K_1,K_2 \rangle$\\
Return $(\pubaux,\privaux)$\\

\medskip
\algorithmv{$\Qry(\pubaux,\privaux,q_x)$}\\
$B \gets \mathsf{bigram}(x)$\\
$\overline{M} \gets 0^m$\\
for each $b \in B$\\
\nudge for $j \in [k]$\\
\nudge\nudge $i \gets F_{K}(\langle 1,b \rangle)+j\cdot F_{K}(\langle 2,b \rangle) \bmod m$\\
\nudge\nudge $\overline{M}[i] \gets 1$\\
if $\overline{M} \in \pubaux$ then Return 1 \\
Return 0
}
\caption{{\bf Left:} $\Rep$ and $\Qry$ algorithms for the linear-PRF
  data structure $\setprim_{\mathrm{lin}}$ {\bf Right:}
The scheme from Neidermayer et al.\cite{xxx}, $\setprim_{\bilin}$. The $\mathsf{bigram}$
algorithm takes a string~$x=x_1,x_2,\ldots,x_\ell$ as input, and
returns the set of bigrams for the left- and right-padded string
$\sqcup x \sqcup$, i.e. $\{\sqcup x_1, x_1x_2,\ldots,x_{\ell-1}x_\ell,x_\ell\sqcup\}$.
}
\label{fig:lin-and-bi-lin}
\label{fig:neidermayer}
\end{figure}

\subsection{Linear PRF-hash Bloom Filter }
%\heading{Linear-PRF construction. }
The domain-separated PRF-hash requires~$k$ PRF evaluations per set-membership query, and this may be unattractive in certain use cases.  Kirsch and Mitzenmacher~\cite{xxx} have shown that two PRF evaluations suffices, defining the hash functions to be $h_i(x)=F_K(\langle 1,x \rangle) + j\cdot F_K(\langle 2,x \rangle) \bmod m$, where~$m$ is the bitlength of the Bloom filter.\footnote{Actually, they define $h_i(x)=F_{K_1}(x)  + j\cdot F_{K_2}(x) \bmod m$, using distinct keys instead of domain-separation, but the constructions are equivalent with respect to security.} We call this the \emph{linear PRF-hash} Bloom Filter.  In this section, we show that this scheme is both correct and private with respect to our SS notion.  

Looking ahead to the next section, we will analyze a closely related construction, which was attacked by Neidermayer et al.~\cite{xxx} in the context of private medical-record linkage.  It uses the same linear-PRF hash functions, but has a $\Rep$ algorithm that that does not ensure that the hash functions are called on \emph{distinct} values.  We show that this is \emph{not} private with respect to our SS notion for a leakage function that leaks nearly the entire set~$S$. \todo{It would be better, perhaps, to show that it is not OW secure, since this implies it isn't SS secure, and OW security is what Neidermayer et al. violate.}  

Taken together, the results of this section and the next highlight how sensitive the privacy of linear PRF-hash constructions is to usage.


\heading{Correctness. }
For now, let $n,k,m \geq 0$.  Then $\setprim_{\mathrm{lin}}= (\Rep, \Qry)$ is defined as in Figure~\ref{fig:lin-and-ds} (left side).  The following result shows, informally, that if~$F$ is a good PRF, then $\setprim_{\mathrm{lin}}$ is correct against adaptive error-finding adversaries.

\begin{theorem}\label{thm1}\label{thm:lin-correctness}
Fix $k,m,n,r>0$, a set~$\elts$, and let $\setprim_{\mathrm{lin}}= (\Rep, \Qry)$ be the set-membership data structure (over~$\elts$) described in Figure~\ref{fig:lin-and-bin-lin}.  For any adversary~$A$ that makes a total of~$q$ queries to its oracle and has time-complexity~$O(t)$, there exist prf-adversaries~$B_1,B_2$ (explicitly constructed in the proof of this theorem) such that
\[
\AdvCorrect{\setprim_{\mathrm{lin}},r}{A} \leq  \AdvPRF{F}{B_1} + \AdvPRF{F}{B_2}  +{\dbinom{q}{r}} \left( (1-e^{-kn/m})^k + O(1/n) \right)^r\,.
\]
Here, $B_1$ and $B_2$, each ask~$q$ oracle queries and have time complexity~$O(t+qm)$.
\end{theorem}
The proof appears in Appendix~\ref{app:xxx}.  We note that Kirsch and Mitzenmacher~\cite{xxx} show
that the ($r=1$, zero-query) FP-probability for~$\setprim_\mathrm{lin}$ is
always within $O(1/n)$ of $(1-e^{-kn/m})^k$, although the asymptotic
rate of convergence is faster than~$1/n$.   As in previous cases, the multiplicative term $\dbinom{q}{r}$ arises when switching from the adaptive to zero-query settings.

\heading{Privacy. }
The SS-privacy of the linear-PRF hash construction follows directly from~Theorem~\ref{thm:xxx}.
\begin{theorem}\label{thm:lin-privacy}
Fix $k,m,n,r>0$, a set $\elts$, and let $\setprim_{\mathrm{lin}}= (\Rep, \Qry)$ be the set-membership data structure (over~$\elts$) described in Figure~\ref{fig:lin-and-bin-lin}. For any set~$S \in 2^{\elts}$, let $\leak(S)=|S|$.  There exists a simulator~$\Sim$ and an adversary~$B$ such that, for any adversary~$A$
\[
\AdvPrivSS{\setprim_{\mathrm{ds}},\leak}{A,\Sim} \leq  \AdvPRF{F}{B}
\]
%(Both~$B$ and~$\Sim$ are explicitly constructed in the proof of this theorem.)
\end{theorem}
\begin{proof}
It suffices to give the decomposition of $\Rep$ in to $\Rep_1$ and $\Rep_2$ as required by Theorem~\ref{thm:xxx}.
On input a set of distinct elements
$S=\{x_1,x_2,\ldots,x_n\}$, algorithm~$\Rep_1$ returns $s_1=\langle 1,x_1 \rangle, s_2=\langle 2,x_1 \rangle, s_3 = \langle 1,x_2\rangle, s_4=\langle 2,x_2 \rangle,\ldots,s_{2n-1}=\langle 1,x_n \rangle, s_{2n}=\langle 2,x_n \rangle$.

On input $n,F_K(s_1),F_K(s_2),\ldots,F_K(s_{2n-1}) , F_K(s_{2n})$, algorithm $\Rep_2$ initializes an all-zero array~$M$ of~$m$-bits, and sets the positions $M[F_K(s_i)+j\cdot F_K(s_{i+1})\bmod m]$ to 1, for each $j \in [k]$.  It returns~$M$ as $\pubaux$.
\end{proof}


%%%%%%%%%%%%%%%%% BIGRAM LINEAR-PRF (AKA "NEIDERMAYER")  %%%%%%%%%%%%%%%%%%%%
\subsection{Bigram Linear-PRF Construction}
%\heading{Bigram Linear-PRF construction (``Neidermayer''). }
%\heading{Correctness.} 
%\todo{Correctness theorem for this scheme.  Should follow pretty closely the one for ``plain'' linear-PRF.}\tsnote{Actually not sure this is that interesting, but for parallel structure one would want to give it.}
\heading{Privacy. }
Consider the set-membership data structure given in Figure~\ref{fig:lin-and-bi-lin} (right side), which we call the \emph{bigram linear-PRF} construction.  This construction is of practical interest, as it has been suggested as a way to address the privacy-preserving record-linkage problem, in particular for linking common records in medical databases~\cite{neidermayer,xxx}. 
We only address the privacy of this construction, as the correctness follows immediately from the correctness of the Linear-PRF contruction.  The privacy, however, does not.  Here we exhibit an efficient SS-privacy attack even when simulator is provided nearly the entire set.  Consider the following ``all-but-last'' leakage function.  On input a set~$S=\{x_1,x_2,\ldots,x_n\}$, it assigns $x'_n \gets x_n[1:|x_n|-1]$ and returns $\mathsf{leak_{abl}}(S)=\{x_1,x_2,\ldots,x'_n\}$.  \todo{OW privacy proof? Or just some text that says that Neidermayer et al show something stronger~---OW-privacy violation---~albeit with $n=10000, m=10000\alpha$ for an $\alpha$ that I need to look up.}

\begin{figure}
\centering
\fpage{.65}
{
\adversaryv{$A$}\\
On input $\emptystring$:\\
\nudge $x_1 \gets \mathrm{cab}$\\
\nudge $x_2 \gets \mathrm{bab}$\\
\nudge $z \getsr \bits$\\
\nudge if $z=1$ then $x_3 \gets \mathrm{caba}$\\
\nudge else $x_3 \gets \mathrm{cabc}$\\
\nudge Return $S=\{x_1,x_2,x_3\}$\\
On input $\pub \neq \emptystring$:\\
\nudge $M_1,M_2,M_3 \gets \pub$\\
\nudge $M_{1,2} \gets M_1 \vee M_2$\\
\nudge if $z=1 \wedge \left(3k < \mathsf{hamming}(M_{1,2} \wedge M_3) = 4k\right)$ then \\
\nudge\nudge Return 1 \\
\nudge if $z=0 \wedge \left(\mathsf{hamming}(M_{1,2} \wedge M_3) \leq 3k\right)$ then \\
\nudge\nudge Return 1 \\
\nudge Return 0
}
\caption{Adversary for Theorem~\ref{thm:bi-lin-ss}. The function $\mathsf{hamming}(M)$ returns the hamming weight of~$M$.}
\label{fig:adv-bi-lin-ss}
\end{figure}

\begin{theorem}\label{thm:bi-lin-ss}
Let $\setprim_{\bilin}= (\Rep, \Qry)$ be the set-membership data structure described in Figure~\ref{fig:neidermayer}, but instantiated with $F_{K}$ replaced by a random functions~$\rho$.  Let $\leak_{\mathsf{abl}}$ be the leakage function just described.   Let~$\elts=\{\mathrm{a,b,c}\}^*$. Let $n=3$,  $m=3\alpha $ for integer $\alpha >> 1$, and let, $k = \lceil \ln(2) \alpha \rceil$.\tsnote{This value of~$k$ is from the classical analysis for minimizing the FP rate for a Bloom filter, and is set this way for compatibility with classical results.  It isn't used in any essential way in the proof sketch.}  Let~$A$ be the adversary in Figure~\ref{fig:adv-bi-lin-ss}.  Then for any simulator $\Sim$, the advantage $\AdvPrivSS{\setprim_{\bilin},\leak_{\mathsf{abl}}}{A,\Sim}$ can be made arbitrarily close to $1/2$ by choice of~$\alpha$.
\end{theorem}
\begin{proof}(Sketch.)
On input the set~$S$ produced by~$A$, the $\Rep$ algorithm turns $x_1=$'cab', $x_2=$'bab' and $x_3$ into their respective bigram sets: $B_{x_1}=\{\sqcup\mathrm{c},\mathrm{ca},\mathrm{ab},\mathrm{b}\sqcup\}$, $B_{x_2}=\{\sqcup\mathrm{b},\mathrm{ba},\mathrm{ab}, \mathrm{b}\sqcup \}$; the set $B_{x_3}$ will be either $\{\sqcup\mathrm{c},\mathrm{ca},\mathrm{ab},\mathrm{ba}, \mathrm{a}\sqcup\}$, or $\{\sqcup\mathrm{c},\mathrm{ca},\mathrm{ab},\mathrm{bc}, \mathrm{c}\sqcup\}$, each with probability 1/2.  It then produces $\alpha$-bit arrays $M_1, M_2, M_3$ by hashing the corresponding bigram sets using the hash functions $h_j(s)=\rho_1(s)+ j\rho_2(s) \bmod m$.  Note that for sufficiently large~$m$, each of $M_1,M_2$ will contain exactly $4k$ positions set to 1; this is with overwhelming probability, up to collisions in the hash functions for different bigrams within a given bigram set.  Likewise, $M_{1,2}=M_1 \vee M_2$ will have exactly $6k$ positions set to 1, because $B_{x_1}$ and $B_{x_2}$ share $\mathrm{ab}$ and $\mathrm{a}\sqcup$.

In the case that $B_{x_3}=\{\sqcup\mathrm{c},\mathrm{ca},\mathrm{ab},\mathrm{ba}, \mathrm{a}\sqcup\}$, only one of its elements ($\mathrm{a}\sqcup$) is not in either of $B_{x_1},B_{x_2}$.  Hence, the hamming weight of $M_{1,2}\wedge M_3$ will be $4k$.  On the other hand, when $B_{x_3} = \{\sqcup\mathrm{c},\mathrm{ca},\mathrm{ab},\mathrm{bc}, \mathrm{c}\sqcup\}$, two of its elements ($\mathrm{bc}, \mathrm{c}\sqcup$) are not covered by $B_{x_1},B_{x_2}$.  In this case, the hamming weight of $M_{1,2} \wedge M_3$ will be $3k$.  Thus $\ExpPrivSSreal{\setprim_{\bilin}}{A}=1$ with overwhelming probability.  

On the other hand, the simulator~$\Sim$ is given 'cab','bab','cab' as
input.  It can perfectly simulate $M_1,M_2$ by lazily sampling random
function~$\rho$ itself.  However, it has no information about the
bit~$z$, so at best it can cause~$A$ to output 1 with
probabilility~$1/2$.  That is $\ExpPrivSSsim{\setprim_{\bilin}}{A,\Sim} \leq 1/2$.
\end{proof}




%%%%%%%%%%%%%%%%% Bigram DOMAIN-SEPARATED PRF %%%%%%%%%%%%%%%%%%%%
\subsection{Bigram Domain-Separated PRF-Hash}
We note that Neidermayer et al. in~\cite{xxx} suggest replacing the linear-PRF hash functions in the previous construction with independently keyed PRFs as a countermeasure to their attack.  In terms of SS-privacy, however, this change offers no help.  The following theorem is proved via a nearly identical argument as Theorem~\ref{thm:bi-lin-ss}

\begin{theorem}\label{thm:bi-ds-ss}
Let $\setprim_{\bilin}= (\Rep, \Qry)$ be the set-membership data structure described in Figure~\ref{fig:neidermayer}, but instantiated with $F_{K}$ replaced by a random functions~$\rho$.  Let $\leak$ be the leakage function described just prior to Theorem~\ref{thm:bi-lin-ss}.   Let~$\elts=\{\mathrm{a,b,c}\}^*$. Let $n=3$,  $m=3\alpha $ for integer $\alpha >> 1$, and let, $k = \lceil \ln(2) \alpha \rceil$.\tsnote{This value of~$k$ is from the classical analysis for minimizing the FP rate for a Bloom filter, and is set this way for compatibility with classical results.  It isn't used in any essential way in the proof sketch.}  Let~$A$ be the adversary in Figure~\ref{fig:adv-bi-lin-ss}.  Then for any simulator $\Sim$,
the advantage $\AdvPrivSS{\setprim_{\mathrm{ds}},\leak}{A,\Sim}$ can be made arbitrarily close to $1/2$ by choice of~$\alpha$.
\end{theorem}

\todo{OW privacy proof?  Distributional SS privacy proof?}

%%%%%%%%%%%%%%%%% GARBLED BF  %%%%%%%%%%%%%%%%%%%%
\subsection{Garbled Bloom Filter}
\begin{figure}
\centering
\fpage{.75}
{
\algorithmv{$\Rep(S)$}\\
$M \gets (\bot)^m$\\
$K_1,K_2,\ldots, K_k \getsr \calK$\\
for $x \in S$\\
\nudge $\finalshare \gets 0^\lambda$; $\emptyslot \gets -1$\\
\nudge for $j \in \{1,2,\ldots, k\}$\\
\nudge \nudge $h_j \gets H_{K_j}(x)$\\
\nudge \nudge if $M[h_j] \neq \bot$ then\\
\nudge \nudge \nudge $\finalshare \gets \finalshare \xor M[h_j]$\\
\nudge \nudge else if $\emptyslot = -1$ then\\
\nudge \nudge \nudge $\emptyslot \gets h_j$\\
\nudge \nudge else\\
\nudge \nudge \nudge $M[h_j] \getsr \bits^\lambda$\\
\nudge \nudge \nudge $\finalshare \gets \finalshare \xor M[h_j]$\\
\nudge $M[\emptyslot] \gets \finalshare \xor x$\\
for $i \in \{1,2,\ldots, m\}$\\
\nudge if $M[i] = \undefined$ then\\
\nudge \nudge  $M[i] \getsr \bits^\lambda$\\
$\pubaux \gets M$; $\privaux \gets \langle K_1,K_2,\ldots,K_k \rangle$ \mbox{ (if secret hash version)}\\
$\pubaux \gets \langle M,K_1,K_2,\ldots,K_k \rangle$; $\privaux \gets \emptystring$ \mbox{(if public hash version)}\\

\medskip
\algorithmv{$\Qry(\pubaux,\privaux,\qry_x)$}\\
if $\privaux \neq \emptystring$ then\\
\nudge $M \gets \pubaux$, $K_1,K_2,\ldots,K_k \gets \privaux$\\
else\\
\nudge $M,K_1,K_2,\ldots,K_k \gets \pubaux$\\
$x' \gets 0^\lambda$\\
for $j \in \{1,2,\dots,k\}$\\
\nudge $h_j \gets H_{K_j}(x)$\\
\nudge $x' \gets x' \xor M[h_j]$\\
if $x' =x $ then Return 1\\
Return 0
}
\caption{ $\Rep$ and $\Qry$ algorithms for garbled Bloom
  filter $\setprim_{\mathrm{garbled}}$ in the ROM for a hash function $H\colon \calK \times 
 \elts \to [m]$. The query set is $\calQ=\{\qry_x\colon
  \univ \to \bits\}$ where $\qry_x(S)=1 \Leftrightarrow x \in S$.}

\end{figure}


\begin{figure}
\centering
\hfpagess{.45}{.5}
{
\adversaryv{$A^{\TestOracle}(\pubaux)$}\\
$M \gets \pubaux$\\
for $1 \leq i_1 < i_2 < \cdots < i_k \leq m$\\
\nudge $x \gets M[i_1] \xor M[i_2] \xor \cdots \xor M[i_k]$\\
\nudge if $\TestOracle(x)=1$ then Return $\bot$
}
{
\adversaryv{$A^{\TestOracle}(\pubaux)$}\\
$M,K_1,K_2,\ldots,K_k \gets \pubaux$\\
for $1 \leq i_1 < i_2 < \cdots < i_k \leq m$\\
\nudge $x \gets M[i_1] \xor M[i_2] \xor \cdots \xor M[i_k]$\\
\nudge $x' \gets H(1,x) \xor H(2,x) \xor \cdots \xor H(k,x)$\\
\nudge if $x=x'$ then\\
\nudge\nudge $\TestOracle(x)$\\
\nudge\nudge Return $\bot$\\
}

\caption{Adversary for Theorem~\ref{thm:xxx}.  \textbf{Left:} secret hash case, \textbf{Right:} public hash case.}
\label{fig:adv-bi-lin-ss}
\end{figure}



\begin{verbatim}
%------------------------------------%
\end{verbatim}

\todo{Below are two lists of things that we should show.}

\heading{Correctness. }  Here we give security proofs and attacks with respect to the correctness notions.
\begin{itemize}
\item Show Bloom Filter is correct in the ROM. \jnote{Should follow along the lines
of Theorem~2, but taking into account adversary's oracle queries.} [DONE]
\item correctness security/attacks on the bigram/PRF-based double-hashing construction ($h_j(x) = F_{K1}(x) + j\cdot F_{K2}(x) \bmod m$ where the~$x$ are bigrams of names, etc.)
\item correctness-security/attacks on the bigram/domain-separated PRF construction ($h_j(x)=F_K(\langle j,x \rangle) \bmod m$, or just assume range of~$F$ is $[m]$.) 
\item correctness-security/attacks Dong, Chen, Wen ``garbled Bloom filter'' construction

\end{itemize}

\heading{Privacy. } 
\begin{itemize}
\item Prove that the basic BF, in the ROM, is private.  \tsnote{Not SS, should be OW}
\item Show that the bigram DS-PRF construction is SS \tsnote{Shows Neidermayer's suggested countermeasure works}
\item Show that the basic BF DS-PRF construction is SS [DONE]
\item Prove privacy bounds for the constructions from NY \tsnote{Follows from Jon's meta-theorem}

\item Cast Neidermayer et al.\ attacks on the bigram/PRF-based double-hashing construction (see their paper) into our formalism.  \tsnote{Note that their attack would allow one to recover \emph{every} name, not just one.}

\item Prove a privacy bound for the bigram/PRF-based double-hashing construction that Neidermayer et al.\ attacks. I expect that the provable security bound is considerably worse than the min-entropy because of the way the representation is made.  (Essentially, the Hamming weight of the representation is a good estimate of the length of the longest surname in filter.)
\jnote{Maybe tailor leakage function appropriately, i.e., so that it only leaks the number of bigrams?}

\item Attack priv-security of Dong, Chen, Wen ``garbled Bloom Filter'' construction when the hash functions are public.\tsnote{I give the attack in the related work section.  Copy it here and flesh it out.}  Prove priv-security privacy of construction when hash functions are secret.  \tsnote{Makes the point that the application setting really matters.}

\item Prove privacy bounds of other suggestions, such as the record-level BF from Durham et al.?  This would be a nice pairing with the Neidermayer et al.\ results \tsnote{No time.}

\end{itemize}
