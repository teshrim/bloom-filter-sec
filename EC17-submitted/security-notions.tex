\section{Security notions}
\tsnote{ALL BELOW IS OLD!  Treat accordingly.}

\paragraph{Alternative Security Definitions. }
NOTE: this section is very sketchy.  It was actually written first, before I realized how underspecified the [NY] syntax is.

The first alternative I'll call “partial-inversion resistance”.  It captures privacy of the Bloom filter input.  Sticking with the [NY] notation, let $B=(B_1, B_2)$ be the bloom filter population and querying algorithms, respectively.  Fix integers $n,m > 0$.  The (still informal) experiment is as follows: 

$S=\{x_1,x_2,..., x_n\}$ sampled from U according to some distribution $D_U$.  $B_1(S,m)$ returns~$M$, an array of~$m$ bits. Adversary~$\advA$ is given an oracle for $B_2(M,\cdot)$  (and assume $B_2$ outputs 0 or 1).  Adversary~$\advA$ halts with output $z$, and wins if $z \in S$.

Essentially, $B_1$ is a mapping from $n$-tuples (over $S$) to $m$-tuples (over $\{0,1\}$, in typical bloom filters).  The adversary's job is to determine some component of the input to $B_1$, i.e. partial inversion of M. \tsnote{How does this relate to Ristenpart et al.'s recent work on model-inversion attacks?}

Definitional choices remain...

Should $B_1$ be randomized, as it is in [NY]?  Traditionally, no: population of the Bloom filter array is a deterministic process.  (Here I'm viewing the sampling of the hash functions as external to $B_1$.)   An implication of deterministic $B_1$ seems to be that distribution $D_U$ must have reasonable min-entropy.  
Should the hash functions used by $B_1$ and $B_2$ in traditional Bloom filters be explicit in the syntax, and explicitly surfaced in the security notion?  
Should they be given as oracles to $\advA$?  That is, should we assume that attackers know the hash functions begin used? Should $\advA$ be given $m = |M|$ as input?  That is, should we assume that attackers know the size of the representation?
Should $B_2$ be deterministic?  Traditionally, yes.  But I allowing it to be stateful admits more advanced Bloom filter constructions, e.g. counting Bloom filters.  [NY] do consider the case that $B_2$ is randomized (“unstable representations”), but I don't know of any Bloom filters whose array representations are probabilistic over time.

Note that false positives do not count as wins in this experiment.  They could be by, say, changing the winning condition to “Adversary $\advA$ wins if $B_2(M,\cdot)$ ever returns 1”.  But I think this muddles together separate security goals --- forging a bogus input, and learning one of the real inputs.  So I prefer to keep them separate.  (I'd want to rewrite the [NY] notion, however.)
