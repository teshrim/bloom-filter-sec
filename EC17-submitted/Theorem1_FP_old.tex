\acnote{I am moving comments to end of the proof in tex file and commenting them as the labels are getting mangled}
\heading{Linear-PRF construction. }
Here we consider a hash-based filter that was attacked by Neidermayer et al. in~\cite{xxx}.  Let $B_{\mathrm{lin}}=(\Hash,\Rep,\Qry)$ be defined as follows.  Fix non-empty sets ~$\univ$, $\calK$ and integers $k,m,n>0$.  Let $F\colon\mathcal{K} \times \univ \to [m]$ be a function family.  The $\Hash$ algorithm samples two keys $K_0,K_1 \getsr \mathcal{K}$ and defines $h_j(x) = F_{K_0}(x) + j\cdot F_{K_1}(x) \bmod m$ for $j \in [1,k]$.  The $\Rep$ and $\Qry$ algorithms are the standard (determinisitic) BF ones. The following result shows, informally, that if~$F$ is a good PRF and the (non-adaptive) FP rate is sufficiently small, then $B_\mathrm{lin}$ is secure against adaptive FP-finding adversaries.

\begin{theorem}\label{thm1}
Fix $k,m,n,r>0$, and let $B_\mathrm{lin}=(\Hash,\Rep,\Qry)$ be the hash-based filter just described. 
For any adversary~$A$ that makes a total of~$q$ queries to its oracle
and has time-complexity~$O(t)$,
there exist adversaries~$B_1,B_2$ (explicitly constructed in the proof of this theorem) such that
\[
\AdvFPSecHash{B_{\mathrm{lin}},r}{A} \leq  \AdvPRF{F}{B_1} + \AdvPRF{F}{B_2}  +{\dbinom{q}{r}} \left( (1-e^{-kn/m})^k + O(1/n) \right)^r\,.
\]
Here, $B_1$ and $B_2$, each ask~$q$ oracle queries and have time complexity~$O(t+qm)$.
\end{theorem}
Before giving the proof, we note that Kirsch and Mitzenmacher~\cite{xxx} show
that the ($r=1$, non-adaptive) FP-probability for~$B_\mathrm{lin}$ is
always within $O(1/n)$ of $(1-e^{-kn/m})^k$, although the asymptotic
rate of convergence is faster than~$1/n$.  The factor of $\dbinom{q}{r}$ arises as a result
of removing adaptivity.  Finally, $\dbinom{q}{r} \leq q^r$ (with
reasonable tightness when~$r$ is small) in which case the final term in the bound behaves as 
$(q (1-e^{-kn/m})^k + O(q/n) )^r$; loosely, the bound one expects for
trying to find a single FP in each of~$r$ independent``rounds'', each round
consisting of~$q$ attempts.
%Before giving the proof, we note that  $(1-e^{-kn/m})^k \rightarrow 0$ as the ratio between the bit-length of the representation~$M$ and the size of the represented set~$S$, i.e.,$m/n$ tends to $\infty$.
%
\begin{proof}[\Cref{thm1}]
The proof of the theorem will use a game playing argument as shown in \Cref{fig:Game}. $\game{0}(A)$ exactly simulates $\ExpFPSecHash{B_\mathrm{lin},r}{A}$ with hash functions $h_j(x) =  F_{K_0}(x) + j\cdot F_{K_1}(x) \bmod m$. So,

\begin{equation}
\AdvFPSecHash{B_{\mathrm{lin}},r}{A} = \Prob{\game{0}(A)=1}\label{eq:0}
\end{equation}

 In $\game{1}(A)$, $F_{K_1}$ is replaced by a random function, whereas in $\game{2}(A)$, both $F_{K_0}$ and $F_{K_1}$ are replaced by random functions as shown below. Here, $\rho_1$ and $\rho_2$ are random functions sampled from $\Func{\univ,[m]}$.
%\begin{align*}
%\game{1}(A)\, : \, h_j(x) &= ( F_{K_0}(x) + j\cdot \rho_2(x)) \bmod m\\
%\game{2}(A)\, : \, h_j(x) &= ( \rho_1(x) + j \cdot \rho_2(x)) \bmod m
%\end{align*}


\begin{figure}
\centering
\fpage{0.9}{
\hpagessl{0.5}{0.5}
{
$\adversaryv{B_1^{\calO}}$\\
$K_1 \getsr \calK $\\
$\calS \getsr \distr{\univ}{n}$\\
$M \getsr \Rep^{\HashOracle}(\calS)$\\
When $A$ asks $\QryOracle(x)$:\\
$\nudge$ Ret $\Qry^{\HashOracle}(M,x)$\\
When $A$ halts with output $\calZ'$: \\
\nudge if $|\calZ'| < r$ or $\calZ' \cap \calS \neq \emptyset$ then \\
\nudge \nudge Ret 0\\
\nudge if $\exists z \in \calZ'$ s.t. $\Qry^{\HashOracle}(M,z)=0$ then\\
\nudge \nudge Ret 0\\
\nudge Ret 1\\\\
\oracle{$\HashOracle(x)$}\\
$ \mathrm{Ret} \nudge F_{K_1}(x) + i\cdot \calO(x)$, for $i\in[1,k]$\\
}
{
$\adversaryv{B_2^{\calO}}$\\
%$d \getsr \bits $\\
%$\rho_2 \getsr Func(\univ, [m])$ \\
$S \getsr \distr{\univ}{n}$\\
$M \getsr \Rep^{\HashOracle}(\calS)$\\
When $A$ asks $\QryOracle(x)$:\\
$\nudge$ Ret $\Qry^{\HashOracle}(M,x)$\\
When $A$ halts with output $\calZ'$: \\
\nudge if $|\calZ'| < r$ or $\calZ' \cap \calS \neq \emptyset$ then \\
\nudge \nudge Ret 0\\
\nudge if $\exists z \in \calZ'$ s.t. $\Qry^{\HashOracle}(M,z)=0$ then\\
\nudge \nudge Ret 0\\
\nudge Ret 1\\\\
\oracle{$\HashOracle(x)$}\\
%if $d=0$\\
$a \getsr [m]$\\
if $T[x] \neq \undefined$\\
\nudge $a \gets T[x]$\\
$T[x] \gets a$\\
$\mathrm{Ret} \nudge \calO(x) + i\cdot a$, for $i\in[1,k]$
}
}
\caption{\Cref{thm1}:Adversary $B_1, B_2$ simulating $\game{1}(A)$ and $\game{2}(A)$ respectively}\label{fig:BGame1}
\end{figure}

The PRF security notion is shown in \Cref{fig:prf}. Let
prf-adversaries $B_1, B_2$ be as shown in \Cref{fig:BGame1}.
Adversary~$B_1$ is constructed so that, if its challenge bit~$b=1$,
then it exactly simulates $\game{0}$ for~$A$; if $b=0$, it
simulates~$\game{1}$.  Similarly, adversary~$B_2$ simulates~$\game{1}$
if its challenge bit~$b=1$, and $\game{2}$ otherwise.  We can conclude
that 
\begin{align*}
\Prob{\ExpPRF{F}{B_1} = 1\,|\,b=1} &= \Prob{\game{0}(A)=1} \\%\label{eq:m1}\\
\Prob{\ExpPRF{F}{B_1} = 1\,|\,b=0} &= 1-\Prob{\game{1}(A)=1} \\%\label{eq:m3}\\
\Prob{\ExpPRF{F}{B_2} = 1\,|\,b=1} &= \Prob{\game{1}(A)=1} \\%\label{eq:m2}\\
\Prob{\ExpPRF{F}{B_2} = 1\,|\,b=0} &= 1-\Prob{\game{2}(A)=1} \\%\label{eq:m4}
\end{align*}
<<<<<<< HEAD
%
\noindent
By a standard conditioning of the PRF advantage, we have 
%From \Cref{eq:m1,eq:m2,eq:m3,eq:m4},
%\begin{align}
%\nonumber \Prob{\ExpPRF{F}{B_1} = 1} &= .5\Prob{\ExpPRF{F}{B_1}=1\,|\,b=0} + .5\Prob{\ExpPRF{F}{B_1} = 1\,|\,b=1}\\
%\nonumber 2\Prob{\ExpPRF{F}{B_1} = 1} &= 1-\Prob{\game{1}(A)=1} + \Prob{\game{0}(A)=1}\\
%\nonumber (2\Prob{\ExpPRF{F}{B_1} = 1} - 1)  &= \Prob{\game{0}(A)=1} - \Prob{\game{1}(A)=1}\\
% \AdvPRF{F}{B_1} &= \Prob{\game{0}(A)=1} - \Prob{\game{1}(A)=1} \label{eq:11}\\
%\nonumber& &\\
%\nonumber \Prob{\ExpPRF{F}{B_2} = 1} &= .5\Prob{\ExpPRF{F}{B_2}=1\,|\,b=0} + .5\Prob{\ExpPRF{F}{B_2} = 1\,|\,b=1}\\
%\nonumber 2\Prob{\ExpPRF{F}{B_2} = 1} &= 1-\Prob{\game{2}(A)=1} + \Prob{\game{1}(A)=1}\\
%\nonumber (2\Prob{\ExpPRF{F}{B_2} = 1} - 1)  &= \Prob{\game{1}(A)=1} - \Prob{\game{2}(A)=1}\\
% \AdvPRF{F}{B_2} &= \Prob{\game{1}(A)=1} - \Prob{\game{2}(A)=1} \label{eq:12}
%\end{align}
%\noindent
%Adding \Cref{eq:11,eq:12}, we have
\begin{equation*}
\Prob{\game{0}(A)=1} = \AdvPRF{F}{B_1} +\AdvPRF{F}{B_2} + \Prob{\game{2}(A)=1}%\label{eq:13}
\end{equation*}


In $\game{3}(A)$, we replace the random functions used in
$\HashOracle$ (in $\game{2}(A)$) with lazy sampling of elements from
$[m]$.  This change does not alter the behavior of the oracle, as the
random functions are (by the no-pointless-queries assumption on~$A$) always called on a new
value of~$x$.
%
Also, the $\QryOracle$ oracle now sets $\bad$ to $\true$ if the
queries indexed by the set $\mathcal{I}$ fail to be false-positives,
or if any of the remaining queries are false-positives. It can be
observed that the $\bad$ flag has no impact on the output of the
$\QryOracle$, nor does it affect  the value returned
by~$\game{3}$. Thus we have $\Prob{\game{2}(A)=1} =
\Prob{\game{3}(A)=1}$ and
\[
\Prob{\game{3}(A)=1 \wedge \game{3}(A): \neg\bad} = \Prob{\game{3}(A)=1} \Prob{\game{3}(A) : \neg\bad}%\label{eq:1} \\
\]

We note that $\game{3}$ and $\game{4}$ are
``identical-untill-$\bad$''~\cite{BeRo-gameplaying}.  In particular,
in $\game{4}$ returns 0 when $\game{3}$ would have returned 1 (i.e.,
the adversary outputs~$r$ false-positives) but $\bad=\true$.  
Moreover,  $\game{4}(A)$ returning 1 implies that $\bad$ is never set to $\true$. Therefore,
$\Prob{\game{4}(A)=1} = \Prob{\game{3}(A)=1 \wedge \neg\bad} =
\Prob{\game{3}(A)=1} \Prob{\game{3}(A) : \neg\bad}$, by our previous observation.

Before continuing the proof, a bit of intuition for the introduction
of the $\bad$-flag in games~$\game{3},\game{4}$.   Intuitively, if the index
set~$\mathcal{I}$ exactly predicts which of the $\QryOracle$ will be
false-positives (and, conversely, which will not be), then a
non-adaptive FP-finding adversary can ``guess'' which of~$A$'s queries require it
to return 1 when simulating the $\QryOracle$, and which require it to
return 0.  In this way, no queries ever actually need to be made.
Moving to a non-adaptive setting will allow us to connect to classical
results on the FP-rate of Bloom filters; more in a moment. 

Now, consider $\Prob{\game{3}(A) : \neg\bad}$.  The indicated event
happens iff the $r$-set~$\calI$ contains exactly the indices of
$\QryOracle$-queries that are false-positives.  Since the value
of~$\bad$ does not alter the responses of~$\QryOracle$, the
set~$\mathcal{Z}$ output by~$A$ and the set~$\calI$ are independent.
Thus $\bad$ remains $\false$ after~$A$ halts with probability
$1/\dbinom{q}{r}$, and we have 
\begin{equation*}
\Prob{\game{3}(A)=1} = \frac{1}{\dbinom{q}{r}}\Prob{\game{4}(A)=1}
\end{equation*}

\fixme{Tom is here in rewriting the proof.}
Moving on $\game{4}(A)$, $A$ wins $\ExpFPSecHash{B_{\mathrm{lin}},r}{A}$ after finding $r$ false positives indexed by $\calI$. Therefore, without loss of generality, it can be assumed that $A$ halts just after it finds the $r$ false positives, and outputs $\calZ$(of size $r$). Further, we can also assume that every point in $\calZ$ has already been queried to $\QryOracle$. This assumption can be  easily enforced (in the original security definitions) by counting as part of the number of queries~$q$, the points in $\mathcal{Z}$. Hence, if $\game{4}(A)=1$, there is a non-adaptive adversary~$D$ that finds~$r$ false positives against $\tilde{B}_{\mathrm{lin}}$, which is $B_{\mathrm{lin}}$ but with $F_{K_1}$ and $F_{K_2}$ replaced by random functions $\rho_1$ and $\rho_2$, respectively as shown in \Cref{fig:D}. So,
=======
In $\game{3}(A)$, we replace the hash functions of $\HashOracle$ (in $\game{2}(A)$) with lazy sampling of elements from $[m]$. This brings us close to the classic Bloom filter's soundness condition, as now the hash indices sampled during a query are random points in $[m]$, and are independent of the hash indices used by the $\Rep$. As mentioned in \acnote{ref :{sec:intro}}, the soundness condition of a classic bloom filter is given as  $\Prob{h_1,h_2,\ldots,h_k \getsr \mathrm{Func}(\univ,[m])\colon M[h_1(y)]=1 \wedge \cdots \wedge M[h_k(y)]=1} \leq \epsilon$. Also, the $\QryOracle$ now sets $\bad$ to $\true$ if the queries indexed by the set $\mathcal{I}$ fail to be false-positives, or if any of the remaining queries are false-positives. It can be observed that the $\bad$ flag has no impact on the output of the $\QryOracle$, nor does it affect  the value returned by~$\game{3}$. Rather it lays foundation for $\game{4}(A)$. From these observations, 
	\begin{align}
	\Prob{\game{3}(A)=1 \wedge \neg\bad} &= \Prob{\game{3}(A)=1} \Prob{\game{3}(A) : \neg\bad}\label{eq:1} \\
	\Prob{\game{2}(A)=1} &= \Prob{\game{3}(A)=1}\label{eq:2}
	\end{align}
$\game{4}(A)$ is identical to $\game{3}(A)$ except it returns 0 if $\bad$ is set to true. So, by construction $\game{4}(A)$ and $\game{3}(A)$ are identical-untill-$\bad$. In addition, $\game{4}(A)$ returning 1 implies that bad is never set to true. Therefore,
\begin{equation}
\Prob{\game{4}(A)=1} = \Prob{\game{3}(A)=1 \wedge \neg\bad}\label{eq:3}
\end{equation}
From \Cref{eq:1,eq:2,eq:3}
\begin{equation}
\Prob{\game{4}(A)=1} =  \Prob{\game{2}(A)=1} \Prob{\game{3}(A) : \neg\bad} \label{eq:4}
\end{equation}
Since, probability of any set of $r$ queries indexed by set $\calI$ to be not false positives are equally likely to any other set sampled from $[\{1,2,\ldots,q\}]^r$, we have
\begin{equation}
\Prob{\game{3}(A) : \neg\bad} = \frac{1}{\dbinom{q}{r}}\label{eq:555}
\end{equation}
From \Cref{eq:4,eq:555}
\begin{equation}
\Prob{\game{4}(A)=1} = \Prob{\game{2}(A)=1}\frac{1}{\dbinom{q}{r}}\label{eq:6}
\end{equation}
In $\game{4}(A)$, $A$ wins $\ExpFPSecHash{B_{\mathrm{lin}},r}{A}$ after finding $r$ false positives indexed by $\calI$. Therefore, without loss of generality, it can be assumed that $A$ halts just after it finds the $r$ false positives, and outputs $\calZ$(of size $r$). Further, we can also assume that every point in $\calZ$ has already been queried to $\QryOracle$. This assumption can be  easily enforced (in the original security definitions) by counting as part of the number of queries~$q$, the points in $\mathcal{Z}$. Hence, if $\game{4}(A)=1$, there is a non-adaptive adversary~$D$ that finds~$r$ false positives against $\tilde{B}_{\mathrm{lin}}$, which is $B_{\mathrm{lin}}$ but with $F_{K_1}$ and $F_{K_2}$ replaced by random functions $\rho_1$ and $\rho_2$, respectively as shown in \Cref{fig:D}. So,
>>>>>>> d0916f8020d3d4c6ec862a055a653b9ded54a449
\begin{align}
\nonumber \Prob{\ExpFPSecHash{\tilde{B}_{\mathrm{lin}},r}{D} = 1} &= \Prob{\game{4}(A) = 1}\\
\mathrm{or} \nudge \AdvFPSecHash{\tilde{B}_{\mathrm{lin}},r}{D} &= \Prob{\game{4}(A) = 1}\label{eq:7}
\end{align}
Let $\calF$ be the event of getting a false positive in a classic Bloom filter. Since, $D$ is a non-adaptive adversary, for $r=1$, it exactly simulates the classic Bloom filter game. Also, as the elements of $\calI$ are randomly sampled from $[\{1,2,\ldots,q\}]^r$, the event of any query indexed by $\calI$ being a false positive is mutually independent of any other query indexed by the same set. Hence,
\begin{align}
\nonumber \AdvFPSecHash{\tilde{B}_{\mathrm{lin}},r}{D} &= [\Prob{\calF}]^r\\
&\leq \left( (1-e^{-kn/m})^k + O(1/n) \right)^r \label{eq:8}
\end{align}
Here, $\left( (1-e^{-kn/m})^k + O(1/n) \right)$ is the upper bound of $\calF$, as given by Kirsch et. al.\ref{KM}.\\
From \Cref{eq:7,eq:8},
\begin{equation}
\Prob{\game{4}(A)=1} \leq  \left( (1-e^{-kn/m})^k + O(1/n) \right)^r \label{eq:9}
\end{equation}
Using \Cref{eq:6} in \Cref{eq:9}
\begin{align}
\Prob{\game{2}(A)=1}\frac{1}{\dbinom{q}{r}} & \leq  \left( (1-e^{-kn/m})^k + O(1/n) \right)^r\\
\Prob{\game{2}(A)=1} &\leq \dbinom{q}{r}\left( (1-e^{-kn/m})^k + O(1/n) \right)^r\label{eq:10}
\end{align}

%\caption{Game playing argument}\label{fig:Game}
\begin{figure}
\fpage{.9}{
\hpagessl{.45}{.5}
{
\underline{\game{0}(A)}\\
$K_1, K_2 \getsr \calK$\\
$S \getsr \distr{\univ}{n}$\\
$M \getsr \Rep^{\HashOracle}(\calS)$\\
$\calZ \getsr A^{\QryOracle}(\calS)$\\
if $|\calZ| < r$ or $\calZ \cap \calS \neq \emptyset$ then \\
\nudge Ret 0\\
if $\exists z \in \calZ$ s.t. $\Qry^{\HashOracle}(M,z)=0$ then\\
\nudge Ret 0\\
Ret 1\\\\
%
\oracle{$\QryOracle(x)$}\\
Ret $\Qry^{\HashOracle}(M,x)$\\\\
%
\oracle{$\HashOracle(x)$}\\
for $i = 1$ to $k$\\
\nudge $h_i(x) = F_{K_1}(x)+i \cdot F_{K_2}(x) \bmod m$\\
Ret $\left(h_1(x),\ldots,h_k(x)\right)$
}
{
\underline{\game{1}(A)}\\
$K_1 \getsr \calK$\\
$\rho \getsr \Func{\univ,[m]}$\\
$S \getsr \distr{\univ}{n}$\\
$M \getsr \Rep^{\HashOracle}(\calS)$\\
$\calZ \getsr A^{\QryOracle}(\calS)$\\
if $|\calZ| < r$ or $\calZ \cap \calS \neq \emptyset$ then \\
\nudge Ret 0\\
if $\exists z \in \calZ$ s.t. $\Qry^{\HashOracle}(M,z)=0$ then\\
\nudge Ret 0\\
Ret 1\\\\
%
\oracle{$\QryOracle(x)$}\\
Ret $\Qry^{\HashOracle}(M,x)$\\\\
%
\oracle{$\HashOracle(x)$}\\
for $i=1$ to $k$\\
\nudge $h_i(x) = F_{K_1}(x)+i \cdot \rho(x) \bmod m$\\
Ret $\left(h_1(x),\ldots,h_k(x)\right)$
}
}
\fpage{.9}{
\hpagessl{.45}{.5}
{
\underline{\game{2}(A)}\\
$\rho_1,\rho_2 \getsr \Func{\univ,[m]}$\\
$S \getsr \distr{\univ}{n}$\\
$M \getsr \Rep^{\HashOracle}(\calS)$\\
$\calZ \getsr A^{\QryOracle}(\calS)$\\
if $|\calZ| < r$ or $\calZ \cap \calS \neq \emptyset$ then \\
\nudge Ret 0\\
if $\exists z \in \calZ$ s.t. $\Qry^{\HashOracle}(M,z)=0$ then\\
\nudge Ret 0\\
Ret 1\\\\
%
\oracle{$\QryOracle(x)$}\\
Ret $\Qry^{\HashOracle}(M,x)$\\\\
%
\oracle{$\HashOracle(x)$}\\
for $i=1$ to $k$\\
\nudge $h_i(x) = \rho_1(x)+i \cdot \rho_2(x) \bmod m$\\
Ret $\left(h_1(x),\ldots,h_k(x)\right)$
}
{
\underline{{$\game{3}(A)$},\fbox{$\game{4}(A)$}}\\
$c\gets 0$, $\bad\gets\false$\\
$\mathcal{I}\getsr [\{1,2,\ldots,q\}]^r$\\
$S \getsr \distr{\univ}{n}$\\
$M \getsr \Rep^{\HashOracle}(\calS)$\\
$\calZ \getsr A^{\QryOracle}(\calS)$\\
if $|\calZ| < r$ or $\calZ \cap \calS \neq \emptyset$ then Ret 0\\
if $\forall z \in \calZ,\,\Qry^{\HashOracle}(M,z)=1$ then \\
\nudge if $\bad=\true$ then \fbox{Ret 0}\\
\nudge Ret 1\\
Ret 0\\\\

\oracle{$\QryOracle(x)$}\\
$c \gets c+1$\\
$v \gets \Qry^{\HashOracle}(M,x)$\\
if $c \in \mathcal{I}$ and $v\neq 1$ then\\
\nudge $\bad \gets \true$ \\
if $c \not\in \mathcal{I}$ and $v\neq 0$ then\\
\nudge $\bad \gets \true$\\
Ret~$v$\\\\
%
\oracle{$\HashOracle(x)$}\\
$a,b \getsr [m]$\\
for $i = 1$ to~$k$\\
\nudge $v_i = a+i \cdot b \bmod m$\\
Ret $\left(v_1,\ldots,v_k\right)$
}
}
\caption{\Cref{thm1}:Game playing argument}\label{fig:Game}
\end{figure}

%\caption{$\Cref{thm1}:\game{5}(A)$ in game playing argument}\label{fig:game5}
%\begin{figure}
%\centering
%\fpage{0.8}
%{
%\hpagess{0.5}{0.4}{
%\underline{$\game{5}(A)$}\\
%$c \gets 0$\\
%$\mathcal{I}\getsr [\{1,2,\ldots,q\}]^r$\\
%$\calY \gets \emptyset$\\
%$S \getsr \distr{\univ}{n}$\\
%$M \getsr \Rep^{\HashOracle}(\calS)$\\
%$\calZ \getsr A^{\QryOracle}(\calS)$\\
%if $|\calY| < r$ or $\calY \cap \calS \neq \emptyset$ then\\
%\nudge Ret 0\\
%if $\forall y \in \calY,\,\Qry^{\HashOracle}(M,y)=1$ then \\
%\nudge Ret 1\\
%Ret 0
%}
%
%{
%\oracle{$\QryOracle(x)$}\\
%$c \gets c+1$\\
%if $c \in \mathcal{I}$ then\\
%\nudge $\calY \gets \calY \cup {x}$\\
%\nudge Ret 1\\
%else\\
%\nudge Ret 0 \\\\
%
%\oracle{$\HashOracle(x)$}\\
%$a,b \getsr [m]$\\
%for $i = 1$ to~$k$\\
%\nudge $v_i = a+i \cdot b \bmod m$\\
%Ret $\left(v_1,\ldots,v_k\right)$
%}
%}
%\caption{\Cref{thm1}:$\game{5}(A)$ in game playing argument}\label{fig:game5}
%\end{figure}

%\caption{PRF game}\label{fig:prf}
\begin{figure}
\centering
\fpage{0.45}{
\hpagess{0.4}{0.35}
{
$\experimentv{\ExpPRF{F}{B}}$\\
$K \getsr \calK $\\
$\rho \getsr Func(\univ,[m])$\\
$b \getsr \bits$\\
$b' \getsr B^\calO$\\
Ret $[b = b']$\\
}
{
$\oracle{\calO \smallskip(x)}$\\
if $ b = 1$ then\\
\nudge Ret $F_K(x)$\\
Ret $\rho(x)$\\
}
}
\caption{The PRF experiment for a function family $F \colon \calK
  \times \calX \to [m]$. }\label{fig:prf}
\end{figure}

%\caption{Adversary $B_1, B_2$ simulating $\game{1}(A)$ and $\game{2}(A)$ respectively}\label{fig:BGame1}

%\caption{Non-adaptive adversary $D$ simulating $\game{4}(A)$} \label{fig:D}
\begin{figure}
\centering
\fpage{0.7}{
\hpagess{0.55}{0.4}
{
$\adversaryv{D^{\QryOracle, \calS}}$\\
$c \gets 0$\\
$\mathcal{I}\getsr [\{1,2,\ldots,q\}]^r$\\
When $A$ asks $\QryOracle(x)$:\\
\nudge $c \gets c+1$\\
\nudge $r \gets 0$\\
\nudge if $c \in \mathcal{I}$\\
\nudge \nudge $r \gets \QryOracle(x)$\\
\nudge Ret $r$\\
When $A$ halts with output $\calZ$\\
\nudge Ret $\calZ$
}
{
\oracle{$\QryOracle(x)$}\\
 Ret $\Qry^{\HashOracle}(M,x)$\\\\
%
\oracle{$\HashOracle(x)$}\\
for $i = 1$ to~$k$\\
\nudge $v_i = \rho_1(x)+i \cdot \rho_2(x) \bmod m$\\
Ret $\left(v_1,\ldots,v_k\right)$
}
}
\caption{Non-adaptive adversary $D$ simulating $\game{4}(A)$} \label{fig:D}
\end{figure}	

From \Cref{eq:0,eq:10,eq:13} \\
\begin{align}
\nonumber \AdvFPSecHash{B_{\mathrm{lin}},r}{A} &- \Prob{\game{2}(A)=1} =  \AdvPRF{F}{B_1} + \AdvPRF{F}{B_2}   \\
\nonumber \AdvFPSecHash{B_{\mathrm{lin}},r}{A}  &=  \AdvPRF{F}{B_1} + \AdvPRF{F}{B_2} + \Prob{\game{2}(A)=1}  \\
\AdvFPSecHash{B_{\mathrm{lin}},r}{A} &\leq  \AdvPRF{F}{B_1} +
\AdvPRF{F}{B_2}  +{\dbinom{q}{r}} \left( (1-e^{-kn/m})^k + O(1/n) \right)^r
\end{align}

\iffalse
%Comments
\tsnote{I'll need to rewrite all below here.  The current text is too hand-wavy and fast.}
\tsnote{`In transitioning from $\game{2}$ to $\game{3}$ we
  replace the all-at-once sampling of random functions~$\rho_1,\rho_2$
with lazying sampling in the $\HashOracle$, which is the only place
these random functions are used.  This does not alter the input-output
behavior of the $\HashOracle$.''  Why do we do this?}
We also sample $r$~distinct random integers are chosen \tsnote{why?
  What are we setting up?} and the $\QryOracle$ oracle sets a $\bad$
flag if query response for any of the $r$ queries is false or any of
the $q-r$ queries return true. \tsnote{We
  set $\bad$ to $\true$ if the queries indexed by the
  set~$\mathcal{I}$ fail to be false-positives, or if any of the
  remaining queries \emph{are} false-positives.  Again, why are we
  doing this?  }
\tsnote{How does the last line follow from the games?  I mean, it
  isn't obvious to me that your $\game{5}$ is equivalent to $\game{4}$.  The
  latter looks at the adversary's output~$\mathcal{Z}$, but the former
ignores this and looks at this new set~$\mathcal{Y}$.  Why are these
equivalent with respect to the events $\game{4}(A)=1$ and
$\game{5}(A)=1$?  Just looking at the $\QryOracle$ in the two games,
it appears that they have pretty different behaviors. I mean, in
$\game{4}$ the oracle still returns what it is supposed to, even if
$\bad\gets\true$.  But in $\game{5}$ the oracle returns fixed values,
no matter if the query is or is not a FP.  You have to explain why
these changes are okay.  Also, why is this new set~$\mathcal{Y}$
necessary?  }
\tsnote{The point that needs to be made about the relationship between
  $\game{4}$ and $\game{5}$ is this.  $\game{4}$ returns 1 only if
  $\bad=\false$ at the end of the game.  In turn, this implies that
  the game managed to ``guess'', prior to the execution of~$A$, which
  of its queries would result in false-positives.  (By the way, we
  need to state an assumption here: that~$A$ halts and
  outputs~$\mathcal{Z}$ (of size~$r$) as soon as its queries
  reveal~$r$ false positives.  This is without loss of generality; say
why.  I guess you also need to assume that every point in
the~$\mathcal{Z}$ output by~$A$ has already been queried
to~$\QryOracle$.  Strictly speaking this is not without loss of
generality, but would be easily enforced (in the original
  security definitions) by counting as part of the number of
  queries~$q$ the points in~$\mathcal{Z}$.  Without these assumptions,
things don't quite work.)  Hence, if $\game{4}(A)=1$ there is a
non-adaptive adversary~$D$ that finds~$r$ false positives against
$\tilde{B}_{\mathrm{lin}}$, which is $B_{\mathrm{lin}}$ but
with $F_{K_1}$ and $F_{K_2}$ replaced by random functions $\rho_1$ and $\rho_2$,
respectively.  (Note, this is \emph{not} the adversary~$D$ that you
give, because the one you give simulates things that it would be given
in the FP game, e.g. the set~$S$.)  That is, $\Prob{\game{4}(A)=1} =
\AdvFPSecHash{\tilde{B}_{\mathrm{lin}},r}{D}$.  (So I guess there's no
need for $\game{5}$.)  Now argue that this advantage is upperbounded
	by what the theorem claims, i.e. the KM bound raised to~$r$.}

In $\game{3}(A)$, the hash functions are linear combinations of two random points
in $[m]$, similar to $\game{2}(A)$. 

So, both have equal probability of returning 1.  \tsnote{``The value
  of~$\bad$ does not affect the value returned by~$\game{3}$. Thus we
  conclude that...'' }
\begin{equation}
\Prob{\game{2}(A)=1} = \Prob{\game{3}(A)=1}\label{eq:3}
\end{equation}
 

The probability of winning  $\game{3}(A)$ is same as $\game{4}(A)$,
except $\game{4}(A)$ outputs 0 if $\bad$ flag is
set.\tsnote{I don't understand this sentence.  What you mean to say is
  something like this: ``By construction, $\game{3}$ and $\game{4}$
  are identical-until-bad.  In addition, $\game{4}$ returning 1
  implies that $\bad$ is never set to~$\true$.  From these two
  observations, we can conclude that $\Prob{\game{4}(A)=1} =
  \Prob{\game{3}(A)=1 \wedge \neg\bad}$.'' }
In short, $\game{3}(A)$ and $\game{4}(A)$ are identical untill $\bad
=true$.  On the  other hand, $\game{5}(A)$ simulates the non-adaptive
counterpart of $\game{4}(A)$.  \tsnote{This statement is unjustified.
(See my comments below.) Actually it's meaningless, because games are
not not adaptive or non-adaptive.  Adversaries are.}

\tsnote{You need to just justify these next lines better. For example,
why is the event $\neg \bad$ independent of the event $\game{3}(A)=1$?}
\begin{align}
\nonumber \Prob{\game{4}(A)=1} &= \Prob{\game{3}(A)=1 \wedge \neg\bad}\\ 
\nonumber & = \Prob{\game{3}(A)=1}\Prob{\game{3}(A):\,\neg\bad}\\
& = \Prob{\game{3}(A)=1}\frac{1}{\dbinom{q}{r}}\label{eq:3a}\\
\Prob{\game{5}(A)=1} &= \Prob{\game{4}(A)=1}\label{eq:3b}
\end{align}

\tsnote{Why is $B_2$
  sampling a random function?  This ruins the tightness of security
  bound w.r.t.\ resources.  Please have~$B_2$ carry out lazy sampling
  of~$\rho_2$.  Also, please use the macro for Func.} 

\tsnote{This proof jumps around too much.  The games should be
  addressed in order, as much as possible.}  
The classical FP requirement considers only non-adaptive (and information-theoretic) adversaries, and~$D$ is one such adversary simulating $\game{5}(A)$ as closely as possible, whereas $D'$ mimics $D$ for $r=1$. $D'$ outputs a FP as in a classic Bloom filter and its FP-error is upperbounded by $[(1-e^{-kn/m})^k +O(1/n)]$ \cite{KirschMitzenmacher}.  %\acnote{changed}\tsnote{The classical FP requirement considers only non-adaptive (and information-theoretic) adversaries.  It suffices to say this.}
\noindent
From \Cref{fig:AGame3}, it is evident that \tsnote{Where is the
  set~$S$ in $D$ and $D'$?  Make sure to give it to~$A$.  Why is this
  extra set~$\mathcal{Y}$ needed?  What is the point of~$D'$? }
\begin{align}
\Prob{\game{5}(A)=1} &= \Prob{D \, \mbox{outputs $r$ FP} \,}\label{eq:4a}\\
\nonumber \Prob{D \, \mbox{outputs r FP} \,} &=  \Big[\Prob{D' \, \mbox{outputs FP} \,}\Big]^r\\
& \leq \Big[(1-e^{-kn/m})^k +O(1/n)\Big]^r \label{eq:4b}
\end{align}
\tsnote{How does the second line follow?  You're claiming independence
  somehow, but you haven't justified it.}
\noindent
From \Cref{eq:4a,eq:4b}
\begin{equation}
\Prob{\game{5}(A)=1} \leq [(1-e^{-kn/m})^k + O(1/n)]^r \label{eq:4}
\end{equation}

\noindent
Using \ref{eq:4} in \ref{eq:3c}:
\begin{align}
\nonumber \Prob{\game{2}(A)=1}\frac{1}{\dbinom{q}{r}}  &\leq [(1-e^{-kn/m})^k +O(1/n)]^r\\
 \Prob{\game{2}(A)=1}  & \leq \dbinom{q}{r}[(1-e^{-kn/m})^k +O(1/n)]^r\label{eq:5}
\end{align}

\fixme{These adversaries are incorrect.  They are FP adversaries, so
  they get~$S$ as input.  They also do not sample random functions;
  use lazy sampling (a la game $\game{3}(A)$.  Please clean things up.)}
\noindent

\fi
\end{proof}

