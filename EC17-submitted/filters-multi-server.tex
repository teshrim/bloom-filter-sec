\section{Security in the Multi-Server Settings }  \jnote{What is the
goal here? If they don't share hash functions, I'm not sure what is
gained by considering this definition.} \tsnote{Perhaps not the same
hash functions, but correlated ones.  For example, server~$j$ uses
hash functions $\tilde{h}_i(x) = h_i(\langle \mathrm{id}_j \rangle \concat x)$ for
some $h_i$ shared among servers.  The $\mathrm{id}_j$ may or may not
be public, e.g. in a P2P network your ID might be a nonce learned only
when you join.  Also, the sets $\mathcal{S_j}$ stored at each server
might be correlated.} In this section, we consider
security notions for settings in which there may be multiple
servers, each holding a filter representation of a set.  To allow
for each server to embed server-specific information into its
representation, we abstract the servers as $B_1,B_2,\ldots,B_\ell$
where each $B_i=(\Hash_i,\Rep_i,\Qry_i)$ is an $(n,k,m)$-filter.
\tsnote{Still some kinks to be ironed out though.  The current
notions sample fresh hash functions for each server.  But in (say)
the Squid setting, the same hash functions are used across all Squid
instances... maybe it suffices, for that setting, to say that the
$\Hash_i$ are all the same, and each simply returns a fixed set
of~$k$ hash functions (i.e., deterministic).} To model insider
attacks, we give the adversary the ability to corrupt (adaptively)
instances of the filter.  Such attacks capture P2P settings, in
which the adversary may join the network as a peer.

\begin{figure}
\centering
\fpage{.6}{
\hpagess{.6}{.35}
{
\experimentv{$\ExpFPSecHash{B_1,B_2,\ldots,B_\ell}{\distr{U}{n}, A}$}\\
$\mathcal{C} \gets \emptyset$\\
for $i=1$ to $\ell$\\
\nudge $S_i \getsr \distr{U}{n}$\\
\nudge $ \mathcal{H}_i \gets \{h^i_1,h^i_2,\ldots,h^i_k\} \getsr \Hash_i$\\
\nudge $M_i \getsr \Rep_i^{\HashOracle(i,\cdot,\cdot)}(S_i)$\\
$(i,x) \getsr A^{\QryOracle,\CorruptOracle}(\{S_i\}_{i=1}^{\ell})$\\
if $i \not\in \mathcal{C} \wedge \Qry_i^{\HashOracle(i,\cdot,\cdot)}(M_i,x)=1$ then \\
\nudge Ret 1\\
Ret 0
}
%
{
\oracle{$\QryOracle(i,x)$}\\
Ret $\Qry_i^{\HashOracle}(M_i,x)$\\

\medskip
\oracle{$\HashOracle(i,j,x)$}\\
Ret $h^i_j(x)$\\

\medskip
\oracle{$\CorruptOracle(i,x)$}\\
$\mathcal{C} \gets \mathcal{C} \cup \{i\}$\\
Ret $M_i,\mathcal{H}_i$\\
}
}
%%%%%%%
\fpage{.6}{
\hpagess{.6}{.35}
{
\experimentv{$\ExpFPPubHash{B_1,B_2,\ldots,B_\ell}{\distr{U}{n}, A}$}\\
$\mathcal{C} \gets \emptyset$\\
for $i=1$ to $\ell$\\
\nudge $S_i \getsr \distr{U}{n}$\\
\nudge $ \mathcal{H}_i \gets \{h^i_1,h^i_2,\ldots,h^i_k\} \getsr \Hash_i$\\
\nudge $M_i \getsr \Rep_i^{\HashOracle(i,\cdot,\cdot)}(S_i)$\\
$(i,x) \getsr A^{\QryOracle,\CorruptOracle}(\{S_i\}_{i=1}^{\ell},\{\mathcal{H}_i\}_{i=1}^{\ell})$\\
if $i \not\in \mathcal{C} \wedge \Qry_i^{\HashOracle(i,\cdot,\cdot)}(M_i,x)=1$ then \\
\nudge Ret 1\\
Ret 0
}
%
{
\oracle{$\QryOracle(i,x)$}\\
Ret $\Qry_i^{\HashOracle}(M_i,x)$\\

\medskip
\oracle{$\HashOracle(i,j,x)$}\\
Ret $h^i_j(x)$\\

\medskip
\oracle{$\CorruptOracle(i,x)$}\\
$\mathcal{C} \gets \mathcal{C} \cup \{i\}$\\
Ret $M_i$\\
}
}
\caption{False-positive security for an $(n,k,m)$-filter~$B$ in the multiple-server setting.  (Three variations, depending on access to/knowledge of hash functions.) \textcolor{cyan}{Not clear which of these notions really makes sense.}}
\label{fig:fp-filter}
\end{figure}
