The Naor-Yogev paper~\cite{naor2015bloom} served as inspiration for our work.  We recommend reading the very good related work section contained in their paper for a discussion of related theory papers.  Here we mention a few additional practical works, but stress that this only scratches the surface.

As previously mentioned Bloom filters and their relatives are some of the most widely used data structures supporting set-membership queries.  We give a few more, here. Hbase, a hadoop based NoSql database designed to handle large datasets has an implementation of Bloom filter and counting Bloom filter. Hbase is the open-source implementation of Google's BigTable storage system~\cite{chang2008bigtable}. 
The Squid proxy~\cite{fan2000summary} uses a bloom filter as a ``summary cache'' of the set of URLs in its cache in order to web object retreival latency.   Crossby and Walach~\cite{crosby2003denial} exploited hash collisions in Squid to increase the average URL load time.  In an older paper, Lipton and Naughton~\cite{lipton1993clocked} showed how timing analysis of record insertion in a hash table can help an adversary to adapt and chose elements that take more time to be mapped into the hash table. Reynolds and Vahdat~\cite{reynolds2003efficient} proposed an efficient distributed search engine that can be used to search for files that contain a particular keyword. This search engine maps the keywords of each file into a Bloom filter; a look-up of the keyword in the Bloom filter tells whether the node has files containing that keyword or not. Record linkage is very often required to collect and combine records from different databases.

Schnell et. al.~\cite{schnell2011novel} proposed the so-called ``Cryptographic Longterm Key'' as a way to support privacy-preserving record linkage. This uses secretly keyed hash functions to create a Bloom filter for each surname. The surname is broken into bigrams and mapped into the Bloom filter with secret hash functions.  Subsequently, Niedermeyer et al.~\cite{niedermeyer2014cryptanalysis} showed an attack on a database of German surnames which were stored in secretly hashed Bloom filters, similar to Cryptographic Longterm Key.

Stochastic Fair Blue~\cite{feng2001stochastic} uses counting Bloom filter to manage non-responsive TCP traffic. 

