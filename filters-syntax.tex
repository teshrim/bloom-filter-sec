\section{Hash-based Filters}
\heading{Preliminaries.}
When~$\univ$ is a set and $n>0$ is an integer, we let $[\univ]^n$ denote the set of all size-$n$ subsets of~$\univ$. We write $x \getsr \univ$ to denote sampling an element from~$\univ$ and assigning this to~$x$, and we extend this notation to randomized algorithms.

\heading{Syntax.}

Fix nonempty sets $U,\Sigma$ and integers $k,m,n>0$.  Fix a symbol $\bot \not\in U$.
An $(n,k,m)$-filter (over universe~$U$) is a tuple $B=(\Hash,\Rep,\Qry)$.
%
The randomized \emph{hash-sampling} algorithm~$\Hash$ samples a
size~$k$ family of functions~$\mathcal{H}=\{h_1,h_2,\ldots,h_k\}$
where each $h_i \in \mathrm{Func}(U,[m])$. \jnote{This doesn't seem
to work super-well with the random oracle model.} \tsnote{Why not?
As we discussed on the phone, this should be fine.  In the ROM, if
we ever need to simulate $\Hash$, we just do it by lazy sampling.}
\jnote{True, it just means that $\Hash$ is not an efficient
algorithm in that case.} \tsnote{Agreed.  Just not sure that matters.}
We write $\mathcal{H} \getsr \Hash$ for this operation.
%
The randomized representation algorithm $\Rep\colon [\univ]^n \rightarrow \Sigma^m \times \bits^*$ takes a set~$S$ of~$n$ elements as input, and outputs representation~$M$ of length~$m$ along with (possibly empty) side-information~$\tau$.
%
The deterministic query algorithm $\Qry\colon \Sigma^m \times \bits^* \times \univ \rightarrow \bits$ takes a representation $M$, side-information~$\tau$, and an element $x \in \univ$, and returns a bit.
%
We assume that both $\Rep$ and $\Qry$ have blackbox access to the functions $h_1,h_2,\ldots,h_k \in \mathcal{H}$.  Typically this access will be implicit, but when we need to make it explicit we will write~$\mathcal{H}$ (or the individual hash functions) as superscripts.
%which we denote by writing~$\mathcal{H}$ as a superscript.
%Thus, we write $M \gets \Rep^{\mathcal{H}}(S)$ and $b \gets \Qry^{\mathcal{H}}(M,x)$ for the execution of the representation and query algorithms, respectively.
%
%We write $M \gets \Rep(S)$ and $b \gets \Qry(M,x)$ for the execution of the representation and query algorithms, respectively.
For correctness, we demand that for all~$S$ of size~$n$ and all $x \in S$, it holds that $\Prob{\mathcal{H}\getsr\Hash;\,(M,\tau)\getsr\Rep(S)\,:\,\Qry(M,\tau,x)=1}=1$, where the probability is over the coins of $\Hash$ and~$\Rep$.


%\heading{Discussion.}
This syntax captures classical Bloom filters by setting~$\Sigma=\bits$, and defining $\Rep,\Qry$ in the obvious way, with $\tau=\emptystring$. We note that even though $\Rep$ may be randomized, it does not have to be; for the classical Bloom filter, it is deterministic.

By allowing $\Rep$ to output side-information, we allow for (say) $\Rep$ and $\Qry$ to share a secret key selected during representation generation.  In our security notions, any side-information will be kept secret from the adversary.
\tsnote{Ultimately it might be cleaner to fold the hash functions into the side-information, too.  But in that case, we will likely want an explicit separation between private side-information (e.g., the key for a secret permutation of the bits of~$M$) and public side-information (e.g., the hash functions used to create~$M$).  Note that we do see constructions that have \emph{no} ``stand-alone'' hash functions yet use a secret key (e.g., the PRF-based double-hashing construction), and also schemes that employ \emph{both} stand-alone hash functions and a secret key (e.g. the record BF from Durham et al.)}
%
\tsnote{Per our phone call 7/8, we're going to leave things as they are for now.  Jon will create an appendix with more general syntax and notions; we'll see how results play out, and decide later if we want to adopt the change in syntax.}
Setting $\Sigma=\bits^{\leq L}$ allows one to store a bitstring of length at most~$L$ at each location.  This allows our syntax to capture the cuckoo-hashing construction from NY, as well as the ``garbled Bloom filter'' from Dong, Chen and Wen~\cite{xxx}, among other things. \fixme{If we change to generalized syntax, let's make sure we still capture all of these constructions.}We note that even though this would allow for the storage of counters (say) at each filter location, our restriction that~$\Qry$ returns a bit does not capture functionality typically demanded of Bloom filters that employ counters, e.g., returning a counter value. (One could, however, have $\Qry$ compute an arbitrary, fixed predicate.  For example, ``has~$x$ been seen at least 5 times?''.) Our syntax also does not capture filters that are mutable, that is, whose representation may change over time. On the other hand, we can state a simple correctness condition for these kinds of filters; this will not be true when we further generalize to capture, properly, counters and mutability.
