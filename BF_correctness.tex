%\section{Proof of Theorem~\ref{thm:bf-correctness}}
%\label{sec:thm3_fp}
\begin{proof}[Theorem~\ref{thm:bf-correctness}]
We use a game playing argument, please refer to Figures~\ref{fig:bf-correctness-games1} and~\ref{fig:bf-correctness-games2}. In game $\game{0}$, the $\HashOracle$ oracle performs lazy sampling of elements from $[m]$ to mimic a random oracle (RO). For every distinct query, $\HashOracle$ returns a random value sampled from $[m]$. It can be seen that, game $\game{0}$ exactly simulates the correctness game of $\setprim_{\mathrm{gbf}}$.  So, $\AdvCorrect{\setprim_{\mathrm{gbf}},r}{A} = \Prob{\game{0}(A) = 1}$.

Game $\game{1}$ is exactly same as game $\game{0}$ except the $\Qry$ call (inside the $\TestOracle$ oracle) is replaced by the code of $\Qry$.  Hence, $\Prob{\game{0}(A) = 1} = \Prob{\game{1}(A) = 1}$.

In game $\game{2}$, the $\Test$ oracle is altered to return the hash values~$h_1,h_2,\ldots, h_k$ in addition to the result of executing the code of $\Qry$.  As this only provides additional information to the adversary, $\Prob{\game{1}(A) = 1} \leq \Prob{\game{2}(A) = 1}$.

In $\game{3}$, we modify the $\HashOracle$ oracle so that on query~$\qry_x$ it returns all of the~$k$ values that previously would have required calls $\HashOracle(1,\qry_y),\HashOracle(2,\qry_y),\ldots,\linebreak\HashOracle(k,\qry_y)$.  This results in adversary getting $(k-1)$ hash values ``for free'', as we charge the adversary only for a single query.  We alter the $\TestOracle$ oracle accordingly, so that it makes a single call $\Hash(\qry_x)$ rather than~$k$ such calls.  We do this so that we can assume that a call~$\HashOracle(\qry_x)$ is pointless after a~$\TestOracle(\qry_x)$ call, and this will aid us in subsequent games.  \fixme{for BF proof, this actually does more, it makes a hash query as good as a test query, because~$M$ is public.} Thus, $\Prob{\game{2}(A) = 1} \leq \Prob{\game{3}(A) = 1}$.  We assume from Game~$\game{3}$ and on that $\Rep$ is also appropriately modifed for the new behavior of~$\HashOracle$.

Game~$\game{4}$ contains some important changes from game~$\game{3}$,
but we will argue that $\Prob{\game{3}(A) = 1} \leq \Prob{\game{4}(A)
  = 1}$. The $\TestOracle$ oracle on input~$\qry_x$ no longer computes
its answer directly, but instead refers to a new table~$\mathrm{Ans}[x]$ which is populated upon calls to the
$\HashOracle$ oracle.  In particular, if $\mathrm{Ans}[x]$ is undefined, $\TestOracle$ makes a $\HashOracle(\qry_x)$ call;
otherwise, it returns the contents of~$\mathrm{Ans}[x]$, which
contains exactly the values that~$\TestOracle(\qry_x)$ would have
returned in the previous game.  The $\HashOracle$ oracle, on input~$\qry_y$ computes the random values~$v_1,v_2,\ldots,v_k$ as in
$\game{3}$.  It maintains the table~$T$ because~$A$ is permitted to call~$\HashOracle$
%(We no longer need to maintain the table~$T$ because hash values are
%only determined by~$\HashOracle$ calls, and these are never repeated
%under our no-pointless-query assumption.)  
It then computes the answer that~$\TestOracle(\qry_x)$ would return in
game~$\game{3}$, and stores the answer in~$\mathrm{Ans}[x]$ for use
by~$\TestOracle$.  
The $\HashOracle$ oracle also introduces a query counter~$c$, which is
incremented whenever~$\HashOracle$ is called, either directly by~$A$
or via~$A$ having made a~$\TestOracle$ query on~$\qry_x$
where~$\mathrm{Ans}[x]$ had not been previously defined by
a~$\HashOracle(\qry_x)$ call. 
% \fixme{for BF proof, we can assume that $\TestOracle(\qry_x)$ is
% never made after a~$\HashOracle(\qry_x)$ call, because~$M$ is
% public.  And vice versa.  So the maximum value of~$c = q_T + q_H$.}
% (Note that the maximum value of~$c$ is $q_T + q_H$.)  
None of these changes alters the behavior of the game from the
adversary's perspective.

At this point, we consider $\Prob{\game{4}(A)=1}$.  Note that is
happens iff there are~$r$ calls to the $\HashOracle$ oracle that
increment~$\err$, i.e., queries $y \not\in S$ such that
$\bigwedge_{j=1}^{k} M[v_j]=1$.  We can assume without loss that~$A$
does not call~$\HashOracle$ on $y \in S$ because the indices
$v_1,v_2,\ldots,v_k$ that will correspond to any $y \not\in S$ will be
\emph{independent} of those corrsponding to any of the $y \in S$.
Thus these queries are wasted.  Now, there are at total of $q_T + q_H$
queries make to~$\HashOracle$, either directly by~$A$ or indirectly
by~$A$ calling~$\TestOracle$.  For $\game{4}(A)$ to return 1, there
must be~$r$ among these that increment the counter.  We can assume
WLOG that~$A$ halts as soon as~$\err=r$ (which it can easily track),
so there are $\binom{q_T+q_H-1}{r-1}$ possible ``winning'' query
patterns to consider.  As each error event is independent of all
previous choices made by the~$\HashOracle$, we need only consider the
probability of a single error. Using Kirsch and Mitzenmacher's upper
bound on false positive probability of a classical Bloom filter, 
we have  $\Prob{\game{4}(A)=1} = \binom{q_T+q_H-1}{r-1}\left( (1-e^{-kn/m})^k + O(1/n) \right)^r$.

This implies that~$\game{4}$ managed to guess exactly the queries that would result in correctness errors, and hence
$\Prob{\game{5}(A)=1} = \dfrac{1}{\binom{q_T + q_H}{r}}\Prob{\game{4}(A)=1}$.
To summarize, 
\[
\AdvCorrect{\setprim_{\mathrm{Bloom}},r}{A} \leq  {\dbinom{q_T + q_H-1}{r-1}} \left( (1-e^{-kn/m})^k + O(1/n) \right)^r\,.
\]
which is the bound claimed in the theorem statement.
\begin{figure}[tp]
\fpage{.9}{
\hpagessl{.45}{.5}
{
\underline{$\game{0}(A)$}\\
$S \getsr A$; $\calC \gets \emptyset$; $\err \gets 0$\\
$(\pubaux,\privaux) \getsr \Rep^{\HashOracle}(S)$\\
$\bot \getsr A^{\TestOracle,\HashOracle}(\pubaux)$\\
if $\err  < r$ then Return 0\\
Return 1\\

\oracle{$\TestOracle(\qry_x)$}\\
if $x \in \mathcal{C}$ then Return $\bot$\\
$\calC \gets \calC \cup x$\\
$a \gets \Qry^{\HashOracle}(\pubaux,\privaux, x)$\\
if $a \neq \qry_x(S)$ then \\
\nudge $\err \gets \err + 1$\\
Return~$a$\\

\oracle{$\HashOracle(i,\qry_y)$}\\
$v \getsr [m]$\\
if $T[i,y] \neq \undefined$\\
\nudge $v \gets T[i,y]$\\
$T[i,y] \gets v$\\
Return $v$
}
{
\underline{$\game{1}(A)$}\\
$S \getsr A$; $\calC \gets \emptyset$; $\err \gets 0$\\
$(\pubaux,\privaux) \getsr \Rep^{\HashOracle}(S)$\\
$\bot \getsr A^{\TestOracle,\HashOracle}(\pubaux)$\\
if $\err  < r$ then Return 0\\
Return 1\\

\oracle{$\TestOracle(\qry_x)$}\\
if $x \in \mathcal{C}$ then Return $\bot$\\
$\calC \gets \calC \cup x$; $M \gets\pubaux$\\
for $j \in \{1,2,\ldots,k\}$\\
\nudge $h_j \gets \HashOracle(j, \qry_x)$\\
\nudge if $M[h_j] \neq 1$ then Return 0\\
$\err \gets \err + 1$\\
Return 1\\

\oracle{$\HashOracle(i,\qry_y)$}\\
$v \getsr [m]$\\
if $T[i,y] \neq \undefined$\\
\nudge $v \gets T[i,y]$\\
$T[i,y] \gets v$\\
Return $v$
}
}
\fpage{.9}{
\hpagessl{.45}{.5}
{
\underline{$\game{2}(A)$}\\
$S \getsr A$; $\calC \gets \emptyset$; $\err \gets 0$\\
$(\pubaux,\privaux) \getsr \Rep^{\HashOracle}(S)$\\
$\bot \getsr A^{\TestOracle,\HashOracle}(\pubaux)$\\
if $\err  < r$ then Return 0\\
Return 1\\ 

\oracle{$\TestOracle(\qry_x)$}\\
if $x \in \mathcal{C}$ then Return $\bot$\\
$\calC \gets \calC \cup x$; $M \gets\pubaux$\\
for $j \in \{1,2,\ldots,k\}$\\
\nudge $h_j \gets \HashOracle(j, \qry_x)$\\
\nudge if $M[h_j] \neq 1$\\
\nudge \nudge Return $(0, (h_1, \ldots, h_j))$\\
$\err \gets \err +1$\\
Return~$(1, (h_1, \ldots, h_{k}) )$\\

\oracle{$\HashOracle(i,\qry_y)$}\\
$v \getsr [m]$\\
if $T[i,y] \neq \undefined$\\
\nudge $v \gets T[i,y]$\\
$T[i,y] \gets v$\\
Return $v$
}
{
\underline{$\game{3}(A)$}\\
$S \getsr A$; $\calC \gets \emptyset$; $\err \gets 0$\\
$(\pubaux,\privaux) \getsr \Rep^{\HashOracle}(S)$\\
$\bot \getsr A^{\TestOracle,\HashOracle}(\pubaux)$\\
if $\err  < r$ then Return 0\\
Return 1\\ 

\oracle{$\TestOracle(\qry_x)$}\\
if $x \in \mathcal{C}$ then Return $\bot$\\
$\calC \gets \calC \cup x$; $M \gets\pubaux$\\
$(h_1, \ldots, h_{k}) \gets \HashOracle(\qry_x)$\\
for $j \in \{1,2,\ldots,k\}$\\
\nudge if $M[h_j] \neq 1$ then \\
\nudge \nudge Return~$(0, (h_1, \ldots, h_{k}) )$\\
$\err \gets \err +1$\\
Return~$(1, (h_1, \ldots, h_{k}) )$\\

\oracle{$\HashOracle(\qry_y)$}\\
for $j \in \{1,2,\ldots,k\}$\\
\nudge $v_j \getsr [m]$\\
\nudge if $T[j,y] \neq \undefined$\\
\nudge \nudge $v_j \gets T[j,y]$\\
\nudge $T[j,y] \gets v_j$\\
Return $\left(v_1,\ldots,v_k\right)$
}
}
\caption{Games $\game{0}$-$\game{3}$ for the
proof of Theorem~\ref{thm:bf-correctness}.}
\label{fig:bf-correctness-games1}
\end{figure}
\begin{figure}[tp]
\fpage{.9}{
\hpagessl{.45}{.5}
{
\underline{$\game{4}(A)$} \\ %\fbox{$\game{5}(A)$}}\\
$S \getsr A$; $\err \gets 0$;  $\calC \gets \emptyset$\\
%$\mathcal{I}\getsr [\{1,2,\ldots,q_T+q_H\}]^r$\\
$(\pubaux,\privaux) \getsr \Rep^{\HashOracle}(S)$\\
$\bot \getsr A^{\TestOracle,\HashOracle}(\pubaux)$\\
%if $\bad = \true$ then \fbox{Return 0}\\
if $\err  < r$ then Return 0\\
Return 1\\
%
\oracle{$\TestOracle(\qry_x)$}\\
if $x \in \mathcal{C}$ then Return $\bot$\\
$\calC \gets \calC \cup x$\\
if $\Ans[x] = \undefined$\\
\nudge $(h_1, \ldots, h_{k}) \gets \HashOracle(\qry_x)$\\
Return $\Ans[x]$\\
%
}
{
\oracle{$\HashOracle(\qry_y)$}\\
for $j \in \{1,2,\ldots,k\}$\\
\nudge $v_j \getsr [m]$\\
\nudge if $T[j,y] \neq \undefined$\\
\nudge \nudge $v_j \gets T[j,y]$\\
\nudge $T[j,y] \gets v_j$\\
%\nudge $c \gets c + 1$\\
%\nudge for $j \in \{1,2,\ldots,k\}$\\
%\nudge $v_j \getsr [m]$\\
%
$M \gets\pubaux$\\
for $j \in \{1,2,\ldots,k\}$\\
\nudge if $M[v_j] \neq 1$ then \\
\nudge \nudge $\Ans[y] \gets (0, (v_1, \ldots, v_{k}) )$\\
%\nudge \nudge if $c \in \calI$ then $\bad \gets \true$\\
\nudge \nudge $\Break$\\
if $j < k$ then Return $\left(v_1,\ldots,v_k\right)$\\
%if $c \notin \calI$ then $\bad \gets \true$\\
if $y \not\in S$ then $\err \gets \err +1$\\
$\Ans[y] \gets (1, (v_1, \ldots, v_{k}))$\\
Return $\left(v_1,\ldots,v_k\right)$
}
}
\caption{Games $\game{4}$-$\game{5}$ for the
proof of Theorem~\ref{thm:bf-correctness}.}
\label{fig:bf-correctness-games2}
\end{figure}
\hfill\qed
\end{proof}
