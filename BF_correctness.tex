%\section{Proof of Theorem~\ref{thm:bf-correctness}}
%\label{sec:thm3_fp}
\begin{proof}[Theorem~\ref{thm:bf-correctness}]
We use a game playing argument; please refer to
Figures~\ref{fig:bf-correctness-games1}
and~\ref{fig:bf-correctness-games2}. In game $\game{0}$, the
$\HashOracle$ oracle performs lazy sampling of elements from $[m]$
to mimic a random oracle. For every distinct query, $\HashOracle$
returns a random value sampled from $[m]$. It can be seen that game
$\game{0}$ exactly simulates the correctness game of
$\setprim_{\mathrm{Bloom}}$, and so
$\AdvCorrect{\setprim_{\mathrm{Bloom}},r}{A} = \Prob{\game{0}(A) =
1}$.

\begin{figure}[tp]
\fpage{.9}{
\hpagessl{.45}{.5}
{
\underline{$\game{0}(A)$}\\
$S \getsr A$; $\calC \gets \emptyset$; $\err \gets 0$\\
$(\pubaux,\privaux) \getsr \Rep^{\HashOracle}(S)$\\
$\bot \getsr A^{\TestOracle,\HashOracle}(\pubaux)$\\
if $\err  < r$ then Return 0\\
Return 1\\

\oracle{$\TestOracle(q_x)$}\\
if $x \in \mathcal{C}$ then Return $\bot$\\
$\calC \gets \calC \cup x$\\
$a \gets \Qry^{\HashOracle}(\pubaux,\privaux, x)$\\
if $a \neq \qry_x(S)$ then \\
\nudge $\err \gets \err + 1$\\
Return~$a$\\

\oracle{$\HashOracle(i,y)$}\\
$v \getsr [m]$\\
if $T[i,y] \neq \undefined$\\
\nudge $v \gets T[i,y]$\\
$T[i,y] \gets v$\\
Return $v$ } {
\underline{$\game{1}(A)$}\\
$S \getsr A$; $\calC \gets \emptyset$; $\err \gets 0$\\
$(\pubaux,\privaux) \getsr \Rep^{\HashOracle}(S)$\\
$\bot \getsr A^{\TestOracle,\HashOracle}(\pubaux)$\\
if $\err  < r$ then Return 0\\
Return 1\\

\oracle{$\TestOracle(q_x)$}\\
if $x \in \mathcal{C}$ then Return $\bot$\\
$\calC \gets \calC \cup x$; $M \gets\pubaux$\\
for $j \in \{1,2,\ldots,k\}$\\
\nudge $h_j \gets \HashOracle(j, x)$\\
\nudge if $M[h_j] \neq 1$ then Return 0\\
if $x \not \in S$ then $\err \gets \err + 1$\\
Return 1\\

\oracle{$\HashOracle(i,y)$}\\
$v \getsr [m]$\\
if $T[i,y] \neq \undefined$\\
\nudge $v \gets T[i,y]$\\
$T[i,y] \gets v$\\
Return $v$ } } \fpage{.9}{ \hpagessl{.45}{.5} {
\underline{$\game{2}(A)$}\\
$S \getsr A$; $\calC \gets \emptyset$; $\err \gets 0$\\
$(\pubaux,\privaux) \getsr \Rep^{\HashOracle}(S)$\\
$\bot \getsr A^{\TestOracle,\HashOracle}(\pubaux)$\\
if $\err  < r$ then Return 0\\
Return 1\\

\oracle{$\TestOracle(q_x)$}\\
if $x \in \mathcal{C}$ then Return $\bot$\\
$\calC \gets \calC \cup x$; $M \gets\pubaux$\\
for $j \in \{1,2,\ldots,k\}$\\
\nudge $h_j \gets \HashOracle(j, x)$\\
\nudge if $M[h_j] \neq 1$\\
\nudge \nudge Return $(0, (h_1, \ldots, h_j))$\\
if $x \not \in S$ then $\err \gets \err + 1$\\
Return~$(1, (h_1, \ldots, h_{k}) )$\\

\oracle{$\HashOracle(i, y)$}\\
$v \getsr [m]$\\
if $T[i,y] \neq \undefined$\\
\nudge $v \gets T[i,y]$\\
$T[i,y] \gets v$\\
Return $v$ } {
\underline{$\game{3}(A)$}\\
$S \getsr A$; $\calC \gets \emptyset$; $\err \gets 0$\\
$(\pubaux,\privaux) \getsr \Rep^{\HashOracle}(S)$\\
$\bot \getsr A^{\TestOracle,\HashOracle}(\pubaux)$\\
if $\err  < r$ then Return 0\\
Return 1\\

\oracle{$\TestOracle(q_x)$}\\
if $x \in \mathcal{C}$ then Return $\bot$\\
$\calC \gets \calC \cup x$; $M \gets\pubaux$\\
$(h_1, \ldots, h_{k}) \gets \HashOracle(x)$\\
for $j \in \{1,2,\ldots,k\}$\\
\nudge if $M[h_j] \neq 1$ then \\
\nudge \nudge Return~$(0, (h_1, \ldots, h_{k}) )$\\
if $x \not \in S$ then $\err \gets \err +1$\\
Return~$(1, (h_1, \ldots, h_{k}) )$\\

\oracle{$\HashOracle(y)$}\\
for $j \in \{1,2,\ldots,k\}$\\
\nudge $v_j \getsr [m]$\\
\nudge if $T[j,y] \neq \undefined$\\
\nudge \nudge $v_j \gets T[j,y]$\\
\nudge $T[j,y] \gets v_j$\\
Return $\left(v_1,\ldots,v_k\right)$ } } \caption{Games
$\game{0}$--$\game{3}$ for the proof of
Theorem~\ref{thm:bf-correctness}.} \label{fig:bf-correctness-games1}
\end{figure}


Game $\game{1}$ is exactly same as game $\game{0}$ except the $\Qry$
call (inside the $\TestOracle$ oracle) is replaced by the code of
$\Qry$.  Hence, $\Prob{\game{0}(A) = 1} = \Prob{\game{1}(A) = 1}$.

In game $\game{2}$, the $\Test$ oracle is altered to return hash
values~$h_1,\ldots$ in addition to the result of executing the code
of $\Qry$.  As this only provides additional information to the
adversary, $\Prob{\game{1}(A) = 1} \leq \Prob{\game{2}(A) = 1}$.
\jnote{Actually this is not true\ldots what you want to say is that
there is another adversary $A'$ making the same number of queries
and with essentially the same running time such that
$\Prob{\game{1}(A) = 1} \leq \Prob{\game{2}(A') = 1}$. Same comment
applies to G3 below.}

\jnote{This is a rather long-winded proof considering that nothing
substantial changes in games G0--G3.}

In $\game{3}$, we modify the $\HashOracle$ oracle so that on
query~$x$ it returns the~$k$ values that previously would have
required $k$ calls $\HashOracle(1,x), \ldots, \HashOracle(k,x)$. We
alter the $\TestOracle$ oracle accordingly, so that it makes a
single call to $\Hash$ rather than~$k$ such calls. (We assume that
$\Rep$ is also appropriately modified for the new behavior
of~$\HashOracle$.) We do this so we can assume that a
call~$\HashOracle(x)$ is pointless after a~$\TestOracle(x)$ call has
been made. This change results in the adversary getting $(k-1)$ hash
values ``for free,'' as we charge the adversary only for a single
query. \jnote{Can we do better by being more careful?} It is obvious
that $\Prob{\game{2}(A) = 1} \leq \Prob{\game{3}(A) = 1}$.

Game~$\game{4}$ contains some important changes from
game~$\game{3}$. In $\game{4}$ the $\TestOracle$ oracle no longer
computes its answer directly, but instead refers to a
table~$\mathrm{Ans}[x]$ that is populated upon calls to the
$\HashOracle$ oracle.  If $\mathrm{Ans}[x]$ is undefined,
$\TestOracle$ makes a call $\HashOracle(x)$; otherwise, it returns
the contents of~$\mathrm{Ans}[x]$, which contains exactly the values
that~$\TestOracle(q_x)$ would have returned in the previous game.
The $\HashOracle$ oracle, on input~$x$, computes
values~$v_1,\ldots,v_k$ as in $\game{3}$. %It maintains the table~$T$
%because~$A$ is permitted to call~$\HashOracle$ on $y \in S$.
%(We no longer need to maintain the table~$T$ because hash values are
%only determined by~$\HashOracle$ calls, and these are never repeated
%under our no-pointless-query assumption.)
It then computes the answer that~$\TestOracle(q_x)$ would return in
game~$\game{3}$, and stores the answer in~$\mathrm{Ans}[x]$ for use
by~$\TestOracle$. None of these changes alters the view of~$A$, and
so $\Prob{\game{3}(A) = 1} = \Prob{\game{4}(A)
  = 1}$. \jnote{You had $\leq$ before.}

Note that $\game{4}(A)=1$ iff there are~$r$ queries to the
$\HashOracle$ oracle that increment~$\err$, i.e., queries $y \not\in
S$ such that $\bigwedge_{j=1}^{k} M[v_j]=1$.
%We can assume without
%loss of generality that~$A$ does not call~$\HashOracle$ on $y \in S$
%because the indices $v_1,v_2,\ldots,v_k$ that will correspond to any
%$y \not\in S$ will be \emph{independent} of those corresponding to
%any of the $y \in S$. Thus these queries are wasted.
There are at total of $q_T + q_H$ queries made to~$\HashOracle$,
either directly by~$A$ or indirectly when~$A$ calls~$\TestOracle$.
For $\game{4}(A)$ to return~1, there must be~$r$ among these that
increment~$\err$. %We can assume without loss of generality that~$A$
%halts as soon as~$\err=r$ (which it can easily track), so
There are $\binom{q_T+q_H}{r}$ possible ``winning'' query patterns
to consider.  \jnote{I changed the bound; I don't think it was
correct before.} As each error event is independent of all previous
choices made by the~$\HashOracle$, we need only consider the
probability of a single error. Using the upper bound on false
positives for the classical Bloom filter shown by Kirsch and
Mitzenmacher~\cite{kirsch2006less}, we have $\Prob{\game{4}(A)=1}
\leq \binom{q_T+q_H}{r}\left( (1-e^{-kn/m})^k + O(1/n) \right)^r$.
We conclude that
\[
\AdvCorrect{\setprim_{\mathrm{Bloom}},r}{A} \leq  {\dbinom{q_T + q_H}{r}} \left( (1-e^{-kn/m})^k + O(1/n) \right)^r\,,
\]
as claimed.
\begin{figure}[tp]
\fpage{.8}{
\hpagessl{.4}{.4}
{
\underline{$\game{4}(A)$} \\ %\fbox{$\game{5}(A)$}}\\
$S \getsr A$; $\err \gets 0$;  $\calC \gets \emptyset$\\
%$\mathcal{I}\getsr [\{1,2,\ldots,q_T+q_H\}]^r$\\
$(\pubaux,\privaux) \getsr \Rep^{\HashOracle}(S)$\\
$\bot \getsr A^{\TestOracle,\HashOracle}(\pubaux)$\\
%if $\bad = \true$ then \fbox{Return 0}\\
if $\err  < r$ then Return 0\\
Return 1\\

\oracle{$\TestOracle(q_x)$}\\
if $x \in \mathcal{C}$ then Return $\bot$\\
$\calC \gets \calC \cup x$\\
if $\Ans[x] = \undefined$\\
\nudge $(h_1, \ldots, h_{k}) \gets \HashOracle(x)$\\
Return $\Ans[x]$\\
%
}
{
\oracle{$\HashOracle(y)$}\\
for $j \in \{1,2,\ldots,k\}$\\
\nudge $v_j \getsr [m]$\\
\nudge if $T[j,y] \neq \undefined$\\
\nudge \nudge $v_j \gets T[j,y]$\\
\nudge $T[j,y] \gets v_j$\\
%\nudge $c \gets c + 1$\\
%\nudge for $j \in \{1,2,\ldots,k\}$\\
%\nudge $v_j \getsr [m]$\\
%
$M \gets\pubaux$\\
for $j \in \{1,2,\ldots,k\}$\\
\nudge if $M[v_j] \neq 1$ then \\
\nudge \nudge $\Ans[y] \gets (0, (v_1, \ldots, v_{k}) )$\\
%\nudge \nudge if $c \in \calI$ then $\bad \gets \true$\\
\nudge \nudge Return $\left(v_1,\ldots,v_k\right)$\\
%\nudge \nudge $\Break$\\
%if $j < k$ then Return $\left(v_1,\ldots,v_k\right)$\\
%if $c \notin \calI$ then $\bad \gets \true$\\
if $y \not\in S$ then $\err \gets \err +1$\\
$\Ans[y] \gets (1, (v_1, \ldots, v_{k}))$\\
Return $\left(v_1,\ldots,v_k\right)$ } } \caption{Game $\game{4}$
for the proof of Theorem~\ref{thm:bf-correctness}. \jnote{The
centering/spacing are not right.}} \label{fig:bf-correctness-games2}
\end{figure}
\hfill\qed
\end{proof}
