\newcommand{\qry}{\mathsf{qry}}
\section{Preliminaries}

When~$\univ$ is a set, we write $x \getsr \univ$ to denote sampling
an element uniformly from~$\univ$ and assigning it to~$x$. We extend
this notation to randomized algorithms, so $x \getsr A(y)$ denotes
that (randomized) algorithm $A$ is run on input~$y$, and the output
of the algorithm is assigned to~$x$. Algorithms may be provided
black-box access to one or more oracles, which we write as
superscripts, e.g., $F^{O_1,O_2,\ldots}$.  All algorithms (and, in
particular, all adversaries) are stateful by default. We write
$\langle x,y,\ldots \rangle$ for the encoding of values $x,y,\ldots$
as a single bitstring, for some implicit and invertible encoding
method. When $x$ is a string, we write $x[i]$ for the $i$-th element
in that string, and $x[i:j]$ for the substring $x[i]x[i+1]\cdots
x[j]$.

\subsection{Data Structures}

We consider an abstract notion of data structures.

%Informally, the representation should serve as a good
%approximation of the multiset, at least with respect to the allowed queries.
\begin{definition}[Data structure] \rm
Fix sets~$\univ, \Gamma$, and $\calQ = \{\qry \colon \univ \to
\Gamma\}$. A \emph{data structure} (supporting queries $\calQ$ over
universe $\univ$) is a tuple $\setprim=(\Rep, \Qry)$ where:
\begin{itemize}
\item $\Rep \colon \univ \to \bits^* \times \bits^*$ is a
    randomized \emph{representation algorithm}, taking as input
    an element $S \in \univ$,
    and outputting %a \emph{representation} $M \in \Sigma^*$,
    \emph{public representation data} $\pubaux \in \bits^*$ and
    \emph{private representation data} $\privaux \in \bits^*$.
    We write $(\pubaux,\privaux) \getsr \Rep(S)$ as shorthand
    for this execution.
\item $\Qry\colon \bits^* \times \bits^* \times \calQ \to
    \Gamma$ is a deterministic \emph{query-evaluation}
    algorithm, taking as input $\pubaux,\privaux \in \bits^*$
    and a query $\qry \in \calQ$, and outputting an
    \emph{answer} $a \in \Gamma$.  We write $a \gets
    \Qry(\pubaux,\privaux,\qry)$ for this execution. \hfill\dqed
\end{itemize}
\end{definition}

\todo{Insert discussion of syntax, if needed.  If we think it is
  helpful to discuss the size of the representation, we can declare it
as $|\pubaux|+|\privaux|$ and comment that if one wants to compare to
classical size bounds, a more fine-grained separation of the output of
$\Rep$ would allow this.}

\tsnote{Do we ever use this next definition?  I guess it is a mapping of
  classical notions to our syntax, but I don't know if we use it.
  Anyway,  it looks a bit weird sitting here on its own.  It made more
  sense when supported by the intuition-building example, now
  commented out. } \jnote{I don't think we do ever use it, but maybe
  we should? E.g., we can compare the (normal) error probability to
  the adversarial soundness probability.}

\begin{definition}[Error probability and rate] \rm
For a specific data structure $\setprim=(\Rep, \Qry)$ supporting
queries $\calQ$ over universe $\univ$, let
$\mathrm{Err}(S,\qry,\pubaux,\privaux)$ be  true iff
$\Qry(\pubaux,\privaux, \qry) \neq \qry(S)$.  Then we say that $\setprim$
has \emph{error probability}~$\epsilon$ if
\[\max_{S,\qry}\Pr[(\pubaux,\privaux) \getsr \Rep(S) :
\mathrm{Err}(S,\qry,\pubaux,\privaux)=1 ] \leq \epsilon.\]
%
We say it has \emph{error rate} $\tilde{\epsilon}$ if
\[{\textstyle \max_{S}\Pr[(\pubaux,\privaux) \getsr \Rep(S);\, \qry \getsr \calQ :
\mathrm{Err}(S,\qry,\pubaux,\privaux)=1]} \leq \tilde{\epsilon}. \]
\hfill\dqed
\end{definition}

%We will sometimes say that a data structure is
%$(1-\epsilon)$-correct, meaning it has error probability~$\epsilon$.
%Note that $1-\epsilon$ is a worst-case lower bound on the
%correctness of the data structure. \tsnote{Don't know if this will be necessary.}

The above definition can be lifted to the random-oracle model by
giving $\Rep, \Qry$ access to a random oracle. In that case, the
probabilities are also taken over choice of the random oracle. We
remark that, while our error probability and error rate do not
distinguish among ``types'' of errors, they also do not preclude
instantiations (like the classical Bloom filter) that admit only one
type of error, e.g., false positives.


\def\bin{{\sf Bin}}
\ignore{
\heading{Error probability vs.\ error rate. } We pause to
highlight an important difference between error probability and
error rate.  In particular, the error probability~$\epsilon$ is
measured only over the coins of~$\Rep$, for a fixed~$S$ and
query~$\qry$. Once the coins of~$\Rep$ are fixed, it is not
straightforward how to connect~$\epsilon$ to events that depend on
other sources of randomness.   We illustrate how this surfaces with
the following example.

\jnote{The following may need to be updated to fit the new syntax.}
Let~$\setprim$ be a data structure that has error-probability $\epsilon$,
and consider a correctness-violating attack that asks random queries
$\qry_1,\qry_2,\ldots,\qry_T$, i.e., uniform samples from~$\calQ$.
Let $\mathrm{Err}_i$ be a random variable indicating the event
$\Qry(\pubaux,\privaux,\qry_i)\neq \qry_i(S)$.  We know that the
probability that $\mathrm{Err}_i=1$ is at most~$\epsilon$, and it is
tempting to model the~$\mathrm{Err}_i$ as i.i.d. Bernoulli trials
with success probability~$\epsilon$. However, a moment's reflection
shows that this is not the case. Concretely, consider the following
scheme. Fix $|S|=n$ for all $S \in \calS$, where $n \ll |\univ|$.
For the queries, let $\calQ = \{\qry_x\colon \calS \to \bits \,|\,
\forall x \in \univ\}$ where $\forall S \in \calS$ the predicate
$\qry_x(S)=1 \Leftrightarrow x \in S$.  Let $\Rep(S)$ pick a random
element~$s \in S$, and return $M = S \setminus \{s\}$ and $\privaux
= \emptystring$.  To respond to queries,
let~$\Qry(\pubaux,\privaux,\qry_x)=1$ iff $x \in M$.  It is easy to
see that this scheme is $1/n$-correct; for any fixed~$S$, there are
exactly~$n$ queries~$\qry_x$ that have probability~$1/n$ (over the
coins of~$\Rep$) of causing a correctness error, and the remaining
$|\univ|-n$ queries have probability zero of causing an error.
Moreover, once the coins of~$\Rep$ are fixed, there is exactly
\emph{one} query that will cause an error.  Thus for random queries
from~$Q$, for all~$i\in[T]$ we have
$\Prob{\mathrm{Err}_i=1}=1/|\univ| \ll 1/n = \epsilon$.
}

%\heading{Data structures for set-membership queries.}
An important special case is a data structure that supports set-membership
queries. \jnote{I changed this from multiplicity to membership.}

\begin{definition} \rm
A \emph{set-membership data structure over $\elts$} is a data
structure with $\univ \subseteq 2^\elts$, $\Gamma=\{0,1\}$, and
$\calQ=\{\qry_x\}_{x \in \univ}$, where $\qry_x(S)$ is true iff $x
\in S$. \hfill\dqed
\end{definition}

Note that we do not require $\univ = 2^\elts$ since, e.g., a
set-membership data structure may only be defined for sets up to
some bounded size.

\ignore{ \jnote{Below needs to change.} For concreteness, we note
that a classical Bloom filter is a particular example of a
set-membership data structure. Specifically,
%\jnote{Only handles sets, not multisets.}
let $\Gamma=\bits$, and fix functions $k(\cdot)$ and $m(\cdot)$
parameterizing the scheme.  Algorithm $\Rep$, on input a set~$S$,
computes $k=k(|S|)$ and $m=m(|S|)$ and then chooses $k$ functions
$h_1, \ldots, h_k \in \mathcal{H}$ for some function family
$\mathcal{H}=\{h \colon \univ \to \mathbb{N}\}$.\footnote{For this
  example, we are implicitly assuming that the range of the hash
  functions admits a natural mapping to $[m]$.  In practice, this
  could be enforced in various ways, including allowing $\Rep$ to
  return an indication of error. }
It then computes an $m$-bit
array~$M$ (initialized to 0 everywhere) by setting $M[h_i(x)]=1$ for
all $i\in [k]$ and $x \in S$.  Finally, $\Rep$ returns
$\privaux=\emptystring$ and $\pubaux = \langle
M,h_1,h_2,\ldots,h_k\rangle$ as the representation.
%
Algorithm~$\Qry$ is defined so that, on input
$(\pubaux,\privaux,\qry_x)$, it returns the minimum of the values
$M[h_1(x)],\ldots,M[h_k(x)]$.  We note that the standard Bloom filter
analysis treats $h_1,h_2,\ldots,h_k$ as independent random functions,
which we may capture in the ROM with some simple adjustments to the above.

As a simple extension, consider $\Gamma=\mathbb{N}$ and modify the
classical Bloom filter so that it stores counters, rather than bits, at
each position in an array.  In particular, for each~$x \in S$ the
$\Rep$~algorithm increments the counters stored at positions
$h_1(x), h_2(x), \ldots, h_k(x)$, returns the final array of
counters as~$M$.  The $\Qry$-algorithm remains the same. Such a
construction gives a static version of a counting Bloom
filter~\cite{fan2000summary}.  We will capture dynamic versions, as well as
count-min sketches~\cite{cormode2005improved}, scalable Bloom filters~\cite{almeida2007scalable},
etc., in future work.\tsnote{?}

A Bloom filter with secret hash functions is captured by setting
$\privaux=(h_1,h_2,\ldots,h_k)$, and setting $\pubaux=M$.
Related constructions that use a secret key~$K$, but not for hashing (e.g.\ the PRP-based
construction from NY), are captured setting $\privaux=K, \pubaux=M$.

\tsnote{needs updating, since $\Sigma$ is gone, and the range of
  $\Rep$ is just $\bits^*$ with implicit encoding of whatever is the
  ``natural'' way to think about $\pubaux$ (e.g. counters, xor-shares)}
By setting $\Sigma=\bits^L$ for a specified $L>0$, we can capture
the so-called ``garbled Bloom filter'' due to Dong, Chen and
Wen~\cite{dong2013private}. \todo{Details have to be filled in.}
This stores xor-shares of bitstring~$x \in S$ at positions
$h_1(x),h_2(x),\ldots,h_k(x)$ in an array. Similarly, the
cuckoo-hashing construction from NY, which (loosely speaking) stores
a hash $g(x)$ at positions determined by
$h_1(x),h_2(x),\ldots,h_k(x)$, is captured when $L=|g(x)|$.
\todo{Check the correctness of this claim.} }
