\section{(Multiset) Data Structures}

\heading{Preliminaries. } When~$\univ$ is a set and $n>0$ is an
integer, we let $[\univ]^n$ denote the set of all size-$n$ subsets
of~$\univ$, and $[[\univ]]^n$ for the set of all multisets
over~$\univ$ with total multiplicity~$n$. \jnote{I guess we are
viewing sets as a special case of multisets, so maybe just say
that.} \tsnote{We may not need all
  of that.} We write $x \getsr \univ$ to denote
sampling an element uniformly from~$\univ$ and assigning it to~$x$,
and we extend this notation to randomized algorithms. When~$X$ is a
string over some alphabet~$\Sigma$, we write $|X|$ for the bitlength
of~$X$, relative to some fixed encoding.  When~$X$ is a multiset, we
overload the notation so that~$|X|$ is the total multiplicity
of~$X$.

Algorithms may be provided black-box access to one or more oracles,
which we write as superscripts, e.g., $F^{O_1,O_2,\ldots}$.  An \emph{adversary} is a randomized algorithm.

\heading{Data structures.}  We begin by defining a general primitive
that produces a representation of a given multiset, and provides a
mechanism to answer a specified collection of questions about that
multiset.
%Informally, the representation should serve as a good
%approximation of the multiset, at least with respect to the allowed queries.
\begin{definition}[Data structure] \rm
Fix a non-empty set~$\univ$, and a collection of (multi)sets~$\calS$
over $\univ$. A \emph{data structure} (for $\calS$) is a tuple
$\setprim=(\calQ,\Rep, \Qry)$ where:
\begin{itemize}
\item $\mathcal{Q}=\{q \colon \mathcal{S} \to \Gamma\}$ is a set
    of supported \emph{queries}, where $\Gamma$ is the set of query \emph{answers}.
\item $\Rep \colon \mathcal{S} \to \bits^*
    \times \bits^*$ is a randomized \emph{representation
    algorithm}, taking as input a (multi)set $S \in \mathcal{S}$, and outputting %a \emph{representation} $M \in \Sigma^*$,
    \emph{public representation data} $\pubaux \in \bits^*$ and
    \emph{private representation data} $\privaux \in \bits^*$. We write
    $(\pubaux,\privaux) \getsr \Rep(S)$ as shorthand for this execution.
\item $\Qry\colon \bits^* \times \bits^* \times \calQ \to
    \Gamma$ is a deterministic \emph{query-evaluation}
    algorithm, taking as input $\pubaux,\privaux \in \bits^*$
    and a query $q \in \calQ$, and outputting an \emph{answer}
    $a \in \Gamma$.  We write $a \gets \Qry(\pubaux,\privaux,q)$
    for this execution.
\end{itemize}
\hfill\dqed
\end{definition}

\todo{Insert discussion of syntax, if needed.  If we think it is
  helpful to discuss the size of the representation, we can declare it
as $|\pubaux|+|\privaux|$ and comment that if one wants to compare to
classical size bounds, a more fine-grained separation of the output of
$\Rep$ would allow this.}
%\tsnote{I implemented this change.}\tsnote{I'm now leaning towards
%having $\Rep$ output~$M$ and explicit pubic/private auxillary data.
%Pushing secrets into~$M$ makes it more difficult to capture (for
%example) the Neidermeier et al.\ attacks in our privacy notions; see
%Figure~\ref{fig:privacy}.  In those attacks, the adversary would get
%the bitarray (which one naturally thinks of as~$M$) but not the
%secret keys for the PRFs/hash functions.  You could say that the
%bitarray is put in $\aux$ and that~$M$ \emph{only} contains the
%secret keys, but this is unnatural.}

%All of the data structures that we will consider have the property that~$|M|$ does not depend on the
%coins of~$\Rep$. Thus we assume that~$|M|$ depends only on~$|S|$. In
%general this need not be true, but the assumption does not restrict
%our syntax from capturing natural constructions (e.g., those used in
%practice), and it makes it easier to state our security notions.

%\jnote{I think we agreed to just have the syntax output $(\pub, \privaux)$, with
%the semantics being that in our definitions $\pub$ is always
%  public and $\privaux$ is always private.
%  Different schemes can then choose whether $M$ is in $\pub$
%  or $\privaux$. If we do this, then need to be careful how to define size, though
%  I don't think we ever explicitly care about size for any of our results.}

\begin{definition}[Error probability and rate] \rm
For a given data structure $\setprim=(\calQ,\Rep, \Qry)$, let
$\mathrm{Err}(S,q,\pubaux,\privaux)$ be a predicate that is true iff
$\Qry(\pubaux,\privaux, q) \neq q(S)$.  Then $\Pi$ has \emph{error
probability} $\epsilon$ if $\max_{S,q}\Pr[(\pubaux,\privaux) \getsr
\Rep(S) : \mathrm{Err}(S,q,\pubaux,\privaux)=1 ] \leq \epsilon$.
%
It has \emph{error rate} $\tilde{\epsilon}$ if
$\max_{S}\Pr[(\pubaux,\privaux) \getsr \Rep(S);\, q \getsr \calQ :
\mathrm{Err}(S,q,\pubaux,\privaux)=1] \leq \tilde{\epsilon}$. \hfill\dqed
\end{definition}

%We will sometimes say that a data structure is
%$(1-\epsilon)$-correct, meaning it has error probability~$\epsilon$.
%Note that $1-\epsilon$ is a worst-case lower bound on the
%correctness of the data structure. \tsnote{Don't know if this will be necessary.}

These definitions, and those that follow, can be lifted to the random-oracle
model by giving $\Rep, \Qry$ access to a random oracle. In that
case, the probabilities defining the error probability and error
rate are also taken over choice of the random oracle.

We note that, while our error probability and error rate do not
distinguish among ``types'' of errors, they also do not preclude
instantiations (like the classical Bloom filter) that admit only one
type of error, e.g., false positives.


\def\bin{{\sf Bin}}
\heading{Error probability vs.\ error rate. }
We pause to highlight an important
difference between error probability and error rate.  In particular,
the error probability~$\epsilon$ is measured only over the coins
of~$\Rep$, for a fixed~$S$ and query~$q$. Once the coins of~$\Rep$
are fixed, it is not straightforward how to connect~$\epsilon$ to
events that depend on other sources of randomness.   We illustrate how
this surfaces with the following example.

Let~$\Pi$ be a data structure that has error-probability $\epsilon$,
and consider a correctness-violating attack that asks random
queries $q_1,q_2,\ldots,q_T$, i.e., uniform samples from~$\calQ$.
Let $\mathrm{Err}_i$ be a random variable indicating the event
$\Qry(\pubaux,\privaux,q_i)\neq q_i(S)$.  We know that the probability
that $\mathrm{Err}_i=1$ is at most~$\epsilon$, and it is tempting to
model the~$\mathrm{Err}_i$ as i.i.d. Bernoulli trials with success
probability~$\epsilon$. However, a moment's reflection shows that
this is not the case.  Concretely, consider the following scheme.
Fix $|S|=n$ for all $S \in \calS$, where $n \ll |\univ|$. For the
queries, let $\calQ = \{q_x\colon \calS \to \bits \,|\, \forall x
\in \univ\}$ where $\forall S \in \calS$ the predicate $q_x(S)=1
\Leftrightarrow x \in S$.  Let $\Rep(S)$ pick a random element~$s
\in S$, and return $M = S \setminus \{s\}$ and $\privaux =
\emptystring$.  To respond to queries,
let~$\Qry(\pubaux,\privaux,q_x)=1$ iff $x \in M$.  It is easy to see
that this scheme is $1/n$-correct; for any fixed~$S$, there are
exactly~$n$ queries~$q_x$ that have probability~$1/n$  (over the
coins of~$\Rep$) of causing a correctness error, and the remaining
$|\univ|-n$ queries have probability zero of causing an error.
Moreover, once the coins of~$\Rep$ are fixed, there is exactly
\emph{one} query that will cause an error.  Thus for random queries
from~$Q$, for all~$i\in[T]$ we have
$\Prob{\mathrm{Err}_i=1}=1/|\univ| \ll 1/n = \epsilon$.


\heading{Data structures for set-multiplicity queries.} Our
particular focus will be on data structures that support
multiplicity queries.

\begin{definition} \rm
A \emph{set-multiplicity data structure} is a data structure with
$\Gamma=\mathbb{N}$ and $\calQ=\{q_x\}_{x \in \univ}$, where
$q_x(S)$ is defined to be the multiplicity of~$x$ in the
multiset~$S$. \hfill\dqed
\end{definition}

The classical Bloom filter is a special case of a set-multiplicity
data structure, as set-membership queries are a type of
set-multiplicity query.  Specifically,
%\jnote{Only handles sets, not multisets.}
let $\Gamma=\bits$, and fix functions $k(\cdot)$ and $m(\cdot)$
parameterizing the scheme. Algorithm $\Rep$, on input a set~$S$,
computes $k=k(|S|)$ and $m=m(|S|)$ and then chooses $k$ functions
$h_1, \ldots, h_k \in \mathcal{H}$ for some function family
$\mathcal{H}=\{h \colon \univ \to \{1,2,\ldots,m\}\}$. \jnote{Note
depends on $m$. It might be easier (especially since we talk about
error) to already let the hash functions be independent random
oracles.} \tsnote{I don't see the problem.  $\Rep$ knows~$m$, so any
function (whose range admits at least~$m$ values) can be effectively
treated as having range~$[m]$. But I added a sentence at the end of
this paragraph, too.  See if that helps.} \jnote{$\Rep$ knows $m$,
but $m$ can take any value so we need an $\mathcal{H}$ that is
defined for every possible~$m$.} It then computes an $m$-bit
array~$M$ (initialized to 0 everywhere) by setting $M[h_i(x)]=1$ for
all $i\in [k]$ and $x \in S$.  Finally, $\Rep$ returns
$\privaux=\emptystring$ and $\pubaux = \langle
M,h_1,h_2,\ldots,h_k\rangle$ as the representation.
%
Algorithm~$\Qry$ is defined so that, on input
$(\pubaux,\privaux,q_x)$, it returns the minimum of the values
$M[h_1(x)],\ldots,M[h_k(x)]$.  We note that the standard Bloom filter
analysis treats $h_1,h_2,\ldots,h_k$ as independent random functions,
which we may capture in the ROM with some simple adjustments to the above.

As a simple extension, consider $\Gamma=\mathbb{N}$ and modify the
classical Bloom filter so that it stores counters, rather than bits, at
each position in an array.  In particular, for each~$x \in S$ the
$\Rep$~algorithm increments the counters stored at positions
$h_1(x), h_2(x), \ldots, h_k(x)$, returns the final array of
counters as~$M$.  The $\Qry$-algorithm remains the same. Such a
construction gives a static version of a counting Bloom
filter~\cite{xxx}.  We will capture dynamic versions, as well as
count-min sketches~\cite{xxx}, scalable Bloom filters~\cite{xxx},
etc., in future work.\tsnote{?}

A Bloom filter with secret hash functions is captured by setting
$\privaux=(h_1,h_2,\ldots,h_k)$, and setting $\pubaux=M$.
Related constructions that use a secret key~$K$, but not for hashing (e.g.\ the PRP-based
construction from NY), are captured setting $\privaux=K, \pubaux=M$.

By setting $\Sigma=\bits^L$ for a specified $L>0$, we can capture
the so-called ``garbled Bloom filter'' due to Dong, Chen and
Wen~\cite{xxx}. \todo{Details have to be filled in.} This stores
xor-shares of bitstring~$x \in S$ at positions
$h_1(x),h_2(x),\ldots,h_k(x)$ in an array. Similarly, the
cuckoo-hashing construction from NY, which (loosely speaking) stores
a hash $g(x)$ at positions determined by
$h_1(x),h_2(x),\ldots,h_k(x)$, is captured when $L=|g(x)|$.
\todo{Check the correctness of this claim.}
