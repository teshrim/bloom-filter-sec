\subsection{Notions of Privacy}
One can also consider the orthogonal notion of privacy for a set-representation.
\jnote{Actually, I am now unsure what the right way to define it is. It seems natural
to give the attacker $\aux$. But if we give them $\aux$ and $M$ then they have everything.
If $S$ has high min-entropy (element-wise), then it might still be
possible to have security here.}\tsnote{I think we will always bump up
against the (element-wise) min-entropy of the distribution over~$S$,
at least in the ``one-wayness'' style definition that I had been imagining. }
There are several possible definitions:
\begin{enumerate}
\item A ``one-wayness'' definition where $S$ has high min-entropy (element-wise), and we
give the attacker $\aux$ and either $M$ or oracle access to $\Qry$. The attacker
succeeds if it can output an element in~$S$.

\item An ``indistinguishability'' definition where either $S_0$ or $S_1$ is used, and the attacker
must determine which. Here the attacker is given $\aux$ but only oracle access to $\Qry$, and
it is disallowed from asking any query $q$ for which $q(S_0) \neq
q(S_1)$. \tsnote{I'm not convinced this is the right restriction.
  This says that security is required \emph{only} for properties that are
  true of both sets.  So a scheme with $M=S$ will be trivially secure
  against set membership queries.  } \tsnote{Do you mean to
  assume that the attacker does not ask~$q$ such that $q(S_0)=q(S_1)$
  because these are pointless?  Otherwise, I don't see a need for restrictions.}
%\tsnote{How will the adversary know that a query satisfies
%  this in (say) the case of the classical BF?  We can, of course,
%enforce it in the experiment, but then our resource accounting risks
%overcounting the number of queries.} \jnote{I was assuming the attacker knows $S_0, S_1$.}
%\tsnote{Also, we need that, in general, $\aux$ itself
%does not leak information about~$S_0$ vs.\ $S_1$. } \jnote{Sure, this would be implied
%by the definition in this case.}
\end{enumerate}

\begin{figure}[htp]
\centering
\hfpages{.45}{
\hpagess{.5}{.45}
{
\experimentv{$\ExpWPriv{\setprim,\distr{\calS}{}}{A}$}\\
$S \getsr \distr{\calS}{}$\\
$(M,\aux) \getsr \Rep(S)$\\
$z \getsr A^{\TestOracle}(\aux)$\\
if $z \in S$ then Return 1\\
Return 0

\medskip
\experimentv{$\ExpPriv{\setprim,\distr{\calS}{}}{A}$}\\
$S \getsr \distr{\calS}{}$\\
$(M,\aux) \getsr \Rep(S)$\\
$z \getsr A(M,\aux)$\\
if $z \in S$ then Return 1\\
Return 0\\
}
%
{
\oracle{$\TestOracle(q)$}\\
$a \gets \Qry(M,\aux,q)$\\
Return~$a$
}
}
%
{
\hpagess{.5}{.45}
{
\experimentv{$\ExpPrivInd{\setprim}{A}$}\\
$b \getsr \bits$\\
$z \getsr A^{\RepOracle,\TestOracle}()$\\
if $z = b$ then Return 1\\
Return 0\\
}
%
{
\oracle{$\RepOracle(S_0,S_1)$}\\
if $\mathrm{used}=\true$ then \\
\nudge Ret $\bot$\\
if $|S_0|\neq|S_1|$ then \\
\nudge Ret $\bot$\\
$\mathrm{used}=\true$\\
$(M,\aux) \getsr \Rep(S_b)$\\
Return $\aux$\\

\medskip
\oracle{$\TestOracle(q)$}\\
$a \gets \Qry(M,\aux,q)$\\
Return~$a$\\
}
}
\caption{Privacy notions. {\bf Left:} ``One-wayness'' style
  definitions. The top is a weaker notion, because the representation
  is hidden from the adversary. {\bf Right:} An
  ``indistinguishability'' style notion. (Still playing...)} 
\label{fig:privacy}
\end{figure}