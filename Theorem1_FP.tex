\heading{Linear-PRF construction. }
Here we consider a hash-based filter that was attacked by Neidermayer et al. in~\cite{xxx}.  Let $B_{\mathrm{lin}}=(\Hash,\Rep,\Qry)$ be defined as follows.  Fix non-empty set ~$\univ$ and integers $k,m,n>0$.  Let $F\colon\mathcal{K} \times \univ \to [m]$ be a function family.  The $\Hash$ algorithm samples two keys $K_0,K_1 \getsr \mathcal{K}$ and defines $h_j(x) = F_{K_0}(x) + j\cdot F_{K_1}(x) \bmod m$ for $j \in [1,k]$.  The $\Rep$ and $\Qry$ algorithms are the standard (determinisitic) BF ones. The following result shows, informally, that if~$F$ is a good PRF and the (non-adaptive) FP rate is sufficiently small, then $B_\mathrm{lin}$ is secure against adaptive FP-finding adversaries.

\fixme{Note: the following theorem is incorrect for $r>1$.  Or, at least, the proof does not cover $r>1$.}\acnote{Modified.}
\begin{theorem}\label{thm1}
Fix $k,m,n>0$, and let $B_\mathrm{lin}=(\Hash,\Rep,\Qry)$ be the hash-based filter just described. 
Let~$A$ be an adversary, compatible with $\ExpFPSecHash{B_\mathrm{lin},r}{\cdot}$, that asks at most~$q \times r$ queries in time $t$ to its $\QryOracle$ oracle.  Then 
\[
\AdvFPSecHash{B_{\mathrm{lin}},r}{A} \leq  \AdvPRF{F}{B_1} + \AdvPRF{F}{B_2}  +{\dbinom{q}{r}} \left( (1-e^{-kn/m})^k + O(1/n) \right)^r
\]
for PRF-adversaries~$B_1$ and $B_2$ given explicitly in the proof of
this theorem. Here, $B_1$ and $B_2$, each ask $q \times r$ oracle queries in time O(t). \tsnote{Statement about resources of these reductions.}
\end{theorem}
Before giving the proof, we note that Kirsch and Mitzenmacher~\cite{xxx} show
that the ($r=1$, non-adaptive) FP-probability for~$B_\mathrm{lin}$ is
always within $O(1/n)$ of $(1-e^{-kn/m})^k$, although the asymptotic
rate of convergence is faster than~$1/n$.  The factor of $\dbinom{q}{r}$ arises as a result
of removing adaptivity.  Finally, $\dbinom{q}{r} \leq q^r$ (with
reasonable tightness when~$r$ is small) in which case the final term in the bound behaves as 
$(q (1-e^{-kn/m})^k + O(q/n) )^r$; loosely, the bound one expects for
trying to find a single FP in each of~$r$ independent``rounds'', each round
consisting of~$q$ attempts.
%Before giving the proof, we note that  $(1-e^{-kn/m})^k \rightarrow 0$ as the ratio between the bit-length of the representation~$M$ and the size of the represented set~$S$, i.e.,$m/n$ tends to $\infty$.
%
\begin{proof}[\Cref{thm1}]
The proof of the theorem will use a game playing argument as shown in \Cref{fig:Game}. $\game{0}(A)$ exactly simulates $\ExpFPSecHash{B_\mathrm{lin},r}{A}$ with hash functions $h_j(x) =  F_{K_0}(x) + j\cdot F_{K_1}(x) \bmod m$. So,

\begin{equation}
\AdvFPSecHash{B_{\mathrm{lin}},r}{A} = \Prob{\game{0}(A)=1}\label{eq:0}
\end{equation}

$\game{1}(A)$ and $\game{2}(A)$ attack 2 different filters with hash functions 
$h_j(x) = ( F_{K_0}(x) + j\cdot \rho_2(x)) \bmod m$ and $h_j(x) = ( \rho_1(x) + j \cdot \rho_2(x)) \bmod m$, respectively.
Here, $\rho_1$ and $\rho_2$ are random functions sampled from $\Func{\univ,[m]}$. \\

In $\game{3}(A)$, $r$ random integers are chosen and the $\QryOracle$ oracle sets a $\bad$ flag if query response for any of the $r$ queries is false or any of the $q-r$ queries return true. Also, the hash functions are linear combinations of two random points in $[m]$, similar to $\game{2}(A)$. So, both have equal probability of returning 1. 

\begin{equation}
\Prob{\game{2}(A)=1} = \Prob{\game{3}(A)=1}\label{eq:3}
\end{equation}

The probability of winning  $\game{3}(A)$ is same as $\game{4}(A)$, except $\game{4}(A)$ outputs 0 if $\bad$ flag is set. In short, $\game{3}(A)$ and $\game{4}(A)$ are identical untill $\bad =true$.  On the  other hand, $\game{5}(A)$ simulates the non-adaptive counterpart of $\game{4}(A)$. 
 \tsnote{I have rewritten game $\game{3}(A)$ and added
  $\game{4}(A)$.  It should be that
  $\Prob{\game{2}(A)}=\Prob{\game{3}(A)=1}$. Moreover
  $\Prob{\game{4}(A)=1} = \Prob{\game{3}(A)=1 \wedge \neg\bad} =
  \Prob{\game{3}(A)=1}\Prob{\game{3}{A}:\,\neg\bad} =
  \Prob{\game{3}(A)=1}\frac{1}{\dbinom{q}{r}}$. Finally the event
  $\game{4}(A)=1$ happens iff $\bad \neq \true$.  That means there is
  a non-adaptive adversary that runs~$A$, simulates $\QryOracle$ by
  simply returning 1 iff $c \in \mathcal{I}$, and uses the queries
  with indices in $\mathcal{I}$ as its non-adaptive FPs.  To be
  totally clear, you might want to add a $\game{5}(A)$ that has this
  $\QryOracle$ behavior, and then give the non-adaptive FP-adversary~$A'$ that
simulates~$\game{5}(A)$, and so on. }\fixme{Clean up everything after this.}

\begin{align}
\nonumber \Prob{\game{4}(A)=1} &= \Prob{\game{3}(A)=1 \wedge \neg\bad}\\ 
\nonumber & = \Prob{\game{3}(A)=1}\Prob{\game{3}(A):\,\neg\bad}\\
& = \Prob{\game{3}(A)=1}\frac{1}{\dbinom{q}{r}}\label{eq:3a}\\
\Prob{\game{5}(A)=1} &= \Prob{\game{4}(A)=1}\label{eq:3b}
\end{align}

From  \Cref{eq:3,eq:3a,eq:3b}:
\begin{equation}
\Prob{\game{5}(A)=1} = \Prob{\game{2}(A)=1}\frac{1}{\dbinom{q}{r}}\label{eq:3c}
\end{equation}

\tsnote{The PRF experiment has not be defined yet, so strictly
  speaking these claims make no sense. }
The PRF game of $F$ is shown in \Cref{fig:prf}. Let prf-adversaries $B_1, B_2$ be as shown in \Cref{fig:BGame1}.  By
construction, these simulate $\game{1}(A)$ and $\game{2}(A)$, respectively. When $b=0$, $B_1$ wins if $\game{1}(A)$ outputs 0; $B_2$ wins if $\game{2}(A)$ returns 0. When $b=1$, $B_1$ and $B_2$ exactly simulate $\game{0}(A)$ and $\game{1}(A)$. So, 
\begin{align}
\Prob{\ExpPRF{F}{B_1} = 1\,|\,b=1} &= \Prob{\game{0}(A)=1}\label{eq:m1}\\
\Prob{\ExpPRF{F}{B_2} = 1\,|\,b=1} &= \Prob{\game{1}(A)=1}\label{eq:m2}\\
\Prob{\ExpPRF{F}{B_1} = 1\,|\,b=0} &= 1-\Prob{\game{1}(A)=1}\label{eq:m3}\\
\Prob{\ExpPRF{F}{B_2} = 1\,|\,b=0} &= 1-\Prob{\game{2}(A)=1}\label{eq:m4}
\end{align}

%\caption{Game playing argument}\label{fig:Game}
\begin{figure}
\fpage{.9}{
\hpagessl{.45}{.5}
{
\underline{\game{0}(A)}\\
$K_1, K_2 \getsr \calK$\\
$S \getsr \distr{\univ}{n}$\\
$M \getsr \Rep^{\HashOracle}(\calS)$\\
$\calZ \getsr A^{\QryOracle}(\calS)$\\
if $|\calZ| < r$ or $\calZ \cap \calS \neq \emptyset$ then \\
\nudge Ret 0\\
if $\exists z \in \calZ$ s.t. $\Qry^{\HashOracle}(M,z)=0$ then\\
\nudge Ret 0\\
Ret 1\\\\
%
\oracle{$\QryOracle(x)$}\\
Ret $\Qry^{\HashOracle}(M,x)$\\\\
%
\oracle{$\HashOracle(x)$}\\
for $i = 1$ to $k$\\
\nudge $h_i(x) = F_{K_1}(x)+i \cdot F_{K_2}(x) \bmod m$\\
Ret $\left(h_1(x),\ldots,h_k(x)\right)$
}
{
\underline{\game{1}(A)}\\
$K_1 \getsr \calK$\\
$\rho \getsr \Func{\univ,[m]}$\\
$S \getsr \distr{\univ}{n}$\\
$M \getsr \Rep^{\HashOracle}(\calS)$\\
$\calZ \getsr A^{\QryOracle}(\calS)$\\
if $|\calZ| < r$ or $\calZ \cap \calS \neq \emptyset$ then \\
\nudge Ret 0\\
if $\exists z \in \calZ$ s.t. $\Qry^{\HashOracle}(M,z)=0$ then\\
\nudge Ret 0\\
Ret 1\\\\
%
\oracle{$\QryOracle(x)$}\\
Ret $\Qry^{\HashOracle}(M,x)$\\\\
%
\oracle{$\HashOracle(x)$}\\
for $i=1$ to $k$\\
\nudge $h_i(x) = F_{K_1}(x)+i \cdot \rho(x) \bmod m$\\
Ret $\left(h_1(x),\ldots,h_k(x)\right)$
}
}
\fpage{.9}{
\hpagessl{.45}{.5}
{
\underline{\game{2}(A)}\\
$\rho_1,\rho_2 \getsr \Func{\univ,[m]}$\\
$S \getsr \distr{\univ}{n}$\\
$M \getsr \Rep^{\HashOracle}(\calS)$\\
$\calZ \getsr A^{\QryOracle}(\calS)$\\
if $|\calZ| < r$ or $\calZ \cap \calS \neq \emptyset$ then \\
\nudge Ret 0\\
if $\exists z \in \calZ$ s.t. $\Qry^{\HashOracle}(M,z)=0$ then\\
\nudge Ret 0\\
Ret 1\\\\
%
\oracle{$\QryOracle(x)$}\\
Ret $\Qry^{\HashOracle}(M,x)$\\\\
%
\oracle{$\HashOracle(x)$}\\
for $i=1$ to $k$\\
\nudge $h_i(x) = \rho_1(x)+i \cdot \rho_2(x) \bmod m$\\
Ret $\left(h_1(x),\ldots,h_k(x)\right)$
}
{
\underline{{$\game{3}(A)$},\fbox{$\game{4}(A)$}}\\
$c \gets 0$\\
$\mathcal{I}\getsr [\{1,2,\ldots,q\}]^r$\\
$\bad \gets \false$\\
%$\calY \gets \emptyset $\\
%$\rho_1,\rho_2 \getsr \Func{\univ,[m]}$\\
$S \getsr \distr{\univ}{n}$\\
$M \getsr \Rep^{\HashOracle}(\calS)$\\
$\calZ \getsr A^{\QryOracle}(\calS)$\\
if $|\calZ| < r$ or $\calZ \cap \calS \neq \emptyset$ then Ret 0\\
if $\forall z \in \calZ,\,\Qry^{\HashOracle}(M,z)=1$ then \\
\nudge if $\bad=\true$ then \fbox{Ret 0}\\
\nudge Ret 1\\
Ret 0\\\\
%if $|\calY| < t$ then \\ 
%\nudge Ret 0\\
%for $i=1$ to $t$ \\
%\nudge $z_i \getsr \calY $\\
%\nudge $\calY \gets \calY \setminus \{z_i\}$\\
%\nudge if $\Qry^{\HashOracle}(M,z_i)=0$ then\\
%\nudge \nudge Ret 0\\
%Ret 1\\\\
%
\oracle{$\QryOracle(x)$}\\
%$\calY \gets \calY \cup \{x\}$\\
$c \gets c+1$\\
$v \gets \Qry^{\HashOracle}(M,x)$\\
if $c \in \mathcal{I}$ and $v\neq 1$ then\\
\nudge $\bad \gets \true$ \\
%\nudge \fbox{Ret 1} \\
if $c \not\in \mathcal{I}$ and $v\neq 0$ then\\
\nudge $\bad \gets \true$\\
%\nudge \fbox{Ret 0}\\
Ret~$v$\\\\
%
\oracle{$\HashOracle(x)$}\\
$a,b \getsr [m]$\\
for $i = 1$ to~$k$\\
\nudge $v_i = a+i \cdot b \bmod m$\\
Ret $\left(v_1,\ldots,v_k\right)$
}
}
\caption{\Cref{thm1}:Game playing argument}\label{fig:Game}
\end{figure}

%\caption{$\Cref{thm1}:\game{5}(A)$ in game playing argument}\label{fig:game5}
\begin{figure}
\centering
\fpage{0.8}
{
\hpagess{0.5}{0.4}{
\underline{$\game{5}(A)$}\\
$c \gets 0$\\
$\mathcal{I}\getsr [\{1,2,\ldots,q\}]^r$\\
$\calY \gets \emptyset$\\
$S \getsr \distr{\univ}{n}$\\
$M \getsr \Rep^{\HashOracle}(\calS)$\\
$\calZ \getsr A^{\QryOracle}(\calS)$\\
if $|\calY| < r$ or $\calY \cap \calS \neq \emptyset$ then\\
\nudge Ret 0\\
if $\forall y \in \calY,\,\Qry^{\HashOracle}(M,y)=1$ then \\
\nudge Ret 1\\
Ret 0
}
%
{
\oracle{$\QryOracle(x)$}\\
$c \gets c+1$\\
if $c \in \mathcal{I}$ then\\
\nudge $\calY \gets \calY \cup {x}$\\
\nudge Ret 1\\
else\\
\nudge Ret 0 \\\\
%
\oracle{$\HashOracle(x)$}\\
$a,b \getsr [m]$\\
for $i = 1$ to~$k$\\
\nudge $v_i = a+i \cdot b \bmod m$\\
Ret $\left(v_1,\ldots,v_k\right)$
}
}
\caption{\Cref{thm1}:$\game{5}(A)$ in game playing argument}\label{fig:game5}
\end{figure}

%\caption{PRF game}\label{fig:prf}
\begin{figure}
\centering
\fpage{0.45}{
\hpagess{0.4}{0.35}
{
$\experimentv{\ExpPRF{F}{B}}$\\
$K \getsr \calK $\\
$\rho \getsr Func(\univ,[m])$\\
$b \getsr \bits$\\
$b' \getsr B^\calO$\\
Ret $[b = b']$\\
}
{
$\oracle{\calO \smallskip(x)}$\\
if $ b = 1$ then\\
\nudge Ret $F_K(x)$\\
Ret $\rho(x)$\\
}
}
\caption{PRF game}\label{fig:prf}
\end{figure}

%\caption{Adversary $B_1, B_2$ simulating $\game{1}(A)$ and $\game{2}(A)$ respectively}\label{fig:BGame1}
\begin{figure}
\centering
\fpage{0.9}{
\hpagessl{0.5}{0.5}
{
$\adversaryv{B_1^{\calO}}$\\
$K_1 \getsr \calK $\\
$\calS \getsr \distr{\univ}{n}$\\
$M \getsr \Rep^{\HashOracle}(\calS)$\\
When $A$ asks $\QryOracle(x)$:\\
$\nudge$ Ret $\Qry^{\HashOracle}(M,x)$\\
When $A$ halts with output $\calZ'$: \\
\nudge if $|\calZ'| < r$ or $\calZ' \cap \calS \neq \emptyset$ then \\
\nudge \nudge Ret 0\\
\nudge if $\exists z \in \calZ'$ s.t. $\Qry^{\HashOracle}(M,z)=0$ then\\
\nudge \nudge Ret 0\\
\nudge Ret 1\\\\
\oracle{$\HashOracle(x)$}\\
$ \mathrm{Ret} \nudge F_{K_1}(x) + i\cdot \calO(x)$, for $i\in[1,k]$\\
}
{
$\adversaryv{B_2^{\calO}}$\\
%$d \getsr \bits $\\
$\rho_2 \getsr \Func(\univ,[m])$ \\
$S \getsr \distr{\univ}{n}$\\
$M \getsr \Rep^{\HashOracle}(\calS)$\\
When $A$ asks $\QryOracle(x)$:\\
$\nudge$ Ret $\Qry^{\HashOracle}(M,x)$\\
When $A$ halts with output $\calZ'$: \\
\nudge if $|\calZ'| < r$ or $\calZ' \cap \calS \neq \emptyset$ then \\
\nudge \nudge Ret 0\\
\nudge if $\exists z \in \calZ'$ s.t. $\Qry^{\HashOracle}(M,z)=0$ then\\
\nudge \nudge Ret 0\\
\nudge Ret 1\\\\
\oracle{$\HashOracle(x)$}\\
%if $d=0$\\
$\mathrm{Ret} \nudge \calO(x) + i\cdot \rho_2(x)$, for $i\in[1,k]$
}
}
\caption{\Cref{thm1}:Adversary $B_1, B_2$ simulating $\game{1}(A)$ and $\game{2}(A)$ respectively}\label{fig:BGame1}
\end{figure}

%\caption{Non-adaptive adversaries simulating $\game{2}(A)$} \label{AGame3}
\begin{figure}
\centering
\fpage{0.8}{
\hpagessl{.5}{.4}
{
$\adversaryv{D^{\HashOracle}}$\\
$c \gets 0$\\
$\mathcal{I}\getsr [\{1,2,\ldots,q\}]^r$\\
$\calY, \calZ \gets \emptyset $ \\
When $A$ asks $\QryOracle(x)$:\\
\nudge $c \gets c+1$\\
\nudge if $c \in \mathcal{I}$ then\\
\nudge \nudge $\calY \gets \calY \cup {x}$\\
\nudge \nudge Ret~$1$\\
\nudge else \\
\nudge  \nudge Ret~$0$\\
When $A$ halts with output $\calZ'$\\
\nudge if $|\calZ'|<r \, \mbox{or} \, |\calY| < r$\\
\nudge \nudge Ret $\emptyset$\\
\nudge if $\forall y \in \calY,\,\Qry^{\HashOracle}(M,y)=1$ then \\
\nudge \nudge Ret $\calY$\\
\nudge Ret $\emptyset$\\\\
%
\oracle{$\HashOracle(x)$}\\
$a,b \getsr [m]$\\
for $i = 1$ to~$k$\\
\nudge $v_i = a+i \cdot b \bmod m$\\
Ret $\left(v_1,\ldots,v_k\right)$
}
{
$\adversaryv{D'^{\HashOracle}}$\\
$c \gets 0$\\
$\ell \getsr [q]$\\
%$\mathcal{I}\getsr [\{1,2,\ldots,q\}]^r$\\
$\calY \gets \emptyset$\\
When $A$ asks $\QryOracle(x)$:\\
$\nudge c \gets c + 1$\\
\nudge if $c == \ell$\\
$\nudge$ $\nudge r \gets \Qry^{\HashOracle}(M,x)$\\
$\nudge$ $\nudge \calY \gets \calY \cup {x}$\\
\nudge else\\
$\nudge$ $\nudge r \gets 0$\\
\nudge Ret $r$\\
When $A$ halts with output $\calZ'$\\
\nudge $z \gets \calY$\\
\nudge if $\Qry^{\HashOracle}(M,z)=1$ then\\
\nudge \nudge Ret $z$\\
\nudge else\\
\nudge \nudge Ret $\lambda$\\\\
%
\oracle{$\HashOracle(x)$}\\
$a,b \getsr [m]$\\
for $i = 1$ to~$k$\\
\nudge $v_i = a+i \cdot b \bmod m$\\
Ret $\left(v_1,\ldots,v_k\right)$
}
}
\caption{Non-adaptive adversaries simulating $\game{3}(A)$} \label{fig:AGame3}
\end{figure}	

%
\noindent
From \Cref{eq:m1,eq:m2,eq:m3,eq:m4}, %\acnote{compacted} \tsnote{this can be compacted}
\begin{align}
\nonumber \Prob{\ExpPRF{F}{B_1} = 1} &= .5\Prob{\ExpPRF{F}{B_1}=1\,|\,b=0} + .5\Prob{\ExpPRF{F}{B_1} = 1\,|\,b=1}\\
\nonumber 2\Prob{\ExpPRF{F}{B_1} = 1} &= 1-\Prob{\game{1}(A)=1} + \Prob{\game{0}(A)=1}\\
\nonumber (2\Prob{\ExpPRF{F}{B_1} = 1} - 1)  &= \Prob{\game{0}(A)=1} - \Prob{\game{1}(A)=1}\\
 \AdvPRF{F}{B_1} &= \Prob{\game{0}(A)=1} - \Prob{\game{1}(A)=1} \label{eq:1}\\
\nonumber& &\\
\nonumber \Prob{\ExpPRF{F}{B_2} = 1} &= .5\Prob{\ExpPRF{F}{B_2}=1\,|\,b=0} + .5\Prob{\ExpPRF{F}{B_2} = 1\,|\,b=1}\\
\nonumber 2\Prob{\ExpPRF{F}{B_2} = 1} &= 1-\Prob{\game{2}(A)=1} + \Prob{\game{1}(A)=1}\\
\nonumber (2\Prob{\ExpPRF{F}{B_2} = 1} - 1)  &= \Prob{\game{1}(A)=1} - \Prob{\game{2}(A)=1}\\
 \AdvPRF{F}{B_2} &= \Prob{\game{1}(A)=1} - \Prob{\game{2}(A)=1} \label{eq:2a}
\end{align}

\noindent
Adding \Cref{eq:1,eq:2a}, we have
\begin{equation}
\AdvPRF{F}{B_1} +\AdvPRF{F}{B_2} = \Prob{\game{0}(A)=1} - \Prob{\game{2}(A)=1}\label{eq:2}
\end{equation}

The classical FP requirement considers only non-adaptive (and information-theoretic) adversaries, and $D$ is one such adversary simulating $\game{5}(A)$ as closely as possible, whereas $D'$ mimics $D$ for $r=1$. $D'$ outputs a FP as in a classic Bloom filter and its FP-error is upperbounded by $[(1-e^{-kn/m})^k +O(1/n)]$ \cite{KirschMitzenmacher}.  %\acnote{changed}\tsnote{The classical FP requirement considers only non-adaptive (and information-theoretic) adversaries.  It suffices to say this.}
\noindent
From \Cref{fig:AGame3}, it is evident that
\begin{align}
\Prob{\game{5}(A)=1} &= \Prob{D \, \mbox{outputs $r$ FP} \,}\label{eq:4a}\\
\nonumber \Prob{D \, \mbox{outputs r FP} \,} &=  \Big[\Prob{D' \, \mbox{outputs FP} \,}\Big]^r\\
& \leq \Big[(1-e^{-kn/m})^k +O(1/n)\Big]^r \label{eq:4b}
\end{align}
\noindent
From \Cref{eq:4a,eq:4b}
\begin{equation}
\Prob{\game{5}(A)=1} \leq [(1-e^{-kn/m})^k + O(1/n)]^r \label{eq:4}
\end{equation}

\noindent
Using \ref{eq:4} in \ref{eq:3c}:
\begin{align}
\nonumber \Prob{\game{2}(A)=1}\frac{1}{\dbinom{q}{r}}  &\leq [(1-e^{-kn/m})^k +O(1/n)]^r\\
 \Prob{\game{2}(A)=1}  & \leq \dbinom{q}{r}[(1-e^{-kn/m})^k +O(1/n)]^r\label{eq:5}
\end{align}

\fixme{These adversaries are incorrect.  They are FP adversaries, so
  they get~$S$ as input.  They also do not sample random functions;
  use lazy sampling (a la game $\game{3}(A)$.  Please clean things up.)}
\noindent

From \Cref{eq:0,eq:2,eq:5} \\
\begin{align}
\nonumber \AdvFPSecHash{B_{\mathrm{lin}},r}{A} &- \Prob{\game{2}(A)=1} =  \AdvPRF{F}{B_1} + \AdvPRF{F}{B_2}   \\
\nonumber \AdvFPSecHash{B_{\mathrm{lin}},r}{A}  &=  \AdvPRF{F}{B_1} + \AdvPRF{F}{B_2} + \Prob{\game{2}(A)=1}  \\
\AdvFPSecHash{B_{\mathrm{lin}},r}{A} &\leq  \AdvPRF{F}{B_1} +
\AdvPRF{F}{B_2}  +{\dbinom{q}{r}} \left( (1-e^{-kn/m})^k + O(1/n) \right)^r
\end{align}

\end{proof}

