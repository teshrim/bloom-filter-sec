\heading{Linear-PRF construction. }
Here we consider a hash-based filter that was attacked by Neidermayer et al. in~\cite{xxx}.  Let $B_{\mathrm{lin}}=(\Hash,\Rep,\Qry)$ be defined as follows.  Fix non-empty sets~$\univ, \univ$ and integers $k,m>0$.  Let $F\colon\mathcal{K} \times \univ \to [m]$ be a function family.  The $\Hash$ algorithm samples two keys $K_0,K_1 \getsr \mathcal{K}$ and defines $h_j(x) = F_{K_0}(x) + j\cdot F_{K_1}(x) \bmod m$ for $j \in [k]$.  The $\Rep$ and $\Qry$ algorithms are the standard (determinisitic) BF ones. \tsnote{Should define these someplace.}  The following result shows, informally, that if~$F$ is a good PRF and the (non-adaptive) FP rate is sufficiently small, then $B_\mathrm{lin}$ is secure against adaptive FP-finding adversaries.

\fixme{Note: the following theorem is incorrect for $t>1$.  Or, at least, the proof does not cover $t>1$.}
\begin{theorem}\label{thm1}
Fix $k,m,n>0$, and let $B_\mathrm{lin}=(\Hash,\Rep,\Qry)$ be the hash-based filter just described. 
Let~$A$ be an adversary, compatible with $\ExpFPSecHash{B_\mathrm{lin},t}{\cdot}$, that asks at most~$q$ queries to its $\QryOracle$ oracle.  Then there exist adversaries~$B_1$ and $B_2$ such that
\[
\AdvFPSecHash{B_{\mathrm{lin}},t}{A} \leq  \AdvPRF{F}{B_1} + \AdvPRF{F}{B_2} + q \cdot  (1-e^{-kn/m})^k
\]
\end{theorem}
Before giving the proof, we note that  $(1-e^{-kn/m})^k \rightarrow 0$ as the ratio between the bit-length of the representation~$M$ and the size of the represented set~$S$, i.e.,$m/n$ tends to $\infty$.
%
\begin{proof}[Theorem~\ref{thm1}]
The proof of the theorem will use a game playing argument as shown in Fig. \ref{fig:Game}. $\game{0}(A)$ exactly simulates $\ExpFPSecHash{B,t}{A}$ with hash functions $h_j(x) =  F_{K_0}(x) + j\cdot F_{K_1}(x) \bmod m$. So,
\begin{equation}
\AdvFPSecHash{B_{\mathrm{lin}},t}{A} = \Prob{\game{0}(A)=1}\label{eq:0}
\end{equation}
 $\game{1}(A)$ and $\game{2}(A)$ attack 2 different filters with hash functions 
$h_j(x) = ( F_{K_0}(x) + j\cdot \rho_2(x)) \bmod m$ and $h_j(x) = ( \rho_1(x) + j \cdot \rho_2(x)) \bmod m$, respectively.
Here, $\rho_1$ and $\rho_2$ are random functions sampled from $\Func{\univ,[m]}$. Let adversaries $B_1, B_2$ simulate $\game{1}(A)$ and $\game{2}(A)$ respectively as shown in Fig. \ref{fig:BGame1}. When $b=1$, adversaries $B_1$ and $B_2$ exactly simulate $\game{0}(A)$ and $\game{1}(A)$, respectively.  When $b=0$, adversary $B_1$ wins if $\game{1}(A)$ outputs 0; adversary $B_2$ wins if $\game{2}(A)$ returns 0.  Thus we have
\begin{eqnarray}
\Prob{\ExpPRF{F}{B_1} = 1\,|\,b=1} &=& \Prob{\game{0}(A)=1}\label{eq:m1}\\
\Prob{\ExpPRF{F}{B_2} = 1\,|\,b=1} &=& \Prob{\game{1}(A)=1}\label{eq:m2}\\
\Prob{\ExpPRF{F}{B_1} = 1\,|\,b=0} &=& 1-\Prob{\game{1}(A)=1}\label{eq:m3}\\
\Prob{\ExpPRF{F}{B_2} = 1\,|\,b=0} &=& 1-\Prob{\game{2}(A)=1}\label{eq:m4}
\end{eqnarray}

\noindent
In a classical Bloom filter, adversary $A^{'}$ is non-adaptive, i.e., for some fixed $x \notin S$, $A^{'}$ targets the false-positive error conditioned on randomness of hash functions. \tsnote{The classical FP requirement considers only non-adaptive (and information-theoretic) adversaries.  It suffices to say this.}  But in the $\ExpFPSecHash{B,t}{A}$ game, $A$~is adaptive, as it can adapt its queries as well as its output based on response of $\QryOracle$ oracle. Let $A$ makes $q>t$ queries $x_1, x_2, ..., x_q$ to the $\QryOracle$ oracle. $\game{3}(A)$ simulates fp-error\tsnote{?} game of~$A^{'}$ as shown in Fig. \ref{fig:Game}.

%\caption{Game playing argument}\label{fig:Game}
\begin{figure}
\fpage{.9}{
\hpagessl{.45}{.5}
{
\underline{\game{0}(A)}\\
$K_1, K_2 \getsr \calK$\\
$S \getsr \distr{\univ}{n}$\\
$(M,\tau) \getsr \Rep^{\HashOracle}(\calS)$\\
$\calX \getsr A^{\QryOracle}(\calS)$\\
if $|\calX| < t$ or $\calX \cap \calS \neq \emptyset$ then \\
\nudge Ret 0\\
if $\exists x \in \calX$ s.t. $\Qry^{\HashOracle}(M,\tau,x)=0$ then\\
\nudge Ret 0\\
Ret 1\\\\
%
\oracle{$\QryOracle(x)$}\\
Ret $\Qry^{\HashOracle}(M,\tau,x)$\\\\
%
\oracle{$\HashOracle(x)$}\\
$h_i(x) = F_{K_1}(x)+i \cdot F_{K_2}(x)$, for $i\in[1,k]$\\
Ret $\left(h_1(x),\ldots,h_k(x)\right)$
}
{
\underline{\game{1}(A)}\\
$K_1 \getsr \calK$\\
$\rho \getsr \Func{\univ,[m]}$\\
$S \getsr \distr{\univ}{n}$\\
$(M,\tau) \getsr \Rep^{\HashOracle}(\calS)$\\
$\calX \getsr A^{\QryOracle}(\calS)$\\
if $|\calX| < t$ or $\calX \cap \calS \neq \emptyset$ then \\
\nudge Ret 0\\
if $\exists x \in \calX$ s.t. $\Qry^{\HashOracle}(M,\tau,x)=0$ then\\
\nudge Ret 0\\
Ret 1\\\\
%
\oracle{$\QryOracle(x)$}\\
Ret $\Qry^{\HashOracle}(M,\tau,x)$\\\\
%
\oracle{$\HashOracle(x)$}\\
$h_i(x) = F_{K_1}(x)+i \cdot \rho(x)$, for $i\in[1,k]$\\
Ret $\left(h_1(x),\ldots,h_k(x)\right)$
}
}
\fpage{.9}{
\hpagessl{.45}{.5}
{
\underline{\game{2}(A)}\\
$\rho_1,\rho_2 \getsr \Func{\univ,[m]}$\\
$S \getsr \distr{\univ}{n}$\\
$(M,\tau) \getsr \Rep^{\HashOracle}(\calS)$\\
$\calX \getsr A^{\QryOracle}(\calS)$\\
if $|\calX| < t$ or $\calX \cap \calS \neq \emptyset$ then \\
\nudge Ret 0\\
if $\exists x \in \calX$ s.t. $\Qry^{\HashOracle}(M,\tau,x)=0$ then\\
\nudge Ret 0\\
Ret 1\\\\
%
\oracle{$\QryOracle(x)$}\\
Ret $\Qry^{\HashOracle}(M,\tau,x)$\\\\
%
\oracle{$\HashOracle(x)$}\\
$h_i(x) = \rho_1(x)+i \cdot \rho_2(x)$, for $i\in[1,k]$\\
Ret $\left(h_1(x),\ldots,h_k(x)\right)$
}
{
\underline{$\game{3}(A)$}\\
$\ell \getsr [q]$\\
$c = 0$ \\%count will be used to keep track of queries made to $\QryOracle$\\
$\win \gets \false$\\
$\rho_1,\rho_2 \getsr \Func{\univ,[m]}$\\
$S \getsr \distr{\univ}{n}$\\
$(M,\tau) \getsr \Rep^{\HashOracle}(\calS)$\\
$\calX \getsr A^{\QryOracle}(\calS)$\\
if $|\calX| < t$ or $\calX \cap \calS \neq \emptyset$ then \\
\nudge Ret 0\\
if $\exists x \in \calX$ s.t. $\Qry^{\HashOracle}(M,\tau,x)=0$ then\\
\nudge Ret 0\\
Ret 1\\\\
%
\oracle{$\QryOracle(x)$}\\
$c=c +1$\\
if $(c == \ell)$\\
$\nudge \win = \true$\\
$\nudge \mathrm{Ret} \nudge \Qry^{\HashOracle}(M,\tau,x)$\\
else \\
$\nudge \mathrm{Ret}\nudge 0$\\\\
%
\oracle{$\HashOracle(x)$}\\
$h_i(x) = \rho_1(x)+i \cdot \rho_2(x)$, for $i\in[1,k]$\\
Ret $\left(h_1(x),\ldots,h_k(x)\right)$
}
}
\caption{Theorem \ref{thm1}:Game playing argument}\label{fig:Game}
\end{figure}

%\caption{Adversary $B_1, B_2$ simulating $\game{1}(A)$ and $\game{2}(A)$ respectively}\label{fig:BGame1}
\begin{figure}
\centering
\fpage{0.9}{
\hpagessl{0.5}{0.5}
{
$\adversaryv{B_1^{\calO}}$\\
$K_1 \getsr \calK $\\
$S \getsr \distr{\univ}{n}$\\
$(M,\tau) \getsr \Rep^{\HashOracle}(\calS)$\\
When $A$ asks $\QryOracle(x)$:\\
$\nudge$ Ret $\Qry^{\HashOracle}(M,\tau,x)$\\
When $A$ halts with output $\calX^{'}$: \\
if $|\calX^{'}| < t$ or $\calX^{'} \cap \calS \neq \emptyset$ then \\
\nudge Ret 0\\
if $\exists x \in \calX^{'}$ s.t. $\Qry^{\HashOracle}(M,\tau,x)=0$ then\\
\nudge Ret 0\\
Ret 1\\\\
\oracle{$\HashOracle(x)$}\\
$ \mathrm{Ret} \nudge F_{K_1}(x) + i\cdot \calO(x)$, for $i\in[1,k]$\\
}
{
$\adversaryv{B_2^{\calO}}$\\
%$d \getsr \bits $\\
$\rho_2 \getsr \Func(\univ,[m])$ \\
$S \getsr \distr{\univ}{n}$\\
$(M,\tau) \getsr \Rep^{\HashOracle}(\calS)$\\
When $A$ asks $\QryOracle(x)$:\\
$\nudge$ Ret $\Qry^{\HashOracle}(M,\tau,x)$\\
When $A$ halts with output $\calX^{'}$: \\
if $|\calX^{'}| < t$ or $\calX^{'} \cap \calS \neq \emptyset$ then \\
\nudge Ret 0\\
if $\exists x \in \calX^{'}$ s.t. $\Qry^{\HashOracle}(M,\tau,x)=0$ then\\
\nudge Ret 0\\
Ret 1\\\\
\oracle{$\HashOracle(x)$}\\
%if $d=0$\\
$\mathrm{Ret} \nudge \calO(x) + i\cdot \rho_2(x)$, for $i\in[1,k]$
}
}
\caption{Theorem \ref{thm1}:Adversary $B_1, B_2$ simulating $\game{1}(A)$ and $\game{2}(A)$ respectively}\label{fig:BGame1}
\end{figure}

From eqns. \ref{eq:m1}, \ref{eq:m2}, \ref{eq:m3} and \ref{eq:m4}, \tsnote{this can be compacted}
\begin{eqnarray}
\nonumber \Prob{\ExpPRF{F}{B_1} = 1} &=& .5\Prob{\ExpPRF{F}{B_1}=1\,|\,b=0} + .5\Prob{\ExpPRF{F}{B_1} = 1\,|\,b=1}\\
\nonumber 2\Prob{\ExpPRF{F}{B_1} = 1} &=& 1-\Prob{\game{1}(A)=1} + \Prob{\game{0}(A)=1}\\
\nonumber (2\Prob{\ExpPRF{F}{B_1} = 1} - 1)  &=& \Prob{\game{0}(A)=1} - \Prob{\game{1}(A)=1}\\
 \AdvPRF{F}{B_1} &=& \Prob{\game{0}(A)=1} - \Prob{\game{1}(A)=1} \label{eq:1}\\
\nonumber& &\\
\nonumber \Prob{\ExpPRF{F}{B_2} = 1} &=& .5\Prob{\ExpPRF{F}{B_2}=1\,|\,b=0} + .5\Prob{\ExpPRF{F}{B_2} = 1\,|\,b=1}\\
\nonumber 2\Prob{\ExpPRF{F}{B_2} = 1} &=& 1-\Prob{\game{2}(A)=1} + \Prob{\game{1}(A)=1}\\
\nonumber (2\Prob{\ExpPRF{F}{B_2} = 1} - 1)  &=& \Prob{\game{1}(A)=1} - \Prob{\game{2}(A)=1}\\
 \AdvPRF{F}{B_2} &=& \Prob{\game{1}(A)=1} - \Prob{\game{2}(A)=1} \label{eq:2}
\end{eqnarray}

\noindent
Adding eqn. \ref{eq:1} and \ref{eq:2}, we have
\begin{equation}
\AdvPRF{F}{B_2} +\AdvPRF{F}{B_1} = \Prob{\game{0}(A)=1} - \Prob{\game{2}(A)=1}\label{eq:2a}
\end{equation}

In $\game{3}(A)$, assume, without loss of generality that $A$ halts with output $x$ immediately, if it asks $x$ to $\QryOracle$ and $\QryOracle$ returns $1$. \fixme{This only works when $t=1$!  This is where the proof breaks down for the case of $t>1$.} Thus, all queries to $\QryOracle$ return 0 until the first one. Hence,

\begin{equation}
\Prob{\game{2}(A)=1} \leq q \cdot \Prob{\game{3}(A)=1} = q \cdot \Prob{\game{3}(A)=1\,|\,\win=\true}\label{eq:3}
\end{equation}

In Fig. \ref{AGame3}, adversary $A'$ simulates $\game{3}(A)$. Since, adversary $A'$ exactly simulates $\game{3}(A)$ conditioned on $\win = \true$, 
\begin{eqnarray}
\nonumber \Prob{\game{3}(A)\,|\,\win=\true} &=& \Prob{A' \mbox{outputs FP}}\\
& \leq & \epsilon,\mbox{ where } \epsilon = (1-e^{-kn/m})^k\label{eq:4}
\end{eqnarray}

%\caption{Adversary $A^{'}$ simulating $\game{3}(A)$} \label{AGame3}
\begin{figure}
\centering
\fpage{0.4}{
$\adversaryv{A'^{\HashOracle}}$\\
$\ell \getsr [q]$, where $[q] = {0,1, \cdots, q-1}$\\
$count = 0$\\
$\rho_1,\rho_2 \getsr \Func{\univ,[m]}$\\
$S \getsr \distr{\univ}{n}$\\
$(M,\tau) \getsr \Rep^{\HashOracle}(\calS)$\\
When $A$ asks $\QryOracle(x)$:\\
$\nudge count = count + 1$\\
\nudge if $count == \ell$\\
$\nudge$ $\nudge r \gets \Qry^{\HashOracle}(M,\tau,x)$\\
\nudge else\\
$\nudge$ $\nudge r = 0$\\
\nudge Ret $r$\\
When $A$ halts with output $\calX'$\\
\nudge if $|\calX'| < t$ or $\calX' \cap \calS \neq \emptyset$ then \\
\nudge\nudge Ret 0\\
\nudge if $\exists x \in \calX^{'}$ s.t. $\Qry^{\HashOracle}(M,\tau,x)=0$ then\\
\nudge \nudge Ret 0\\
\nudge Ret 1\\
%
\oracle{$\HashOracle(x)$}\\
$h_i(x) = \rho_1(x)+i \cdot \rho_2(x)$, for $i\in[1,k]$\\
Ret $\left(h_1(x),\ldots,h_k(x)\right)$
}
\caption{Theorem \ref{thm1}:Adversary $A'$ simulating $\game{3}(A)$} \label{AGame3}
\end{figure}	

From eqn. \ref{eq:0}, \ref{eq:2a}, \ref{eq:3} and \ref{eq:4}: \\
\begin{eqnarray}
\nonumber \AdvFPSecHash{B_{\mathrm{lin}},t}{A} &-& \Prob{\game{2}(A)=1} =  \AdvPRF{F}{B_1} + \AdvPRF{F}{B_2}   \\
\nonumber \AdvFPSecHash{B_{\mathrm{lin}},t}{A}  &=&  \AdvPRF{F}{B_1} + \AdvPRF{F}{B_2} + \Prob{\game{2}(A)=1}  \\
\AdvFPSecHash{B_{\mathrm{lin}},t}{A} &\leq&  \AdvPRF{F}{B_1} + \AdvPRF{F}{B_2}  + q \cdot \epsilon
\end{eqnarray}
\end{proof}


\clearpage
