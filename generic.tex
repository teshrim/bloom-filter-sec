\section{Secure Multiset Representation}
\def\pub{{\sf pub}}
\def\aux{{\sf pub}}

\heading{Preliminaries. }
When~$\univ$ is a set and $n>0$ is an integer, we let $[\univ]^n$ denote the set of all size-$n$ subsets of~$\univ$. We write $x \getsr \univ$ to denote sampling an element from~$\univ$ and assigning this to~$x$, and we extend this notation to randomized algorithms.  When~$X$ is a string over some alphabet~$\Sigma$, we write $|X|$ for the bitlength of~$X$, relative to some fixed encoding.  When~$X$ is a multiset, we overload the notation so that~$|X|$ is the number of elements in~$X$ (with mutiplicity).

heading{{Static (Multi)set-representations. }
\begin{definition} 
Fix a universe~$U$ and an alphabet~$\Sigma$.  A \emph{static (multi)set-representation primitive} is a pair of algorithms $(\Rep, \Qry)$
parameterized by a collection $\mathcal{S}$ of multisets, each consisting of elements from~$U$ and a collection
of supported queries $\mathcal{Q}=\{q \colon \mathcal{S} \to \mathbb{R}\}$, where:
\begin{itemize}
\item $\Rep$ is a randomized algorithm taking as input $S \in \mathcal{S}$,
and outputting a pair of strings $(M,\aux)$, the \emph{representation} $M \in \Sigma^*$ (of~$S$) 
and the \emph{auxillary information} $\aux \in \Sigma^*$ (of~$M$).
\item $\Qry$ is a deterministic algorithm that takes as input $M, \aux$,
and a query $q\in \mathcal{Q}$, and returns value~$a$.
\end{itemize}
A set-representation is \emph{$\epsilon$-correct} if for all $S \in \mathcal{S}$ and
all $q \in \mathcal{Q}$ it holds that
\begin{eqnarray}
\Pr[(M,\aux) \getsr \Rep(S) : \Qry(M, \aux, q) \neq q(S)] \leq \epsilon, \label{eqn:correctness}
\end{eqnarray}
%where $q(S)$ denotes the correct answer to the query $q$ on the set~$S$.
\end{definition}

\todo{Readdress issue of representation size.  A general statement might be something like `` The \emph{size} of a set-representation is the maximum (bit)length of any~$M$ that can be output by $\Rep$.''  But all of the natural constructions will have representation whose size will not depend on the coins of $\Rep$, i.e. they depend on $|S|$, number of hash functions, desired error rate. }

Our treatment of $M,\aux$ is not symmetric, since we only count $M$ against the size
of the set-representation. When we define security
we may give the attacker $\aux$ but not~$M$, and in that case the set-representation cannot
trivially make everything part of~$\aux$. 
%\tsnote{Encoding secrets into~$M$ may cause some headaches in the security definitions; we'll see. }
%\tsnote{By the way, I'm coming around to the
%  viewpoint that it is fair to include in the ``size'' of the
%  representation any secret information.  In makes more difficult
%  comparisons to classical lower-bounds (which [NY] makes, incorrectly
% IMO), but it does seem fair to say ``this is how many bits of memory
% you need to allocate to represent this set \emph{securely}.''}

The above definition can be lifted to the random-oracle model in the natural way
by giving $\Rep, \Qry$ access to a random oracle. In that case,
the probability in Equation~(\ref{eqn:correctness})
is also taken over choice of the random oracle.

\heading{Examples.}
It is not hard to see Bloom filters are one possible instantiation of the above. If we fix some
size bound~$n$, then $\mathcal{S}$ becomes the collection of all subsets of $U$
of size (at most)~$n$. We also fix $\mathcal{Q} = U$, where a query ``$x$'' has
the semantics ``is $x$ in $S$?'' Algorithm $\Rep$, which has values $k, m$ hardwired,
chooses a set of hash functions $h_1, \ldots, h_k$ and computes an $m$-bit
array in the usual way; a natural possibility is to set $\aux = (h_1, \ldots, h_k)$ (encoded as a string) and
to let~$M$ be the bit array.  
%\tsnote{If $\mathcal{Q}=U$ and we use the
%  semantics you suggest, then isn't Equation~\ref{eqn:correctness}
%  both correctness and soundness?  If so, I don't see how this
%  captures the classical BF, which requires $\epsilon=0$ for one kind
%  of error, but allows $\epsilon>0$ for the other. } 
%\jnote{The definition does not distinguish
%  between elements in the set (for which $q(S)=1$) or outside the set (for which
%  $q(S)=0$). But a Bloom filter does satisfy Equation~\ref{eqn:correctness} in either case.}
%  \jnote{In that sense, I am using ``soundness'' and ``correctness'' interchangeably here.}
  \jnote{I'm not tied to allowing false negatives when we specialize the definition to the case
  of set-membership queries. But in the abstract definition it seems unclear why one would choose
  some queries on which error is ok and other queries where error is not allowed. Though if one
  wanted to be general, one could simply have two (disjoint) sets of queries $\mathcal{Q}_1, \mathcal{Q}_2$
  where error is allowed on the former but not on the latter.}
We note that while our correctness condition does not distinguish among types of errors, it also does not preclude instantiations (like the classical Bloom filter) that admit only one type of error, e.g. false-positives.  

Setting $\Sigma=\bits^{\leq L}$ allows one to store a bitstring of length at most~$L$ at each location.  This allows our syntax to capture the cuckoo-hashing construction from NY, as well as the ``garbled Bloom filter'' from Dong, Chen and Wen~\cite{xxx}, among other things.  Setting $\Sigma=\mathbb{N}$ would allow for the storage of counters (say) at each position in~$M$, and a query ``$x$'' with the semantics ``what is the multiplicity of~$x$ in~$S$.''  (This does not capture counting Bloom filters, or the like, which do not have static representations.)  \tsnote{I guess this can all be accomplished with $\Sigma=\bits^*$, but it isn't necessarily the ``natural'' way.}
%Our syntax also does not capture filters that are mutable, that is, whose representation may change over time. On the other hand, we can state a simple correctness condition for these kinds of filters; this will not be true when we further generalize to capture, properly, counters and mutability.
