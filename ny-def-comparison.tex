\section{Comparison to the Naor-Yogev definition.}
The structure of the Naor-Yogev definition is different from ours.
Lifting their definition\footnote{They focus specifically on set-representation data
structures
with no false negatives, and their data-structure syntax
does not distinguish public and private portions of the representation. Nevertheless, it is straightforward
to extend their definition to more closely match ours.}
to our setting,
the experiment is similar but the attacker
succeeds only if it outputs a (single) query $\qry$ for which
$\Qry(\pubaux,\privaux, \qry) \neq \qry(S)$  \emph{subject to the restriction that it did not
previously query $\qry$ to its $\TestOracle$ oracle}. We refer to
Figure~\ref{fig:NYcorrectness} for a definition of the relevant experiment.
We define the advantage of adversary~$A$ as
$\NYAdvCorrect{\Pi}{A} = \Prob{\NYExpCorrect{\Pi}{A}=1}$,
and write $\NYAdvCorrect{\Pi,r}{t,q}$ for the maximum over
all~$t$-time adversaries that ask at most~$q$ queries. We say that a
data structure~$\Pi$ is $(t,q,\epsilon)$-NY-correct if
$\NYAdvCorrect{\Pi}{t,q} \leq \epsilon$. \jnote{This should all probably
move to an appendix; it's a lot of notation to digest for something we never use again
in the paper.}

\jnote{Just noticed something (else) odd about the NY definition: we cannot assume w.l.o.g.\
that $A$ makes exactly $q$ queries, and in fact it is possible to have cases where increasing
the number of queries the attacker makes can decrease its advantage! This seems like
another drawback of the definition.}\tsnote{Really?  That seems worth
pointing out, as part of our list of complaints.}

\begin{figure}[htp]
\centering
\fpage{.6}{
\hpagess{.5}{.45}
{
\experimentv{$\NYExpCorrect{\setprim}{A}$}\\
$S \getsr A$\\
$\mathcal{C} \gets \emptyset$\\
%$\mathrm{err}\gets 0$\\
$(\pubaux,\privaux) \getsr \Rep(S)$\\
$\qry \getsr A^{\TestOracle}(\pubaux)$\\
if $\left(\qry \not \in \mathcal{C}\right.$ and \\
\nudge $\left.\Qry(\pubaux,\privaux,\qry) \neq \qry(S)\right)$ \\
\nudge \nudge then Return 1\\
Return 0
}
%
{
\oracle{$\TestOracle(\qry)$}\\
$a \gets \Qry(\pubaux,\privaux,\qry)$\\
$\mathcal{C} \gets \mathcal{C} \cup \{\qry\}$ \\
%\nudge $\mathrm{err}\gets\mathrm{err}+1$\\
Return~$a$
}
}
\caption{The Naor-Yogev definition of
correctness, adapted to our setting.}
\label{fig:NYcorrectness}
\end{figure}

With the above in place we can now relate their definition to ours.

\jnote{Do we have a compelling example of why the definition in Figure~\ref{fig:correctness}
is ``better'' than the definition in
Figure~\ref{fig:NYcorrectness}?}\tsnote{I don't know, but we do seem
to have a compelling list of shortcomings of their notion, which ours
doesn't seem to suffer. Anyway, it's a bit of an unfair comparison, in
their favor.  You're lifting their notion to ours, which is more
general.  And by the time we get to that, we have already done them
the favor of cleaning up the syntax, so that their ``lifted'' notion
looks better than it was.} \jnote{I agree, it's just that once we have ``lifted'' it
appropriately, it's not clear what's wrong with it and why we introduce a different
style of definition.}

\begin{theorem}
If $\Pi$ is $(t,q,\epsilon)$-NY-correct, then it is also
$(t, q', r, q'\epsilon/r)$-correct for any $q' \leq q+1$ and $r\geq
1$.
\end{theorem}
\begin{proof}
Let $\Pi$ be a data structure that is $(t,q,\epsilon)$-NY-correct,
and assume that for some $q' \leq q+1$ and
$r \geq 1$
there is an adversary $A$ running in time~$t$
and making $q'$ oracle queries such that
$\AdvCorrect{\Pi,r}{A} > q'\epsilon/r$.
(Note that
we may assume $A$ always makes exactly $q'$ queries without loss of generality.)
With probability at least
$q'\epsilon/r$ in an execution of $\ExpCorrect{\Pi,r}{A}$, we have that
$A$ makes at least $r$ distinct queries to $\TestOracle$ for which
an incorrect answer is returned. Let $A'$ be the algorithm that simply
runs~$A$, but chooses
uniformly one of the $q'$ queries of $A$ to its $\TestOracle$ oracle and outputs that query
as its final output. Then with probability at least $r/q' \cdot (q'\epsilon/r)=\epsilon$
the query chosen by $A'$
leads to an incorrect answer, and was not previously asked to the $\TestOracle$ oracle.
Since the running time of $A'$ is at most $t$, and it makes at most $q'-1 \leq q$ queries
to its oracle, this is a contradiction.
\end{proof}

\jnote{The above is tight, at least for $r=1$. Specifically, consider a scheme
in which every query is independently
answered incorrectly with probability~$\epsilon$. Such a
scheme will be $(t,q,\epsilon)$-NY-correct
for any $t, q$, however an adversary making $q=1/\epsilon$ queries has constant advantage
under our correctness definition (for $r=1$).}

\begin{theorem}
If $\Pi$ is $(t, q, 1, \epsilon)$-correct, then it is also $(t, q-1, \epsilon)$-NY-correct.
\end{theorem}

\begin{proof}
Let $\Pi$ be a data structure that is $(t,q,1,\epsilon)$-correct,
and assume
there is an adversary $A$ running in time~$t$
and making at most $q-1$ oracle queries such that
$\NYAdvCorrect{\Pi}{A} > \epsilon$.
Let $A'$ be the algorithm that simply
runs~$A$, passing the oracle queries of~$A$ to its own oracle, until~$A$ terminates
with output~$\qry$; then, $A'$ sends~$\qry$ to $\TestOracle$. It is immediate that $A'$ makes at most $q$ oracle queries, and
$\AdvCorrect{\Pi,1}{A'} \geq \NYAdvCorrect{\Pi}{A}$, a contradiction.
\end{proof}

\begin{theorem}
There exists a data structure $\Pi$ that
is $(t, q, 2, 0)$-correct, but is not $(t, 0, \epsilon)$-NY-correct
for any $\epsilon<1$.
\end{theorem}
\begin{proof}
Let $\Pi$ be the set-membership data structure in which $\Rep(S)$ outputs
$\pubaux=\privaux=S$, and
$\Qry(S, S, \qry_y)$ outputs $\qry_y(S)$ if $y \neq x$ but outputs $1-\qry_x(S)$ otherwise,
where $x \in {\cal X}$ is a fixed element of the universe. This
scheme always answers incorrectly for a single, fixed query, and answers
correctly for every other query. The theorem follows.
\end{proof}

\jnote{I'm not sure what to conclude from any of the results above. The proofs
are also all trivial.} \tsnote{Trivial or not, it's all part of the
overall contribution, which is putting this topic on a solid
foundation.  We can move it all to an appendix, but I still see it as
very worthwhile.} \jnote{It's worth having for completeness, I'm just not sure
what we (or the reader) is supposed to do with it. It's not like we can draw
any simple conclusions like ``our definition implies theirs'' or anything.}

%%%%%%%%%%%%%%%%%%%%%%%%%%%%%%%%%%%%%%%%%%%%%%%%%%%%%
%%%%%%%%%%%%%%%%%%%%%%%%%%%%%%%%%%%%%%%%%%%%%%%%%%%%%
%%%%%%%%%%%%%%%%%%%%%%%%%%%%%%%%%%%%%%%%%%%%%%%%%%%%%