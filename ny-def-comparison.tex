\section{Comparison to the Naor-Yogev Definition}
\label{sec:compare-defs}
%\fixme{Uses $(t,q,\epsilon)$-style notion and theorem statements.  May
%be okay because this is a self contained appendix, but better if
%consistent with the rest of the document.}
We briefly describe the differences between the Naor-Yogev definition of correctness and ours.
For starters,
they focus specifically on set-representation data
structures with no false negatives, and (more importantly) their data-structure syntax
does not distinguish between public and private portions of a
representation. Nevertheless, it is straightforward
to extend their definition to more closely match ours; we do that below.
A more fundamental difference is that in
their experiment for defining correctness, the attacker
succeeds only if it outputs a (single) query $\qry$ for which
$\Qry(\pubaux,\privaux, \qry) \neq \qry(S)$  \emph{and for which it did not
previously query $\qry$ to its $\TestOracle$ oracle}. We refer to
Figure~\ref{fig:NYcorrectness} for a definition of the relevant experiment.
We define the advantage of adversary~$A$ relative to their experiment
as
$\NYAdvCorrect{\Pi}{A} = \Prob{\NYExpCorrect{\Pi}{A}=1}$,
and let $\NYAdvCorrect{\Pi}{t,q}$ be the maximum of this value taken over
all~$t$-time adversaries that ask at most~$q$ queries. %We say a
%data structure~$\Pi$ is \emph{$(t,q,\epsilon)$-NY-correct} if
%$\NYAdvCorrect{\Pi}{t,q} \leq \epsilon$.
%\jnote{This should all probably
%move to an appendix; it's a lot of notation to digest for something we never use again
%in the paper.}

%\jnote{Just noticed something (else) odd about the NY definition: we cannot assume w.l.o.g.\
%that $A$ makes exactly $q$ queries, and in fact it is possible to have cases where increasing
%the number of queries the attacker makes can decrease its advantage! This seems like
%another drawback of the definition.}\tsnote{Really?  That seems worth
%pointing out, as part of our list of complaints.}

\begin{figure}[htp]
\centering
\fpage{.65}{
\hpagess{.5}{.45}
{
\experimentv{$\NYExpCorrect{\setprim}{A}$}\\
$S \getsr A$\\
$\mathcal{C} \gets \emptyset$\\
%$\mathrm{err}\gets 0$\\
$(\pubaux,\privaux) \getsr \Rep(S)$\\
$\qry \getsr A^{\TestOracle}(\pubaux)$\\
if $\left(\qry \not \in \mathcal{C}\right.$ and \\
\nudge $\left.\Qry(\pubaux,\privaux,\qry) \neq \qry(S)\right)$ \\
\nudge \nudge then Return 1\\
Return 0
}
%
{
\oracle{$\TestOracle(\qry)$}\\
$a \gets \Qry(\pubaux,\privaux,\qry)$\\
$\mathcal{C} \gets \mathcal{C} \cup \{\qry\}$ \\
%\nudge $\mathrm{err}\gets\mathrm{err}+1$\\
Return~$a$
}
}
\caption{The Naor-Yogev definition of correctness, adapted to our setting.}
\label{fig:NYcorrectness}
\end{figure}

With the above in place we can now relate their definition to ours.
\begin{theorem}
Fix $\Pi$. If $\NYAdvCorrect{\Pi}{t,q} \leq \epsilon$, then for any $r \geq 1$ and $q' \leq q+1$ it holds that
$\AdvCorrect{\setprim,r}{t,q'} \leq q'\epsilon/r$.
\end{theorem}
\begin{proof}
Assume that for some $q' \leq q+1$ and
$r \geq 1$
there is an adversary $A$ running in time~$t$
and making $q'$ oracle queries such that
$\AdvCorrect{\Pi,r}{A} > q'\epsilon/r$.
(Note
we may assume that $A$ always makes exactly $q'$ queries without loss of generality.)
This means that with probability at least
$q'\epsilon/r$ in an execution of $\ExpCorrect{\Pi,r}{A}$, we have that
$A$ makes at least $r$ distinct queries to $\TestOracle$ for which
an incorrect answer is returned. Let $A'$ be the algorithm that simply
runs~$A$, but chooses
uniformly one of the $q'$ queries of $A$ to its $\TestOracle$ oracle and outputs that query
as its final output. Then with probability at least $r/q' \cdot (q'\epsilon/r)=\epsilon$
the query chosen by $A'$
leads to an incorrect answer, and was not previously asked to the $\TestOracle$ oracle.
Since the running time of $A'$ is at most $t$, and it makes at most $q'-1 \leq q$ queries
to its oracle, this is a contradiction. \hfill\qed
\end{proof}

We remark that the above is tight, at least for $r=1$. Specifically, consider a scheme
in which every query is independently
answered incorrectly with probability~$\epsilon$. Such a
scheme satisfies $\NYAdvCorrect{\Pi}{t,q} \leq \epsilon$ for any $t, q$,
however an adversary making $q=1/\epsilon$ queries has constant advantage
with respect to our correctness definition (for $r=1$).

In the other direction, we show:

\begin{theorem}
Fix $\Pi$. If $\AdvCorrect{\setprim,1}{t,q} \leq \epsilon$, then
$\NYAdvCorrect{\Pi}{t,q-1} \leq \epsilon$.
\end{theorem}
\begin{proof}
Assume
there is an adversary $A$ running in time~$t$
and making at most $q-1$ oracle queries such that
$\NYAdvCorrect{\Pi}{A} > \epsilon$.
Let $A'$ be the algorithm that simply
runs~$A$, forwarding the oracle queries of~$A$ to its own oracle, until~$A$ terminates
with output~$\qry$; then, $A'$ sends~$\qry$ to $\TestOracle$. It is immediate that $A'$ makes at most $q$ oracle queries, and
$\AdvCorrect{\Pi,1}{A'} \geq \NYAdvCorrect{\Pi}{A}$, a contradiction. \hfill\qed
\end{proof}

\begin{theorem}
There is a data structure $\Pi$ over ${\cal X}$
for which $\AdvCorrect{\setprim,2}{t,q}=0$ for all $t, q$,
but $\NYAdvCorrect{\Pi}{O(1),0} = 1$.
\end{theorem}
\begin{proof}
Let $x \in {\cal X}$ be
an arbitrary fixed element.
Define $\Pi$ such that $\Rep(S)$ outputs
$\pubaux=\privaux=S$, and
$\Qry(S, S, \qry_y)$ outputs $\qry_y(S)$ if $y \neq x$ but outputs $1-\qry_x(S)$ otherwise.
This scheme always answers incorrectly for a single, fixed query, and answers
correctly for every other query. The theorem follows.
\end{proof}


%%%%%%%%%%%%%%%%%%%%%%%%%%%%%%%%%%%%%%%%%%%%%%%%%%%%%
%%%%%%%%%%%%%%%%%%%%%%%%%%%%%%%%%%%%%%%%%%%%%%%%%%%%%
%%%%%%%%%%%%%%%%%%%%%%%%%%%%%%%%%%%%%%%%%%%%%%%%%%%%%
