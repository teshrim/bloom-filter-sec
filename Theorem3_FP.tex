\acnote{To be verified before moving the theorem to section 4}
\heading{Classical Bloom filter in random oracle model (ROM). }
As discussed earlier, the classical Bloom filter (BF) is an instantiation of a set multiplicity data structure and admits errors in form of false  positives (FP) only. In the ROM, $\Rep$ queries the random oracle to get the hash indices of elements of set $S$. It then computes the $m$-bit array, $M$ and outputs $\privaux=\emptystring$, $\pubaux = \emptystring$ and $M$ as the representation of
the Bloom filter. The $\Qry$ oracle uses the random oracle in a similar way. For any query $q$, it gets the hash indices $h_1, \ldots, h_k$ from the random oracle and returns 1 if $M[h_i(x)]=1$ for all $i\in [k]$, and 0 otherwise.
\begin{theorem}\label{thm3}
Fix $k,m,n,r>0$. Let $A$ be an adaptive adversary (for attacking BF) asking $q$ queries to the random oracle. Let $t$ be the running time of $A$ given these queries. Then, 
\[
\AdvCorrect{{BF},\distr{\calS}{},r}{A} \leq  {\dbinom{q}{r}} \left( (1-e^{-kn/m})^k + O(1/n) \right)^r\,.
\]
\end{theorem}

\begin{proof}[\Cref{thm3}]
The proof of this theorem will follow along the lines of \Cref{thm2}.
\end{proof}
