\acnote{INCOMPLETE. To be verified before moving the theorem to section 4}
\heading{Classical Bloom filter in random oracle model (ROM). }
Fix $n,k,m \geq 0$ and let $\mathcal{S}=[\univ]^n$.  Then $\Pi_{\mathrm{Bloom}}= (\calQ,\Rep, \Qry)$ is defined as in Figure~\ref{fig:bf-and-garbled-bf} (left side).  The following result shows,
\begin{theorem}\label{thm3}
Fix $k,m,n,r>0$, and let $\Pi_{\mathrm{Bloom}}= (\calQ,\Rep, \Qry)$ be the set-multiplicity data structure just described. For any adversary~$A$ that makes a total of~$q$ queries to the oracle and has time-complexity~$O(t)$, there exist prf-adversaries~$B_1,B_2$ (explicitly constructed in the proof of this theorem) such that
\[
\AdvCorrect{\Pi_{\mathrm{Bloom}},\distr{\calS}{},r}{A} \leq  \AdvPRF{F}{B_1} + \AdvPRF{F}{B_2}  +{\dbinom{q}{r}} \left( (1-e^{-kn/m})^k + O(1/n) \right)^r\,.
\]
\end{theorem}

\begin{proof}[\Cref{thm3}]
We use a game playing argument to prove this theorem in the ROM. In $\game{0}$, the $\HashOracle$ does lazy sampling of elements from $[m]$ to mimic a random oracle(RO). For every distinct query, $\Hash$ returns a random value sampled from $[m]$. Here, the adversary $A$, along with $\Rep$ and $\Qry$ are given access to $\Hash$. It can be seen that, $\game{0}$ exactly simulates the correctness game of $\Pi_{\mathrm{Bloom}}$. So, $\AdvCorrect{\Pi_{\mathrm{Bloom}},\distr{\calS}{},r}{A} = \Prob{\game{0}(A) = 1}$.

\begin{figure}
\fpage{.9}{
\hpagessl{.45}{.5}
{
\underline{\game{0}(A)}\\
$S \getsr \distr{\calS}{}$, $\err \gets 0$\\
$(\pubaux,\privaux) \getsr \Rep^{\HashOracle}(S)$\\
$z \getsr A^{\TestOracle, \HashOracle}(S,\pubaux)$\\
if $\err  < r$ then Return 0\\
Return 1\\
%
\oracle{$\TestOracle(q_x)$}\\
$a \gets \Qry^{\HashOracle}(\pubaux,\privaux,q_x)$\\
if $a \neq q_x(S)$ then \\
\nudge $\err \gets\err +1$\\
Return~$(a,\err )$\\
%
\oracle{$\HashOracle(i,q_y)$}\\
$v \getsr [m]$\\
if $T[i,q_y] \neq \undefined$\\
\nudge $v \gets T[i,q_y]$\\
$T[i,q_y] \gets v$\\
Ret $v$
}
{
\underline{$\game{1}(A)$}\\
$S \getsr \distr{\calS}{}$, $\err \gets 0$\\
$(\pubaux,\privaux) \getsr \Rep^{\HashOracle}(S)$\\
$z \getsr A^{\TestOracle, \HashOracle}(S,\pubaux)$\\
if $\err  < r$ then Return 0\\
Return 1\\ 
%
\oracle{$\TestOracle(q_x)$}\\
$M \gets\pubaux$\\
for $j \in \{1,2,\ldots,k\}$\\
\nudge $h_j \gets \HashOracle(j,q_x)$\\
\nudge if $M[h_j] \neq 1$ then Return $(0, \err)$\\
$\err \gets \err +1$\\
Return~$(1,\err )$\\
%
\oracle{$\HashOracle(i,q_y)$}\\
$v \getsr [m]$\\
if $T[i,q_y] \neq \undefined$\\
\nudge $v \gets T[i,q_y]$\\
$T[i,q_y] \gets v$\\
Ret $v$
}
}
\fpage{.9}{
\hpagessl{.45}{.5}
{
\underline{$\game{2}(A)$}\\
$S \getsr \distr{\calS}{}$, $\err \gets 0$\\
$(\pubaux,\privaux) \getsr \Rep^{\HashOracle}(S)$\\
$z \getsr A^{\TestOracle, \HashOracle}(S,\pubaux)$\\
if $\err  < r$ then Return 0\\
Return 1\\ 
%
\oracle{$\TestOracle(q_x)$}\\
$M \gets\pubaux$\\
for $j \in \{1,2,\ldots,k\}$\\
\nudge $h_j \gets \HashOracle(j,q_x)$\\
\nudge if $M[h_j] \neq 1$\\
\nudge \nudge if $j = 1$ then Return $(0, \err, \emptystring)$\\
\nudge \nudge Return $(0, \err, (h_1, \ldots, h_{j-1}))$\\
$\err \gets \err +1$\\
Return~$(1,\err, (h_1, \ldots, h_{k}) )$\\
%
\oracle{$\HashOracle(i,q_y)$}\\
$v \getsr [m]$\\
if $T[i,q_y] \neq \undefined$\\
\nudge $v \gets T[i,q_y]$\\
$T[i,q_y] \gets v$\\
Ret $v$
}
{
\underline{$\game{3}(A)$}\\
$S \getsr \distr{\calS}{}$, $\err \gets 0$\\
$(\pubaux,\privaux) \getsr \Rep^{\HashOracle}(S)$\\
$z \getsr A^{\TestOracle, \HashOracle}(S,\pubaux)$\\
if $\err  < r$ then Return 0\\
Return 1\\ 
%
\oracle{$\TestOracle(q_x)$}\\
$M \gets\pubaux $\\
$(h_1, \ldots, h_{k}) \gets \HashOracle(q_x)$\\
for $j \in \{1,2,\ldots,k\}$\\
\nudge if $M[h_j] \neq 1$ then \\
\nudge \nudge Return~$(0,\err, (h_1, \ldots, h_{k}) )$\\
$\err \gets \err +1$\\
Return~$(1,\err, (h_1, \ldots, h_{k}) )$\\
%
\oracle{$\HashOracle(q_y)$}\\
$v_j \getsr [m]$\\
for $j \in \{1,2,\ldots,k\}$\\
\nudge if $T[j,q_y] \neq \undefined$\\
\nudge \nudge $v_j \gets T[j,q_y]$\\
\nudge $T[j,q_y] \gets v_j$\\
Ret $\left(v_1,\ldots,v_k\right)$
}
}
\caption{\Cref{thm3}: Game playing argument}\label{fig:3TGame}
\end{figure}
\begin{figure}
\fpage{.9}{
\hpagessl{.45}{.5}
{
\underline{\game{4}(A)\fbox{\game{5}(A)}}\\
$S \getsr \distr{\calS}{}$\\
$\err \gets 0$, $\Ans \gets \emptyset$\\
$c \gets 0$, $\bad \gets \false$\\
$\mathcal{I}\getsr [\{1,2,\ldots,q\}]^r$\\
$(\pubaux,\privaux) \getsr \Rep^{\HashOracle}(S)$\\
$z \getsr A^{\TestOracle, \HashOracle}(S,\pubaux)$\\
if $\bad = \true$ then \fbox{Return 0}\\
if $\err  < r$ then Return 0\\
Return 1\\
%
\oracle{$\TestOracle(q_x)$}\\
if $\Ans[x] \neq \undefined$\\
\nudge $(h_1, \ldots, h_{k}) \gets \HashOracle(q_x)$\\
Return $\Ans[x]$\\
%
}
{
\oracle{$\HashOracle(q_y)$}\\
$c \gets c + 1$\\
$v_j \getsr [m]$\\
for $j \in \{1,2,\ldots,k\}$\\
\nudge if $T[j,x] \neq \undefined$\\
\nudge \nudge $v_j \gets T[j,q_y]$\\
\nudge $T[j,q_y] \gets h_j$\\
%
$M \gets\pubaux$\\
for $j \in \{1,2,\ldots,k\}$\\
\nudge if $M[v_j] \neq 1$ then \\
\nudge \nudge $\Ans[q_y] \gets (0,\err, (v_1, \ldots, v_{k}) )$\\
\nudge \nudge if $c \in \calI$ then $\bad \gets \true$\\
\nudge \nudge $\Break$\\
if $c \notin \calI$ then $\bad \gets \true$\\
$\err \gets \err +1$\\
$\Ans[q_y] \gets (0,\err, (v_1, \ldots, v_{k}))$\\
Ret $\left(v_1,\ldots,v_k\right)$
}
}
\caption{\Cref{thm3}: Game playing argument}\label{fig:3TGame}
\end{figure}

%The proof of this theorem will follow along the lines of \Cref{thm2}. As in previous proofs, we use a game playing argument to prove this theorem as well. In $\game{0}$, the $\HashOracle$ does lazy sampling of elements from $[m]$ to mimic a random oracle(RO). It can be observed that, $\game{0}$ exactly simulates the BF's correctness game attacked by an adaptive adversary. So, $\AdvCorrect{{BF},\distr{\calS}{},r}{A} = \Prob{\game{0}(A) = 1}$.
%
%In $\game{1}$, assuming $A$ does not make any pointless queries, $\HashOracle$ is modified to sample random elements from $[m]$. In addition, $\Test$ sets $\bad$ to $\true$ if queries indexed by set $\calI$ fail to be FPs, or if any of the remaining queries are FPs. $\game{2}$, on the other hand is exactly same as $\game{1}$ except it returns 0, if $\bad$ is set to $\true$. It can be easily seen that, $\game{1}$ and $\game{2}$ are exactly same as $\game{2}$ and $\game{3}$ of \Cref{thm2}'s game playing argument. So, 
%\begin{align*}
%\Prob{\game{1}(A)=1} = \Prob{\game{0}(A)=1}\\
%\Prob{\game{2}(A)=1} = \frac{1}{\dbinom{q}{r}}\Prob{\game{1}(A)=1}
%\end{align*}
%\Cref{fig:D} shows non-adaptive adversary, $D$ that finds $r$ FPs against the standard BF. Using a similar explanation from previous proofs,
%\begin{align*}
%\AdvCorrect{{BF},\distr{\calS}{},r}{D} &= \Prob{\game{2}(A) = 1}\\
%\AdvCorrect{{BF},\distr{\calS}{},r}{D} &\leq \left( (1-e^{-kn/m})^k + O(1/n) \right)^r \,.
%\end{align*}
%To summarize, we have
%\[
%\AdvCorrect{{BF},\distr{\calS}{},r}{A} \leq  {\dbinom{q}{r}} \left( (1-e^{-kn/m})^k + O(1/n) \right)^r\,.
%\]

\end{proof}
