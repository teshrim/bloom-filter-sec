Naor and Yogev~\cite{C:NaoYog15} recently considered the problem of
adversarial soundness for Bloom filters and, indeed, their work
served as inspiration for our own. Our work extends their work
significantly in several directions. For starters, we consider
abstract data structures (rather than only the case of
set-membership data structures), and consider privacy in addition to
correctness. Even with respect to the specific case of correctness
for set-membership data structures, our work offers several
advantages to the Naor-Yogev treatment:
\begin{itemize}
\item We explicitly distinguish between public and private
    portions of a data structure, something they leave
    ambiguous.\footnote{In fact, it isn't clear that their
formalization actually allows for standard Bloom filters,
    although fixing this ambiguity is straightforward once
    recognized.} This allows us, for example, to distinguish
    between the hash-function keys and the bit array when
    discussing classical Bloom filters.
\item The Naor-Yogev definition of soundness allows the
    adversary to make several set-membership queries, some of
    which may produce incorrect results, and the attacker then
    succeeds if it can output a \emph{fresh} query which leads
    to an error. We propose instead what we find to be a more
    natural definition in which the attacker succeeds if it can
    cause a certain number of incorrect results.
\item Naor and Yogev analyze a new Bloom-filter construction of
    their own design. In contrast, we are interested in
    analyzing practical constructions that have been proposed in
    prior work to understand their security. In particular, we
    analyze Bloom filters in the random-oracle model, something
    not considered by Naor and Yogev.
\end{itemize}

%We give a technical comparison of our correctness notion and theirs (lifted to our more general syntax)
%in Appendix~\ref{sec:compare-defs}.


We recommend reading the Naor-Yogev paper for a survey of related
work and a discussion of related theory papers. Here we mention a
few additional practical works, but stress that this only scratches
the surface.

\jnote{Rather random collection of papers using Bloom filters and
variants. I removed it for now, since it's not clear that they have
any particular relevance to us.}
%As previously mentioned, Bloom
%filters and their relatives are some of the most widely used data
%structures supporting set-membership queries. As examples, Hbase,
%the open-source implementation of Google's BigTable storage
%system~\cite{chang2008bigtable},
%%a Hadoop-based NoSql database designed to handle large datasets,
%includes an implementation of Bloom filters and counting Bloom
%filters, and the Squid proxy~\cite{fan2000summary} uses a Bloom
%filter as a ``summary'' of the set of URLs in its cache in order to
%improve latency for web-object retrieval. Reynolds and
%Vahdat~\cite{reynolds2003efficient} proposed an efficient
%distributed search engine that can be used to search for files
%containing a particular keyword. Their search engine maps the
%keywords of each file into a Bloom filter; a look-up of the keyword
%in the Bloom filter tells whether the node has files containing that
%keyword or not. Stochastic Fair Blue~\cite{feng2001stochastic} uses
%counting Bloom filter to manage non-responsive TCP traffic.

Lipton and Naughton~\cite{lipton1993clocked} showed how timing
analysis of record insertion in a hash table can help an adversary
to adaptively chose elements so as to increase the running time.
Crosby and Walach~\cite{crosby2003denial} exploited hash collisions
in Squid to increase the average URL load time.

Schnell et al.~\cite{schnell2011novel} proposed a way to support
privacy-preserving record linkage by using secretly keyed hash
functions. Subsequently, Niedermeyer et
al.~\cite{niedermeyer2014cryptanalysis} showed an attack on a
similar system when used on a database of German surnames.
\jnote{See Section~???}
